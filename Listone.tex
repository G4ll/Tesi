\documentclass[a4paper]{article}

\makeatletter
\title{Teoremi}\let\Title\@title
\author{Andrea Gallese}\let\Author\@author
\date{\today}\let\Date\@date

\usepackage{SetUp}

\begin{document}
	
	\section{Parte 1 -- Strumenti}
	
	\subsection{$ G $-moduli.}
	\begin{definition}{Modulo Indotto}
		\[ \Ind_H^G(A) = \Hom_H(G, \, A) = \{ f \colon G \to A \mid f(hg) = h f(g) \forall h \in H \} \]
		su cui agiamo a sinistra: $ g \cdot f(x) = f(xg) $. In realtà questa era la definizione di modulo coindotto, secondo Maffei! Il grande trucco è che useremo solo gruppi finiti, per i quali le definizioni coincidono
		\[ \Ind_H^G(A) = \Z[G] \otimes_{\Z[H]} A \]
	\end{definition}

	\subsection{La Coomologia}
	\begin{theorem}
		$ Mod_G $ ha abbastanza iniettivi.
	\end{theorem}

	Questo ci serve per poter definire i funtori derivati: prendiamo il funtore $ F \colon A \mapsto A^G $ esatto a sinistra e definiamo $ H^i(G, A) = R^iF(A) $, il suo derivato destro.
	
	\begin{theorem}[Proprietà fondamentali della coomologia]
		Abbiamo
		\begin{enumerate}
			\item $ \HH^0(G,\, A) = A^G $.
			\item Gli iniettivi sono aciclici.
			\item Per ogni successione esatta corta, ne abbiamo una esatta lunga in coomologia.
		\end{enumerate}
	\end{theorem}

	\begin{theorem}
		\[ \Hom_H(B,\, A) = \Hom_G(B, \, \Ind_H^G(A)) \]
	\end{theorem}

	\begin{theorem}
		\[ \HH^i(G, \, \lim_{\rightarrow} A_j) =  \lim_{\rightarrow}\HH^i(G, \, A_j) \]
	\end{theorem}

	\begin{theorem}
		\[ \HH^i(G, \, A) = Ext^i(\Z, \, A) \]
	\end{theorem}

	\subsection{Calcolo tramite cocatene}
	\begin{theorem}
		Otteniamo gli $ H^i $ come coomologia del complesso
		\[ K^0 \to K^1 \to K^2 \to \dots \]
		dove $ K^i = \{ f \colon G^i \to A \} $ e i differenziali sono i soliti.
	\end{theorem}

	\begin{theorem}
		Se $ G $ ed $ A $ sono finiti, pure i $ K^i $ sono finiti, dunque anche gli $ H^i. $
	\end{theorem}
	
	\subsection{Successione spettrale di Hochschild-Serre}
	Se ho una mappa $ f \colon G' \to G $ e uno $ G $-modulo $ A $, ho automaticamente un funtore $ f^\times $ che "lo vede" come $ G' $-modulo con
	\[ g' \cdot a = f(g') \cdot a. \]
	Quando applico il funtore $ F \colon A \mapsto A^G $ ottengo un'inclusione
	\[ A^G \hookrightarrow (f^\times A)^{G'}, \]
	che sarà in realta un morfismo di complessi tra risoluzione iniettive, che ci dona una mappa funtoriale $$  f_i^\star \colon H^i(G, \, ) \to H^i(G', \, f^\times ).  $$
	
	\begin{definition}[Restrizione]
		Chiamiamo $$  Res \colon H^i(G, \, A) \to H^i(H, \, A)  $$ la mappa indotta dall'inclusione naturale $ H \to G $.
	\end{definition}

	In generale, se abbiamo un $ G' $-morfismo $ u \colon f^\times A \to A'  $ otteniamo una mappa $ u^\star $ in coomologia, che possiamo comporre con $ f^\star $ tranquillamente per ottenere un morfismo funtoriale $ H^i(G, \, A) \to H^i(G', \, A') $.
	
	\begin{definition}[Inflazione]
		Considerando ora la proiezione canonica $ f \colon G \to \sfrac{G}{H} $ e l'inclusione tra $ \sfrac{G}{H} $-moduli $ u \colon A^H \to A $, otteniamo la mappa funtoriale \[ Inf \colon H^i(\sfrac{G}{H}, \, A^H) \to H^i(G, \, A). \]
	\end{definition}
	
	\begin{theorem}[Lemma di Shapiro]
		Prendendo l'inclusione $ f \colon H \to G $ e la mappa $ u \colon Ind_H^G(A) \to A $ che manda ogni morfirmo nel suo valore in 1, otteniamo degli \textbf{isomorfismi} \[ H^i(G, \, Ind_H^G(A)) \to H^i(H, \, A) \]
	\end{theorem}

	\begin{theorem}[Invarianza per automorfismi interni]
		Prendendo $ f = \sigma_t \colon G \to G $ un automorfismo interno e $ u \colon A \to A $ la moltiplicazione per $ t $ a sinistra, otteniamo degli isomorfismi in coomologia.
	\end{theorem}

	\begin{theorem}
		Dalla successione spettrale \[ E^{p, q}_2 = H^p(\sfrac{G}{H}, \, H^q(H, \, A)) \RR H^{p+q}(G,\, A) \] otteniamo due successioni esatte che useremo tantissimo
		\[ \begin{tikzcd}[column sep = small]
		0 \rar & H^1(\sfrac{G}{H}, A^H) \rar["Inf"] & H^1(G, \, A) \rar["Res"] & H^1(H, \, A)^{\sfrac{G}{H}} \rar & H^2(\sfrac{G}{H}, \, A^H) \rar["Inf"] & H^2(G, \, A)
		\end{tikzcd} \]
	\end{theorem}

	\begin{theorem}
		Se $ H^i(G, \, A) = 0 $ per ogni $ 0 < i < q $, abbiamo
		\[\begin{tikzcd}
		0 \rar & H^q(\sfrac{G}{H}, A^H) \rar["Inf"] & H^q(G, \, A) \rar["Res"] & H^q(H, \, A)^{\sfrac{G}{H}}.
		\end{tikzcd} \]
	\end{theorem}
	
	\subsection{Corestrizione}
	Definiamo una buffa norma
	\begin{align*}
		N_{\sfrac{G}{H}} : & A^H \to A^G \\
		& a \to \sum_{g \in \sfrac{G}{H} } g\cdot a,
	\end{align*}
	dopodiché prendo una risoluzione iniettiva di $ A \to I^\bullet $, che tramite $ f^\times $ diventa una risoluzione iniettiva di $ f^\times A $, ho così:
	\[ \begin{tikzcd}
	0 \rar & f^\times A^H \rar \dar["N"] & (f^\times I^\bullet) ^H \dar["N"] \\
	0 \rar & A^G \rar & (I^\bullet)^G
	\end{tikzcd} \]
	che ci dona un meraviglioso morfismo in coomologia che chiamiamo
	\[ coRes \colon H^i(H,\, A) \to H^i(G, \, A)  \]
	
	\begin{theorem}
		Quando componiamo $ CoR \circ Res $ otteniamo la moltiplicazione per l'indice del sottogruppo
		\[ \begin{tikzcd}
		H^i(G, \, A) \rar["Res"] \arrow["\cdot {\left[G \colon H \right]}" description, bend right = 20]{rr} & H^i(H, \, A)\rar["CoRes"] & H^i(G, \, A)
		\end{tikzcd} \]
	\end{theorem}

	Questo è piuttosto piacevole! Quando considero il sottogruppo banale $ H = \{ e \} $, scopro che
	
	\begin{theorem}
		I gruppi $ H^i(G, \, A) $ sono di $ \mid G \mid  $-torsione per $ i > 0. $
	\end{theorem}

	Da cui, passando nuovamente per la descrizione in catene, scopriamo che
	\begin{theorem}
		Se $ G $ è finito e $ A $ è finitamente generato, allora $ H^i $ è finito (perché finitamente generato e di torsione).
	\end{theorem}

	\subsection{Gruppi Modificati di Tate}
	Il funtore $ A \mapsto \sfrac{A}{I_GA} $  è esatto a destra, posso quindi prenderne il funtore derivato a sinistra, che mi definisce l'omologia $ H_i. $ A questo punto riallaccio il tutto:
	\[ \begin{tikzcd}
	0 \rar &  \widehat{H}^{-1} \rar & H_0 \rar["N"] & H^0 \rar & \widehat{H}^0 \rar & 0
	\end{tikzcd} \]
	
	Mi ricavo facile che
	\begin{theorem}
		\[ H_1(G, \, \Z) \cong G^{ab} \]
	\end{theorem}

	\subsection{Coomologia dei Gruppi Ciclici}
	Il risultato fondamentale è che la coomologia è semplice: $ \widehat{H}^i \cong \widehat{H}^{i+2} $.
	
	\begin{definition}[Queziente d'Hebrand]
		Vuole essere il magico numero associato alla coomologia
		\[ h(A) = \frac{h^0(A)}{h^1(A)}. \]
	\end{definition}	

	\begin{theorem}[Proprietà del Quoziente d'Hebrand]
		Abbiamo che
		\begin{enumerate}
			\item Se ho $ 0 \to A \to B \to C \to 0 $ esatta e due quozienti definiti, è definito anche il terzo
			\item Se $ A $ è finito, allora $ h(A) = 1 $.
			\item Se ho $ f \colon A \to B $ con sia $ \ker $ che $ coker$ finiti e uno dei due $ h $ deifnito, è definito anche l'altro e sono uguali.
		\end{enumerate}
	\end{theorem}

	\subsection{Tazza Prodotto}
	
	\newpage
	\section{ Parte 2 -- Cose}
	
	\subsection{Numeri Primi}
	\begin{lemma}
		$ G $ un $ p $-gruppo, $ A $ di torsione $ p $-primario. $ A^G = 0 \RR A = 0. $ [Formula delle Classi]
	\end{lemma}
	
	\begin{lemma}
		Prendo $ H < G $ con $ p \nmid [G \colon H ] $, la nostra amica $ Res $ è iniettiva sulla componenti $ p $-primarie.
	\end{lemma}

	\begin{lemma}
		$ B $ indotto, allora $ \Hom_\Z(A, \, B) $ è indotto.
	\end{lemma}
	
	\begin{definition}
		Diciamo che un modulo $ A $ è coomologicamente banale se $ \HH^n(H, \, A) = 0 $ per ogni $ H < G $ e $ n > 0. $
	\end{definition}

	Qui c'è da fare attenzione: se cambio Sylow non cambia nulla, perché nel capitolo della successione spettrale c'è scritto che gli automorfismi interni di $ G $ mi danno gruppi di coomologia isomorfi.
	
	\begin{lemma}
		$ A $ è $ G$-c.b. se e solo se è $ G_p$-c.b. per ogni $ p. $
	\end{lemma}

	\begin{theorem}
		$ G $ un $ p $-gruppo, $ A $ di $ p $-torsione, esiste $ q $ per cui $ \Hh^q(G, \, A) = 0 $, allora $ A $ è \textbf{indotto}! ($ \F_p[G] $-libero, in effetti)
	\end{theorem}

	\begin{theorem}
		Se per ogni primo trovo $ q $ per cui $$  \Hh ^q(G_p,\, A) = \Hh^{q+1}(G_p,\, A) = 0  $$ allora $ A $ è indotto
	\end{theorem}

	\subsection{Dualità di Tate-Nakayama}
	
\end{document}