\documentclass[a4paper]{article}

\makeatletter
\title{Teoremi}\let\Title\@title
\author{Andrea Gallese}\let\Author\@author
\date{\today}\let\Date\@date

\usepackage{SetUp}

\begin{document}
	\section{Parte 1 -- Strumenti}
	
	\subsection{$ G $-moduli.}
	\begin{definition}{Modulo Indotto}
		\[ \Ind_H^G(A) = \Hom_H(G, \, A) = \{ f \colon G \to A \mid f(hg) = h f(g) \forall h \in H \} \]
		su cui agiamo a sinistra: $ g \cdot f(x) = f(xg) $. In realtà questa era la definizione di modulo coindotto, secondo Maffei! Il grande trucco è che useremo solo gruppi finiti, per i quali le definizioni coincidono
		\[ \Ind_H^G(A) = \Z[G] \otimes_{\Z[H]} A \]
	\end{definition}

	\subsection{La Coomologia}
	\begin{lemma}
		$ Mod_G $ ha abbastanza iniettivi.
	\end{lemma}

	Questo ci serve per poter definire la coomologia passando per le risoluzioni iniettive: prendiamo il funtore $ F \colon A \mapsto A^G $ e definiamo $ H^i(G, A) = R^iF(A) $.
	
	\begin{theorem}[Proprietà fondamentali della coomologia]
		Abbiamo
		\begin{enumerate}
			\item $ H^0(G,\, A) = A^G $.
			\item Gli iniettivi sono aciclici.
			\item Per ogni successione esatta corta, ne abbiamo una esatta lunga in coomologia.
		\end{enumerate}
	\end{theorem}

	\begin{lemma}
		\[ \Hom_H(B,\, A) = \Hom_G(B, \, \Ind_H^G(A)) \]
	\end{lemma}
	
	
\end{document}