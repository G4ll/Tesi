\documentclass{beamer}
\usetheme{Boadilla}
\useinnertheme{rectangles}

\usepackage[utf8]{inputenc}
\usepackage[italian]{babel}
%\usepackage[usenames,dvipsnames]{color}
\usepackage{xcolor}
\usepackage{amsmath}%
\usepackage{amsfonts}%
\usepackage{amssymb}%
\usepackage{amsthm}%
\usepackage{graphicx}
\usepackage{tikz}
\usepackage{tikz-cd}

%Comandi specifici
\newcommand{\leftquote}{\textquotedblleft}
\newcommand{\N}{\mathbb{N}}
\newcommand{\Z}{\mathbb{Z}}
\newcommand{\Q}{\mathbb{Q}}
\newcommand{\Qp}{\mathbb{Q}_p}
\newcommand{\R}{\mathbb{R}}
\newcommand{\F}{\mathbb{F}}

% Galois
\newcommand{\K}{K}
\newcommand{\knr}{K_\mathtt{nr}}
\newcommand{\knrr}{{K^\mathtt{nr}_n}}
\newcommand{\f}{f^\times}
\newcommand{\Gal}[1]{\mathcal{G}al\left( #1 \right)}

\DeclareMathOperator{\inv}{inv}
\DeclareMathOperator{\Br}{Br}

% Omologica
\newcommand{\Aut}[1]{\mathrm{Aut}\left( #1 \right)}

\DeclareMathOperator{\Hom}{Hom}

\DeclareMathOperator{\HH}{H}
\DeclareMathOperator{\Hh}{\widehat{H}^\mathnormal{i}}
\DeclareMathOperator{\Hhh}{\widehat{H}}
\DeclareMathOperator{\Hha}{\widehat{H}^\mathnormal{-i}}
\DeclareMathOperator{\Hhb}{\widehat{H}^{\mathnormal{i}+1}}
\DeclareMathOperator{\Hhc}{\widehat{H}^{\mathnormal{i+r}}}
\DeclareMathOperator{\Hhid}{\widehat{H}^{\mathnormal{i}+2}}

\DeclareMathOperator{\Hhq}{\widehat{H}^\mathnormal{q}}
\DeclareMathOperator{\Hhqu}{\widehat{H}^{\mathnormal{q}+1}}
\DeclareMathOperator{\Hhqd}{\widehat{H}^{\mathnormal{q}+2}}
\DeclareMathOperator{\Hhdq}{\widehat{H}^{2-\mathnormal{q}}}

\DeclareMathOperator{\Hhmm}{\widehat{H}^{-2}}
\DeclareMathOperator{\Hhm}{\widehat{H}^{-1}}
\DeclareMathOperator{\Hhz}{\widehat{H}^{0}}
\DeclareMathOperator{\Hhu}{\widehat{H}^{1}}
\DeclareMathOperator{\Hhd}{\widehat{H}^{2}}
\DeclareMathOperator{\Hht}{\widehat{H}^{3}}

\DeclareMathOperator{\Hhnu}{\widehat{H}^{\mathnormal{n}+1}}
\DeclareMathOperator{\Hhnd}{\widehat{H}^{\mathnormal{n}+2}}

\DeclareMathOperator{\Hhnru}{\widehat{H}^{\mathnormal{n}+1+\mathnormal{r}}}
\DeclareMathOperator{\Hhnr}{\widehat{H}^{\mathnormal{n}+\mathnormal{r}}}

\DeclareMathOperator{\Hhn}{\widehat{H}^{\mathnormal{n}}}
\DeclareMathOperator{\Hhnq}{\widehat{H}^{\mathnormal{n}+\mathnormal{q}}}

\DeclareMathOperator{\Hhnp}{\widehat{H}^{\mathnormal{n_p}}}
\DeclareMathOperator{\Hhnpp}{\widehat{H}^{\mathnormal{n_p}+1}}
\DeclareMathOperator{\Hhnppp}{\widehat{H}^{\mathnormal{n_p}+2}}

\DeclareMathOperator{\Hhnpq}{\widehat{H}^{\mathnormal{n_p}+\mathnormal{q}}}
\DeclareMathOperator{\Hhnppq}{\widehat{H}^{\mathnormal{n_p}+\mathnormal{q}+1}}
\DeclareMathOperator{\Hhnpppq}{\widehat{H}^{\mathnormal{n_p}+\mathnormal{q}+2}}


\DeclareMathOperator{\Ind}{Ind}
\DeclareMathOperator{\coInd}{coInd}
\DeclareMathOperator{\Res}{\mathtt{res}}
\DeclareMathOperator{\Cor}{\mathtt{cor}}
\DeclareMathOperator{\Inf}{\mathtt{inf}}

\DeclareMathOperator{\Ext}{Ext}
\DeclareMathOperator{\Tor}{Tor}

\DeclareMathOperator{\im}{im}
\DeclareMathOperator{\id}{id}
\DeclareMathOperator{\cd}{cd}
\DeclareMathOperator{\coker}{coker}

% Cose da decidere
\newcommand{\Gmod}{\mathsf{Mod}_\mathsf{G}}
\newcommand{\Hmod}{\mathsf{Mod}_\mathsf{H}}

\newcommand{\HS}{Hochschild-Serre}


%\renewcommand{\varinjlim}{\textstyle\varinjlim}
%\renewcommand{\varprojlim}{\textstyle\varprojlim}


\title{Teoria del Campo di Classe Locale}
\author{Andrea Gallese}
\date{7 luglio 2019}

% Cosa vogliamo raccontare??

\begin{document}
	\begin{frame}
	\titlepage
\end{frame}

% Cos'è la teoria del campo di classe e perché vogliamo farla:
% Vogliamo studiare le estensioni abeliane di un campo, un buon risultato sarebbe per esempio ottenere una descrizione maneggevole del gruppo di Galois della massima estenione abeliana! Non è chiarissimo cosa sia una buona descrizione di questo gruppo... l'idea della CFT è di descriverlo in termini del campo stesso. Vorremmo farlo su Q (campo globale) ma è complicato, così si comincia a farlo nei suoi completamenti metrici (campi locali) e dopodiché si ricostruisce il risultato per Q.

\begin{frame}[fragile]{Qual è l'obiettivo?}
% Fissiamo un campo \only<1>{$k$}\only<2->{$K$ locale}, una chiusura separabile \only<1>{$\bar{k}$}\only<2->{$\bar{K}$} e il gruppo di Galois assoluto \only<1>{$\Gamma = \Gal{\bar{k}/k}$}\only<2->{$\Gamma = \Gal{\bar{K}/K}$}.

\begin{itemize}
	\item<1-> Studiare le estensioni abeliane di un campo {$k$}.
	\item<5-> Descrivere il gruppo di Galois di un'estensione abeliana in termini della struttura aritmetica del campo stesso.
\end{itemize}

\begin{example}<2->
	Sia $ F $ l'estensione di $ \Q $, composizione di tutte le estensioni quadratiche:
	\[\begin{tikzcd}[row sep = small, column sep = tiny]
	&F&& \\
	\Q(i) 		  \arrow[dash]{ur}  \arrow[dash]{dr}
	&\Q(\sqrt{2}) \arrow[dash]{u}   \arrow[dash]{d}
	&\Q(\sqrt{3}) \arrow[dash]{ul}  \arrow[dash]{dl}
	&\dots 		  \arrow[dash]{ull} \arrow[dash]{dll} \\
	&\Q&&	\end{tikzcd}  \]
	
	Chi è il gruppo di Galois dell'estensione? $\only<3>{\prod \Z/2\Z.} \only<4->{ \Q^\times/ {\Q^\times}^2.} $
\end{example}

%L'objet de la théorie du corps de classes est de montrer comment un corps possède en soi les éléments de son propre dépassement. -Chevalley
\end{frame}

\begin{frame}{Teoria del campo di classe {locale}}
% Fissiamo un campo \only<1>{$k$}\only<2->{$K$ locale}, una chiusura separabile \only<1>{$\bar{k}$}\only<2->{$\bar{K}$} e il gruppo di Galois assoluto \only<1>{$\Gamma = \Gal{\bar{k}/k}$}\only<2->{$\Gamma = \Gal{\bar{K}/K}$}.

\begin{itemize}
	\item<1-> Studiare le estensioni abeliane di un campo {locale $K$}.
	\item<1-> Descrivere il gruppo di Galois di un'estensione abeliana in termini della struttura aritmetica del campo stesso.
\end{itemize}

Sia $ K $ un campo locale, fissiamo una chiusura separabile $ \bar{K} $ e chiamiamo $ \Gamma $ il relativo gruppo di Galois assoluto.

%L'objet de la théorie du corps de classes est de montrer comment un corps possède en soi les éléments de son propre dépassement. -Chevalley

{\begin{theorem}<3->
		Il gruppo di Galois dell'estensione abeliana massimale $\Gamma^\mathtt{ab}$ di un campo locale $K$ è isomorfo al completamento profinito del gruppo moltiplicativo del campo stesso:
		\[  \Gamma^\mathtt{ab} \simeq \hat{K}^\times. \]
\end{theorem}}
\end{frame}

% Come procediamo?
% Lo strumento più moderno per affrontare problemi in Teoria di Galois è la coomologia di gruppi, i quali si ineriscono naturalemnte nel discorso permettendoci di capire in che modo il funtore degli invarianti non è esatto: esempio? 
\section{Coomologia}
\begin{frame}{Come procederemo?}
Sia $ \Gmod $ la categoria degli $ \Z[G] $-moduli.
% Hi(G,.) è il derivato destro del funtore (esatto a sinistra) che prende gli elementi invarianti per l'azione del gruppo G.
\begin{align*} 
\mathcal{F} \colon \Gmod &\to \mathcal{A}b \\
A &\mapsto A^G = \{ a \in A \mid ga = a \; \forall \, g \in G \},
\end{align*}

\begin{definition}
	I gruppi di coomologia sono i derivati destri di $ \mathcal{F} $:
	\[ \HH^i(G, \, \bullet\,) : = R^i\mathcal{F}(\,\bullet\,). \]
\end{definition}


% Se G è il gruppo di Galois di un'estensione L/K, ci sono dei coefficienti naturali su cui calcolare la coomologia: L e L*.
\begin{itemize}
	\item Calcoleremo la coomologia di $ \Gamma $ con opportuni coefficienti,
	riconducendoci alle sottoestensioni finite:
\end{itemize}
\[ \HH^n \left(\Gamma, \, A \right)  = \varinjlim_{U < \Gamma} \HH^n(\Gamma/U, \, A^U). \]

\end{frame}

\begin{frame}[fragile]{Coomologia di Gruppi}
\begin{theorem}\label{fond}
Data una successione esatta corta di $ G $ moduli
\[\begin{tikzcd}[column sep = small]
0 \rar & A \rar & B \rar & C \rar & 0,
\end{tikzcd}\]
si ha una successione esatta lunga di gruppi abeliani
\[\begin{tikzcd}[column sep = small, row sep = small]
0 \rar
&A^G \rar\arrow[d,phantom, ""{coordinate, name=A, near end}]
& B^G \arrow[d,phantom, ""{coordinate, name=B}] \rar
& C^G \arrow[dll, 
rounded corners = 5 pt, 
to path={ --([xshift=3ex]\tikztostart.east)
	|- (A)\tikztonodes
	-| ([xshift=-3ex]\tikztotarget.west)
	-- (\tikztotarget)}]  \\
\qquad \quad
&\HH^1(G, \, A)  \arrow[d,phantom, ""{coordinate, name=X}] \rar 
& \HH^1(G, \, B) \arrow[d,phantom, ""{coordinate, name=Y}] \rar
& \HH^1(G, \, C) \arrow[d,phantom, ""{coordinate, name=Z}] \arrow[dll, 
rounded corners = 5 pt, 
to path={ --([xshift=3ex]\tikztostart.east)
	|- (X)\tikztonodes
	-| ([xshift=-3ex]\tikztotarget.west)
	-- (\tikztotarget)}]  \\
&\HH^2(G, \, A) \rar
& \dots
& \qquad \end{tikzcd}\]
\end{theorem}
\end{frame}

% Useremo questi invarianti come strumento per studiare i gruppi di Galois: fissiamo un'estensione L/K e andiamo a calcolarci H(G, L) e H(G, L*)... i primi sono banali (Teorema della Base Normale) e gli altri sono banali in 1 (Hilbert 90) e via via sempre più interessanti. Alla fine ci servirà un modo di risalire dalla coomologia al gruppo.

\begin{frame}{Coomologia di Galois}
\begin{theorem} \label{Hadd}
Sia $ L / k $ un'estensione di Galois. Il gruppo additivo di $ L $ è coomologicamente banale.
\end{theorem}
\begin{theorem}[Hilbert 90]\label{H90}
Sia $ L/k $ un'estensione di Galois. Si ha
\[ \HH^1(\Gal{ L / k}, \, L^\times) = 0. \]
\end{theorem}
Brauer??
\end{frame}

\section{Campi Locali}
% Perché i campi locali?
% Per campo locale intendiamo un campo completo rispetto a una valutazione discreta di rango uno: questi hanno una piacevola struttura che permette un calcolo agevole degli Hi. I due riultati fondamentali sono che BrK = Q/Z e che la dimensione coomologica è 2.

\begin{frame}{Cos'é un campo locale?}
\Huge
\[ \Qp \]
\end{frame}

\begin{frame}[fragile]{Cos'é un campo locale?}
	\begin{definition}
		Un campo completo rispetto a una valutazione discreta di rango 1.
	\end{definition}

% perché ci piacciono
	\[ \begin{tikzcd}[row sep = small]
	K \rar & \mathcal{O}_K \rar & \pi_K \rar & \kappa \\
	L \rar & \mathcal{O}_L \rar & \pi_L^e \rar& \lambda
	\end{tikzcd} \]
	

	
% conto
	\[ \begin{tikzcd}[column sep = small]
	\inv_\K \colon \HH^2(G, \, \knr^\times)  \rar["v^*"]
	& \HH^2(G, \, \Z) \rar["\delta"]
	& \HH^1(G, \, \Q/\Z) \rar["\gamma"]
	& \Q/\Z.
	\end{tikzcd} \]
\end{frame}

\begin{frame}[fragile]
	% le estensioni non ramificate ci piacciono particolarmente
	\[ \begin{tikzcd}[row sep = tiny, column sep = tiny]
	&&K^\mathtt{nr}&& \\
	&\vdots&\vdots&\vdots& \\
	&&&K^\mathtt{nr}_{n+1}\arrow[dash]{dddl}\arrow[dash]{ul}& \\
	K^\mathtt{nr}_n\arrow[dash]{ddrr}\arrow[dash]{uur}\arrow[dash]{uurrr}&&&&\\
	&&&&K^\mathtt{nr}_{n-1}\arrow[dash]{dll}\arrow[dash]{uuul} \\
	&& K &&
	\end{tikzcd}
	
	\begin{tikzcd}[row sep = tiny, column sep = tiny]
	&&\bar{\F}_q&& \\
	&&&& \\
	&&&& \\
	\F_{q^n} \arrow[dash]{drr}&&&& \\
	&& \F_q &&
	\end{tikzcd}\]
\end{frame}

\begin{frame}[fragile]{Campi Locali}

Per campo locale intendiamo  Si può pensare, senza perdita di generalità, a estensioni finite dei $ p $-adici $ \Q_p $ o delle serie Laurent su campi finiti $ \F_p((t)) $.

\begin{equation}\label{local1}
\begin{tikzcd}[column sep = small]
1 \rar
& \mathcal{O}_{\K}^\times \rar
& \K^\times \rar["v"]
& \Z \rar
& 0.
\end{tikzcd}
\end{equation}

\begin{equation}\label{local3}
\begin{tikzcd}[column sep = small]
1 \rar
& U_K^{1} \rar
& \mathcal{O}_{\K}^\times \rar
& \kappa^\times \rar
&1,
\end{tikzcd}
\end{equation}


\begin{equation}\label{local2}
\begin{tikzcd}[column sep = small]
1 \rar
& U_K^{i+1} \rar
& U_{K}^i \rar
& \kappa \rar
& 0.
\end{tikzcd}
\end{equation}

$ \Br K = \Q / \Z. $

\begin{theorem}\label{cdim2}
La dimensione coomologica di un campo locale è 2.
\end{theorem}

\end{frame}

% Che cosa ce ne facciamo?
% Introduciamo il cup prodotto. Sotto ipotesi ragionevoli otteniamo Tate-Nakayama, che ci permette di arrivare alla reciprocità locale, che è praticamente quello che volevamo: l'ultimo passo è mostrare che il limite buffo è in realtà il limite del completamento profinito, che forse vorremmo dire cos'è. Infine presentiamo la dualità di Tate, con cui conludiamo il risultato nel caso p-adico.

\begin{frame}{Prodotto Tazza}
\[ \cup \colon \HH^p(G, \, A) \times \HH^q(G, \, B) \to \HH^{p+q}(G, \, A \otimes B)  \]


\end{frame}

\begin{frame}
\begin{theorem}[Reciprocità Locale]
Ogni estensione abeliana finita $ L/K $ è accompagnata da un isomorfismo
\[ \omega_L\colon \frac{K^\times}{NL^\times} \to \Gal{L/K}, \]
dove $ N $ è, come al solito, la norma.
\end{theorem}

\begin{equation}\label{lim}
\Gamma_K^{\texttt{ab}} = \varprojlim_L \frac{K^\times}{NL^\times}.
\end{equation}

\end{frame}
\begin{frame}
\begin{theorem}[Dualità di Tate]
Sia $ K $ un campo $ p $-adico e $ M $ un $ \Gamma $-modulo finito. Il tazza-prodotto induce un accoppiamento di dualità tra gruppi finiti
\[ \HH^i(\Gamma, \, A) \times \HH^{2-i}(\Gamma, \, A^*) \to \HH^2(\Gamma, \, \bar{K}^\times) = \Q / \Z \]
per $ i = 0, \, 1, \, 2 $.
\end{theorem}

\end{frame}

% Slide conclusiva

\begin{frame}{The End}

\begin{theorem}
$ \Gamma^\mathtt{ab} $ è isomorfo al completamento profinito $ \hat{K}^\times $.
\end{theorem}

\begin{align*}
\Gamma^\mathtt{ab}
& = \Hom_\mathsf{Gr}(\Gamma,\, \Q/\Z)^\vee \\[0.1em]
& = \Hom_\mathsf{Gr}(\Gamma,\, \textstyle\varinjlim_n \Z/n\Z)^\vee \\ 
& = \textstyle\varprojlim_n \Hom_\mathsf{Gr}(\Gamma,\, \Z/n\Z)^\vee \\
& = \textstyle\varprojlim_n \HH^1(\Gamma,\, \Z/n\Z)^\vee \\
& = \textstyle\varprojlim_n \HH^1(\Gamma,\, \mu_n) \\
& = \textstyle\varprojlim_n \K^\times / {\K^\times}^n \\
& = \hat{K}^\times.
\end{align*}

\end{frame}

\end{document}


