\chapter{Primi risultati}
In questo capitolo giocheremo con il nostro nuovo strumento, esplorandone le potenzialità e gettando delle solide basi sui cui poggeranno i calcoli successivi. Il primo obbiettivo sarà studiare la coomologia dei gruppi ciclici, che si rivelerà piuttosto semplice e di evidente importanza in tutto il seguito. Segue immediatamente un tentativo di studiare i gruppi per approssimazione: partendo da quelli ciclici, passando per i $ p $-gruppi, per poi affrontare gruppi finiti qualsivoglia. Sarà dunque importante capire come ricondurre il calcolo operativo della coomologia di un gruppo a quello dei suoi Sylow. Nel farlo, produrremo un criterio per stabilire quando i gruppi di Tate di un modulo sono tutti banali. \\

Sistemati i giocattoli, sarà il momento di concentrarci sul vero obbiettivo del nostro lavoro: i gruppi di Galois. Per affrontare la Teoria di Galois avremo bisogno di abbandonare l'ipotesi di finitezza, producendo qualche risultato anche per gruppi profiniti. Introdurremo dunque la nozione di dimensione coomologica, come indice di complessità di un gruppo, di particolare interesse nella caratterizzazione dei gruppi infiniti.

\section{Gruppi Ciclici}
Sia nel seguito $ G $ un gruppo ciclico di ordine $ n $ di cui fissiamo un generatore $ \sigma $. La coomologia di $ G $ è particolarmente semplice da descrivere: è sufficiente calcolare i gruppi di Tate in grado $ i = 0, \, -1 $.

\begin{theorem}\label{ciclici}
	Siano $ G $ un gruppo ciclico finito e $ A $ uno $ G $-modulo. I gruppi di Tate sono 2-periodici nel grado:
	\[ \Hhid(G, \, A) = \Hh(G, \, A). \]
\end{theorem}

\begin{proof}
	Siano $ N = 1 + \sigma + \sigma^2 + \dots + \sigma^{n-1} $ e $ D = 1 - \sigma $ due elementi di $ \Z[G] $. Il teorema si basa interamente sullo scrivere la risoluzione proiettiva
	\[ \begin{tikzcd}[column sep = small]
	\dots \rar["N"]
	& \Z[G] \rar["D"]
	& \Z[G] \rar["N"]
	& \Z[G] \rar["D"]
	& \Z[G] \rar
	& \Z \rar & 0,
	\end{tikzcd} \]
	da cui segue la 2-periodicità sia in coomologia che in omologia. Un elegante metodo per estendere la periodicità ai gradi $ i = 0, \, -1 $ si ottiene considerando i moduli con coomologia traslata: possiamo così spostare i gruppi problematici, per dire, in grado positivo. 
\end{proof}

Per i moduli i cui gruppi di Tate sono finiti abbiamo scoperto due importanti invarianti: la cardinalità dei gruppi in grado pari e dei gruppi in grado dispari, che chiamiamo $ h_0(A) $ e $ h_1(A) $.

\begin{definition}(Quoziente di Herbrand)
	Se $ A $ ha gruppi di Tate finiti chiamiamo \emph{quoziente di Herbrand} il numero naturale
	\[ {h}(A) := \frac{h_0(A)}{h_1(A)} = \frac{|\Hhz(G, \, A)|}{|\Hhu(G, \, A)|}. \]
\end{definition}

Enunciamo ora alcune piacevoli proprietà.

\begin{lemma}\label{Herb1}
	Data una successione esatta corta
	\[ 0 \to A \to B \to C \to 0, \]
	se il quoziente $ h $ è definito per due moduli su tre, è definito anche per il terzo. In questo caso vale $ h(A)h(C) = h(B) $.
\end{lemma}
\begin{proof}
	La successione esatta lunghissima associata può essere riassunta in una successione \leftquote esagonalmente esatta"
	\[\begin{tikzcd}[column sep={1cm,between origins}, row sep={1.732050808cm,between origins}]
	& \Hhz(G, \, A) \arrow[rr] && \Hhz(G, \, B) \arrow[rd] &  \\
	\Hhu(G, \, C) \arrow[ru]&  &&  & \Hhz(G, \, C) \arrow[ld] \\				& \Hhu(G, \, B) \arrow[lu] && \Hhu(G, \, A) \arrow[ll], & 
	\end{tikzcd}\]
	che possiamo slacciare in un punto a piacere
	\[ 0 \to Q \to \Hhz(G, \, A) \to \Hhz(G, \, B) \to \dots \to \Hhu(G, \, C) \to Q \to 0, \]
	aggiungendo un opportuno modulo $$  Q = \ker \left[\Hhz(G, \, A) \to \Hhz(G, \, B)\right] = \coker\left[ \Hhu(G, \, B) \to \Hhu(G, \, C) \right].  $$
	Dobbiamo pertanto avere, per esattezza, che
	\[  h_0(B) h_1(A) h_1(C) \cdot |\,Q\,| = h_0(A) h_0(C) h_1(B) \cdot |\,Q\,|, \]			da cui la tesi.
\end{proof}

\begin{lemma}\label{Herb2}
	Se $ A $ è finito, $ h(A) = 1 $.
\end{lemma}
\begin{proof}
	Nella definizione di gruppi di Tate compariva la successione esatta
	\begin{equation*}			\begin{tikzcd}[column sep = small]
	0 \rar& \Hh^{-1}(G, \, A) \rar& A_G \rar["N"]& A^G \rar&\Hh^0(G, \, A) \rar& 0,
	\end{tikzcd}
	\end{equation*}
	da cui deduciamo che $ h_1(A) \cdot |\, A^G\, | =  h_0(A) \cdot |\, A_G\, | $; siccome il gruppo è ciclico, abbiamo anche la desiderata relazione fra invarianti e co-invarianti:
	\begin{equation*}
	\begin{tikzcd}[column sep = small]
	0 \rar& A_G \rar& A \rar["D"]& A \rar& A^G \rar& 0.
	\end{tikzcd}
	\end{equation*}
	Grazie alle cui relazioni sugli ordini dei gruppi concludiamo.
\end{proof}


%\begin{lemma}\label{Herb3}
%	Se $ f \colon A \to B $ ha nucleo e conucleo finiti e uno tra $ h(A) $ e $h(B) $ è definito, allora lo è anche l'altro e sono uguali.
%\end{lemma}
%
%\begin{proof}
%	È\ sufficiente scrivere le ipotesi in forma di successioni esatte
%	\[ 0 \to \ker f \to A \to Q \to 0 \qquad\text{ e } \qquad 0 \to Q \to B \to \coker f \to 0, \]
%	applicare il primo punto per dedurre la finitezza dei gruppi di Tate, il secondo per ottenere l'uguaglianza desiderata.
%\end{proof}
%
%Quest'ultimo lemma sembra, più che una proprietà elementare del quoziente di Herbrand, un risultato tecnico e particolarmente specifico: è così, ci servirà più avanti in una dimostrazione particolarmente impegnativa.


\section{Banalità Coomologica}
Lo studio della coomologia di un generico gruppo si fa troppo complessa per poter sperare di enunciare teoremi di portata analoga a quelli validi per i gruppi ciclici. Si riesce però a dire qualcosa sui moduli che hanno tutti i gruppi di coomologia banali.

\begin{definition}
	Diciamo che uno $ G $ modulo $ A $ è \emph{coomologicamente banale} se
	$$  \HH^i(H, \, A) = 0 \qquad \forall \, i > 0, \quad  \forall \, H < G.  $$
\end{definition}

Il lettore più accorto potrebbe trovarsi perplesso da questa scelta: perché, dopo i generosi risultati dei capitoli precedenti, abbiamo abbandonato i gruppi di Tate? Scopriremo nel seguito che le due definizioni sono equivalenti. \\

Abbiamo già mostrato che i moduli indotti sono coomologicamente banali (Lemma \ref{indotti}). In particolare i moduli liberi sono coomologicamente banali e, di conseguenza, i moduli proiettivi (ragionando per décalage). Anche i moduli iniettivi sono coomologicamente banali, per come abbiamo costruito i gruppi di coomologia (\ref{injban}). \\

In generale non è facile risalire dalla coomologia dei Sylow a quella del gruppo originario, una delle ragioni per cui siamo interessati ai moduli coomologicamente banali è che si riesce a spezzarne lo studio in parti più semplici. Per ogni primo $ p $ fissiamo un $ p $-Sylow $ G_p $ di $ G $ e studiamone la coomologia; la scelta del $ p $-Sylow non intacca la generalità dello studio, infatti il gruppo $ \Hh(G_p, \, A) $ non dipende dalla scelta del Sylow: ogni automorfismo interno $ \sigma_t $ di $ G $ induce infatti un isomorfismo in coomologia. 

\begin{lemma}\label{banSylow}
	Uno $ G $-modulo $ A $ è coomologicamente banale se e solo se è coomologicamente banale come $ G_p $-modulo per ogni $ p $ primo.
\end{lemma}
\begin{proof}
	Supponiamo che $ A $ sia $ G_p $-coomologicamente banale per ogni $ p $ (l'altra implicazione è ovvia). Fissiamo un sottogruppo $ H < G $. Vogliamo mostrare che
	$  \HH^i(H, \, A) = 0 $ per ogni $ i > 0 $. Scelto un primo $ p $, prendiamo un Sylow $ H_p < H $ che possiamo supporre, senza perdita di generalità, incluso in $ G_p $: per ipotesi $ \HH^i(H_p, \, A) $ è nullo.
	Sapendo ora che l'omomorfismo di restrizione è iniettivo sulla componente $ p $-primaria (lemma \ref{injp}),
	$$ \begin{tikzcd}
	0 \rar & T_p\HH^i(H, \, A) \rar["\Res"] & \HH^i(H_p, A) = 0,
	\end{tikzcd}  $$
	scopriamo che questa è nulla. Ricordando che i gruppi di coomologia sono di torsione (\ref{ntors}), la tesi segue dall'arbitrarietà di $ p $.
\end{proof}

Convinti ora che studiare la coomologia dei $ p $-gruppi sia una strada promettente, 
cominciamo con un lemma tecnico.

\begin{lemma}\label{ban1}
	Sia $ G $ un $ p $-gruppo finito, ogni modulo $ A $ di torsione $ p $-primario ha sottomodulo degli invarianti non banale. Detto altrimenti: se $ A^G = 0 $, allora $ A = 0 $.
\end{lemma}
\begin{proof}
	Per ogni elemento $ a \in A $ consideriamo il sottomodulo finito $ M $ generato da $ a $, questo viene partizionato dall'azione di $ G $ in orbite, che possiamo separare in banali e non per scrivere l'equazione delle classi:
	\[ |\,M\,| = |\, M^G\,| +\sum_{x \in R} \frac{|\,G\,|}{|Stab(x)|},  \]
	dove $ R $ è un opportuno sistema di rappresentanti. Concludiamo che, dividendo tutti gli altri addendi, $ p $ deve dividere anche $ |\, M^G \, | $.
\end{proof}

Questo lemma ci tornerà particolarmente utile nella dimostrazione del risultato seguente: un criterio per stabilire la banalità dei moduli di $ p $-torsione.

\begin{proposition} \label{ban2}
	Sia $ A $ uno $ G_p $-modulo di $ p $-torsione.
	Se esiste un indice $ q $ per cui $ \Hhq(G, \, A) = 0 $, allora $ A $ è indotto.
\end{proposition}

\begin{proof}
	Dimostreremo che $ A $ è uno $ \F_p[G] $-modulo libero; il che è equivalente alla tesi perché $ \F_p[G] = \Z[G] \otimes_\Z \F_p $ è indotto. \\
	
	Cominciamo osservando che $ A^G $ è un $ \F_p $-modulo libero per ipotesi. Scelta una base di $ A^G $, sia $ F $ il $ \F_p[G] $-modulo libero generato sulla stessa base, per cui abbiamo un isomorfismo
	\[ j \colon A^G \to F^G. \]
	
	Mostreremo che esiste un sollevamento di $ j $ a un isomorfismo tra $ F $ ed $ A $. Consideriamo la successione esatta
	\[ 0 \to A^G \to A \to A/A^G \to 0 \]
	e prendiamone gli $ \Hom_\Z(\, \bullet, \, F) $
	\begin{equation}\label{homsucc1}
		0 \to \Hom_\Z(A/A^G, \, F) \to \Hom_\Z(A, \, F) \to \Hom_\Z(A^G, \, F).
	\end{equation}
	Poiché $ F $ è indotto, anche $ \Hom_\Z(A/A^G, \, F) $ è indotto! Nella successione esatta lunga associata troviamo dunque la suriezione
	\[ \begin{tikzcd}[column sep = small]
	\Hom_G(A, \, F) \rar & \Hom_G(A^G, \, F) \rar & 0,
	\end{tikzcd}\]
	da cui segue l'esistenza di un sollevamento $ J \colon A \to F $ di $ j \colon A^G \to F^G. $ \\
	
	Non ci resta che mostrare che $ J $ è biettivo. L'iniettività è immediata: $ \ker J $ è uno $ G $-modulo di $ p $-torsione senza invarianti, infatti
	\[ (\ker J)^G = \ker j = 0, \]
	dunque è banale per l'appena dimostrato Lemma \ref{ban1}. Rimaniamo pertanto con la successione esatta
	\[ \begin{tikzcd}[column sep = small]
	0 \rar& A \rar["J"]& F \rar& Q \rar& 0,
	\end{tikzcd} \]
	la cui successione esatta lunga associata comincia con
	\[ \begin{tikzcd}[column sep = small]
	0 \rar & A^G \rar["j"] & F^G \rar & Q^G \rar & \HH^1(G, \, A).
	\end{tikzcd} \]
	Se $ q = 1 $ ne deduciamo che, essendo $ j $ un isomorfismo, $ Q^G \to 0 $ è iniettiva, dunque $ Q^G = 0 $ e la tesi segue applicando nuovamente il Lemma \ref{ban1}. Se $ q \neq 1 $ abbiamo bisogno di ingegnarci altrimenti: dobbiamo traslare l'ipotesi. Il modulo con coomologia traslata $ A_{q-1} $ ricade nelle ipotesi del caso precedente ed è pertanto coomologicamente banale: ci forza così l'ipotesi 
	\[ \Hhu(G, \, A) = \Hhdq(G, \, A_{q-1}) = 0 \]
	di cui avevamo bisogno per concludere.
\end{proof}

Con un ulteriore piccolo sforzo è possibile rimuovere l'ipotesi di $ p $-torsione, ottenendo un magnifico criterio per la banalità di un modulo. Per quanto osservato a inizio capitolo (\ref{banSylow}), continuiamo a enunciare i teoremi per $ p $-gruppi, sottolineando però che per estenderli al caso generale è sufficiente richiedere le stesse ipotesi, ma su ogni primo.

\begin{theorem} \label{CriterioBanalita}
	Sia $ A $ uno $ G_p $-modulo. Se esiste un indice $ q $ per cui
	\[ \Hhq(G_p, \, A) = \Hhqu(G_p, \, A) = 0, \]
	allora $ A $ è coomologicamente banale.
\end{theorem}

\begin{proof}
	Vogliamo mostrare che $ A $ è proiettivo. Esiste un modulo libero $ F $ per cui possiamo scrivere
	\begin{equation*}
		0 \to R \to F \to A \to 0,
	\end{equation*}
	non ci resta che trovare una sezione.
	Passando alla successione esatta lunghissima, poiché $ F $ è coomologicamente banale, riusciamo a spostare l'ipotesi su $ R $:
	\[ \Hhqu(G_p, \, R) = \Hhqd(G_p, \, R) = 0. \]
	Possiamo pertanto supporre senza perdita di generalità che $ A $ sia libero come gruppo abeliano: il nucleo $ R $ lo è comunque e se fosse coomologicamente banale lascerebbe $ A $ da solo ad affogare in un mare di zeri. In questo caso $ \Hom_\Z(A, \, \bullet) $ è esatto e ci restituisce quindi una successione esatta corta
	\begin{equation}\label{homban3}
		0 \to \Hom_\Z(A, \, R) \to \Hom_\Z(A, \, F) \to \Hom_\Z(A, \, A) \to 0.
	\end{equation}
	Mostriamo che sotto queste ipotesi, il primo termine ha primo gruppo di coomologia banale.
	\begin{lemma}
		Detto $ M = \Hom_\Z(A, R) $, troviamo $ \HH^1(G, \, M) = 0 $.
	\end{lemma}
	\begin{proof}
		Consideriamo la successione esatta
		\begin{equation*}
		\begin{tikzcd}[column sep = small]
		0 \rar & R \rar["\cdot p"] & R \rar & R/pR \rar & 0,
		\end{tikzcd}
		\end{equation*}
		e la successione esatta lunghissima associata; il conucleo $ R/pR $ è uno $ G_p $-modulo di $ p $-torsione per cui $ \Hhqu(G_p, R/pR) = 0 $ e pertanto, rileggendo la proposizione precedente (\ref{ban2}), scopriamo che è indotto. Applicando $ \Hom_\Z(A, \, \bullet) $ alla successione esatta otteniamo
		\begin{equation*}
		\begin{tikzcd}[column sep = small]
		0 \rar & M \rar["\cdot p"] & M \rar & \Hom_\Z(A, \, R/pR) \rar & 0,
		\end{tikzcd}
		\end{equation*}
		nella cui successione esatta lunga, poiché $ \Hom_\Z(A, \, R/pR) $ è a sua volta indotto, troviamo
		\[ \begin{tikzcd}
		\HH^1(G_p, \, M) \rar["\cdot p"] & \HH^1(G_p, \, M) \rar & 0.
		\end{tikzcd}  \]
		Se la moltiplicazione per $ p $ è suriettiva, il gruppo abeliano $ \HH^1(G_p, \, M) $ non può avere $ p $-torsione, ne tanto meno componente $ p $-primaria. Sapendo però che dev'essere un gruppo di $ |G_p | $-torsione (per \ref{ntors}), siamo costretti a concludere che è nullo. 
	\end{proof}
	Prendendo la successione esatta lunga associata alla $ \eqref{homban3} $ scopriamo che
	\[\begin{tikzcd}
	\Hom_G(A, \, F) \rar & \Hom_G(A, \, A) \rar & 0,
	\end{tikzcd} \]
	ovvero che esiste una sezione dell'identità $ \id_A $.
\end{proof}

Segue immediatamente che:

\begin{corollary}
	Se $ A $ è coomologicamente banale, tutti gli $ \Hh(G, \, A) $ sono banali.
\end{corollary}

Concludiamo il capitolo con un risultato la cui utilità sarà chiara solo fra qualche capitolo.

\begin{corollary}\label{tensor magic}
	Sia $ A $ un modulo coomologicamente banale e $ D $ un modulo piatto. L'estensione per scalari $ A \otimes D $ è coomologicamente banale.
\end{corollary}
\begin{proof}
	La dimostrazione del teorema precedente mostra che se $ A $ è coomologicamente banale, allora esistono $ F $ libero e $ P $ proiettivo per cui la successione
	\[ 0 \to P \to F \to A \to 0 \]
	è esatta. Per piattezza anche la successione
	\[ 0 \to P \otimes D \to F\otimes D \to A\otimes D \to 0 \]
	è esatta. Poiché $ F \otimes D $ rimane indotto e $ P \otimes D $ rimane addendo diretto di un indotto, nella successione esatta lunghissima associata troviamo i gruppi di coomologia con coefficienti in $ A \otimes D $ immersi in un mare di zeri; concludiamo che $ A \otimes D $ dev'essere coomologicamente banale.
\end{proof}

\section{Dimensione Coomologica}
La dimensione coomologica di un gruppo è un indice di complessità: è il numero di gruppi di coomologia non nulli che ci aspettiamo di trovare, più o meno indipendentemente dal modulo dal quale peschiamo i coefficienti. Questo tipo di studio risulta interessante per lo più per gruppi infiniti; è giunto il momento di abbandonare l'ipotesi di finitezza.
\begin{definition}[Dimensione coomologica]
	Sia $ G $ un gruppo profinito e $ p $ un numero primo. La \emph{$ p $-dimensione coomologica} di $ G $ è il più piccolo intero $ n $ per cui: per ogni $ G $-modulo $ A $ di torsione, la componente $ p $-primaria di $ \HH^q(G, \, A) $ è nulla non appena $ q > n $, volendo
	\[ \cd_p(G) := \inf \{ n \in \N \; \mid T_p\HH^q(G, \, A)=0\quad \forall \, q > n, \quad \forall A \text{ di torsione} \}. \]
	La \emph{dimensione coomologica} è definita di conseguenza come $ \cd(G) := \sup_p \cd_p(G). $
\end{definition}

È necessaria una certa attenzione nel maneggiare la definizione di dimensione coomologica: abbiamo richiesto che $ A $ fosse un modulo \emph{di torsione}, rimuovendo questa ipotesi si finisce a parlare di dimensione coomologica stretta, che è tutta un'altra storia (anzi, ancora peggio, una storia leggermente diversa). \\

%% vorrei capire cosa mi serve a cosa mi serve prima di scrivere un sacco di roba inutile...
Siamo in grado di formulare un criterio per il calcolo della dimensione coomologica, riducendo i moduli su cui dobbiamo calcolare la coomologia a solamente quelli semplici.

\begin{proposition}
	Sia $ G_p $ un $ p $-gruppo profinito. La dimensione $ \cd_p(G_p) \leq n $ se e solo se $ \HH^{n+1}(G_p, \, \Z/p\Z) = 0 $. 
\end{proposition}

\begin{proof}
	Il risultato è vagamente intuitivo e la dimostrazione sarà modellata attorno a questa sensazione: la componente di $ p $-torsione si può calcolare senza perdita di generalità sui moduli $ p $-primari, che a loro volta potremo spezzare in sottomoduli semplici, riducendo il calcolo all'unico modulo semplice di $ p $-torsione che esiste, $ \Z/p\Z $. Procediamo con ordine. \\
	
	Mostriamo che l'unico $ G $-modulo semplice di $ p $-torsione è $ \Z/p\Z $ (con l'azione banale!). Un modulo di questo tipo dev'essere, ovviamente, finito; nell'intersezione degli stablizzatori dei vari elementi, che ricordiamo essere aperti, troviamo un sottogruppo $ U < G $ aperto e normale per cui $ A $ è un $ G/U $-modulo semplice, riportandoci a dimostrare il risultato per un gruppo finito, per cui la tesi è chiara (grazie al lemma \ref{ban1}). \\
	
	Se $ \HH^{n+1}(G_p, \, \Z/p\Z) $ è nullo, allora $ \HH^{n+1}(G_p, \, A) $ è nullo per ogni modulo $ A $ finito di $ p $-torsione: In questo caso possiamo infatti scrivere una serie di composizione
	\[ 0 = A^m \hookrightarrow A^{m-1} \hookrightarrow \dots \hookrightarrow A^1 \hookrightarrow A^0 = A, \]
	i cui quozienti sono tutti semplici, ovvero, per quanto appena mostrato, si trovano in successioni esatte corte della forma
	\[ 0 \to A^{i+1} \to A^i \to \Z/p\Z \to 0. \]
	La collezione di successioni esatte lunghe associate restituisce una catena di mappe suriettive in grado $ n+1 $ per ipotesi,
	\[ 0 = \HH^{n+1}(G, \, A^0) \twoheadrightarrow \HH^{n+1}(G, \, A^1) \twoheadrightarrow \dots \twoheadrightarrow \HH^{n+1}(G, \, A), \]
	da cui segue la banalità di tutti i gruppi coinvolti. Rimuoviamo facilmente l'ipotesi di finitezza, sfruttando il solito risultato di passaggio al limite \ref{limite}, una volta scritto $ A $ come unione (o, per farla difficile, limite induttivo) dei suoi sottomoduli finiti. \\
	
	Per concludere, dimostriamo per décalage che i gruppi di coomologia in grado maggiore di $ n+1 $ sono anch'essi nulli, per ogni modulo $ A $ di $ p $-torsione. È sufficiente scrivere il modulo di coomologia traslata $ A_{r} $ e osservare che, nonostante $ G $ non sia finito, continua ad avere coomologia traslata per $ r \geq 0 $ perché l'indotto $ \Ind_1^G(A) $ è anch'esso di $ p $-torsione, essendo composto da mappe che, per essere continue, devono essere localmente costanti e pertanto con immagine finita. Concludiamo dunque che	
	\begin{equation*}
	 \Hhnru(G, \, A) = \Hhnu(G, \, A_r) = 0 \qquad \forall\, r > 0. \qedhere
	\end{equation*}
\end{proof}

Calcoliamo la dimensione coomologica del gruppo additivo degli interi $ p $-adici $ \Z_p = \varprojlim_n \Z/p^n\Z $. Per il solito teorema di passaggio al limite $ (\ref{limite}) $, si ha che 
\[ \HH^2(\Z_p , \, \Z/p\Z) = {\textstyle\varinjlim_n} \HH^2(\Z/p^n\Z, \, \Z/p\Z) = {\textstyle\varinjlim_n} \Hhz(\Z/p^n\Z, \, \Z/p\Z). \]
I gruppi di Tate di grado zero li conosciamo esplicitamente: per definizione sono $ \Z/p\Z $ quozientato per l'immagine della norma che, trovando l'azione banale, coincide con la moltiplicazione per l'ordine del gruppo $ p^n $ e quindi è la mappa nulla. Le mappe di collegamento sono indotte dalla proiezione naturale e coincidono quindi con l'inflazione, che in grado zero è per definizione la norma. Poiché tutte le mappe sono nulle
\[ \HH^2(\Z_p , \, \Z/p\Z) = \varinjlim \Z/p\Z = 0. \]
Per la proposizione appena dimostrata $ \cd_p(\Z_p) \leq 1 $; otteniamo l'altra uguaglianza calcolando esplicitamente $$  \HH^1(\Z_p , \, \Z/p\Z) = \Hom(\Z_p , \, \Z/p\Z) = \Z/p\Z \neq 0.  $$
Dunque $ \cd_p(\Z_p) = 1 $.
Riprendiamo lo studio della dimensione coomologica, mostrando in che modo la dimensione di un gruppo è legata a quella dei suoi sottogruppi.

\begin{proposition}\label{cd2}
	Sia $ G $ un gruppo profinito e $ H $ un suo sottogruppo chiuso. Per ogni primo $ p $ si ha
	\[ \cd_p(H) \leq \cd_p(G). \]
	Se per di più l'indice $ [G\,\colon H] $ è coprimo con $ p $, si ha uguaglianza.
\end{proposition}
\begin{proof}
	Se $ A $ è un $ H $-modulo $ p $-primario, allora anche $ \Ind_H^G(A) $ è $ p $-primario. Per il Lemma di Shapiro si ha che per ogni indice $ i $
	\[ \HH^i(H, \, A) = \HH^i(G, \, \Ind_H^G(A)),  \]
	da cui la tesi. Se l'indice $ [G\,\colon H] $ è coprimo con $ p $, per ogni modulo la restrizione fornisce una mappa nella direzione opposta
	\[\begin{tikzcd}
	0 \rar & T_p\HH^i(G, \, A) \rar["\Res"] & \HH^i(H, \, A),
	\end{tikzcd}  \]
	iniettiva per il lemma \ref{injp}.
\end{proof}

Quest'ultima proposizione ci permette di ricondurre il calcolo della dimensione di un gruppo a quella dei suoi Sylow. L'esempio per eccellenza è $ \hat{\Z} = \varprojlim_n \Z/n\Z $, dei cui Sylow $ \Z_p$ conosciamo già la dimensione: per ogni primo $ p $ si ha
\[ \cd_p(\hat{\Z}) = \cd_p(\Z_p) = 1, \]
da cui $ \cd(\hat{\Z})=1 $.\\

% Mi serve il lemma che lega sottogruppi e quozienti
Non possiamo sperare in un risultato analogo al precedente per i quozienti: per esempio $ \Z_p $ ha dimensione coomologica finita, mentre i suoi quozienti finiti essendo ciclici hanno gruppi non nulli di indice arbitrariamente grande. Troviamo comunque una disuguaglianza della direzione opposta.
\begin{proposition}\label{quozienti}
	Sia $ H $ un sottogruppo chiuso e normale di $ G $. Allora, per ogni numero primo $ p $, si ha che
	\[ \cd_p(G) \leq \cd_p(H) + \cd_p(G/H). \]
\end{proposition}
\begin{proof}
	Segue dall'esistenza successione spettrale di \HS: per ogni modulo di $ A $ di torsione, la successione ci fornisce una filtrazione di $ \HH^n(G, \, A) $ i cui quozienti sono, partendo dalla descrizione della seconda pagina, sottoquozienti di $ \HH^i(G/H, \, \HH^j(H, \, A)) $ per $ i+j = n $. Osserviamo che $ \HH^j(H, \, A) $ sarà nullo non appena $ j > \cd_p(H) $ e che, essendo $ \HH^j(H, \, A) $ di torsione a sua volta (\ref{ntors}), anche $ \HH^i(G/H, \, \HH^j(H, \, A)) $ sarà nullo non appena $ i > \cd_p(G/H) $; da cui la disuguaglianza voluta.
\end{proof}