\chapter{Primi risultati}
In questo capitolo giocheremo con il nostro nuovo strumento, esplorandone le potenzialità e gettando delle solide basi sui cui poggeranno i calcoli successivi. Il primo obbiettivo sarà studiare la coomologia dei gruppi ciclici, che si rivelerà piuttosto semplice e di evidente importanza in tutto il seguito. Vorremmo poi capire come calcolare operativamente la coomologia di un gruppo, per esempio riconducendola a quella dei suoi Sylow. \\

Sistemati i giocattoli, sarà il momento di concentrarci sul vero obbiettivo del nostro lavoro: i gruppi di Galois. Per affrontare la Teoria di Galois avremo bisogno di rimuovere l'ipotesi di finitezza su cui abbiamo fondato la nostra teoria, in modo da comprendere quantomeno i gruppi profiniti.

\section{Gruppi Ciclici}
Sia nel seguito $ G $ un gruppo ciclico di ordine $ n $ e $ \sigma $ un generatore fissato. La coomologia di $ G $ è particolarmente semplice da descrivere: qualunque sia il modulo $ A $, i gruppi di Tate sono tutti uguali a $ \Hh^0(G, \, A) $ oppure a $ \Hh^1(G, \, A) $, per entrambi i quali conosciamo una descrizione esplicita! \todo[questa definizione dei gruppi di Tate mi sposta le righe! Sistemare!]

\begin{theorem}[dei Gruppi Ciclici]\label{ciclici}
	Per un gruppo ciclico finito $ G $ e un suo modulo $ A $, i gruppi di Tate sono 2-periodici nel grado:
	\[ \Hh^{i+2}(G, \, A) = \Hh^{i}(G, \, A). \]
\end{theorem}

\begin{proof}
	Siano $ N = 1 + \sigma + \sigma^2 + \dots + \sigma^{n-1} $ e $ D = 1 - \sigma $ due elementi di $ \Z[G] $. Il teorema si basa interamente sullo scrivere la risoluzione proiettiva
	\[ \begin{tikzcd}[column sep = small]
	\dots \rar["N"]
	& \Z[G] \rar["D"]
	& \Z[G] \rar["N"]
	& \Z[G] \rar["D"]
	& \Z[G] \rar
	& \Z \rar & 0,
	\end{tikzcd} \]
	da cui segue la 2-periodicità sia in coomologia che in omologia. 
	
	\todo[finire: come ci appiccichiamo i gradi bassi?]
\end{proof}

Per i moduli i cui gruppi di Tate sono finitamente generati abbiamo scoperto due importanti invarianti: la cardinalità dei gruppi in grado pari e dei gruppi in grado dispari, che chiamiamo $ h_0(A) $ e $ h_1(A) $.

\begin{definition}(Quoziente di Herbrand)
	Se $ A $ ha gruppi di Tate finiti chiamiamo \emph{quoziente di Herbrand} il numero naturale
	\[ {h}(A) := \frac{h_0(A)}{h_1(A)} = \frac{|\Hh^0(G, \, A)|}{|\Hh^1(G, \, A)|}. \]
\end{definition}

\begin{proposition}[Herbrand]
	Alcune proprietà del quoziente:
	\begin{enumerate}
		\item Data una successione esatta corta
		\[ 0 \to A \to B \to C \to 0, \]
		se il quoziente $ h $ è definito per due moduli su tre, è definito anche per il terzo. In questo caso vale $ h(A)h(C) = h(B) $.
		\item Se $ A $ è finito, $ h(A) = 1 $.
		\item Se $ f \colon A \to B $ ha nucleo e conucleo finiti e uno tra $ h(A) $ e $h(B) $ è definito, allora lo è anche l'altro e sono uguali.
	\end{enumerate}
\end{proposition}

\begin{proof}
	\begin{enumerate}
		\item La successione esatta lunghissima associata può essere ririassunta in una successione \leftquote esagonalmente esatta"
		\[\begin{tikzcd}[column sep={1cm,between origins}, row sep={1.732050808cm,between origins}]
			& \Hh^0(G, \, A) \arrow[rr] && \Hh^0(G, \, B) \arrow[rd] &  \\
			\Hh^1(G, \, C) \arrow[ru]&  &&  & \Hh^0(G, \, C) \arrow[ld] \\				& \Hh^1(G, \, B) \arrow[lu] && \Hh^1(G, \, A) \arrow[ll], & 
		\end{tikzcd}\]
		che possiamo slacciare in un punto a piacere
		\[ 0 \to Q \to \Hh^0(G, \, A) \to \Hh^0(G, \, B) \to \Hh^0(G, \, C) \to \Hh^1(G, \, A) \to \Hh^1(G, \, B) \to \Hh^1(G, \, C) \to Q \to 0, \]
		aggiungendo un opportuno modulo $ Q = \ker \left[\Hh^0(G, \, A) \to \Hh^0(G, \, B)\right] = \left[ \Hh^1(G, \, B) \to \Hh^1(G, \, C) \right] $.
		Dobbiamo pertanto avere, per esattezza, che
		\[  h_0(B) h_1(A) h_1(C) \cdot |\,Q\,| = h_0(A) h_0(C) h_1(B) \cdot |\,Q\,|, \]			da cui la tesi.
		
		\item Per come abbiamo definito i gruppi di Tate abbiamo la successione esatta
		\begin{equation*}			\begin{tikzcd}[column sep = small]
			0 \rar& \Hh^{-1}(G, \, A) \rar& A_G \rar["N"]& A^G \rar&\Hh^0(G, \, A) \rar& 0,
			\end{tikzcd}
		\end{equation*}
		da cui deduciamo che $ h_1(A) \cdot |\, A^G\, | =  h_0(A) \cdot |\, A_G\, | $; siccome il gruppo è ciclico siamo inoltre in grado di ottenere la desiderata relazione fra invarianti e co-invarianti:
		\begin{equation*}
		\begin{tikzcd}[column sep = small]
			0 \rar& A_G \rar& A \rar["D"]& A \rar& A^G \rar& 0.
			\end{tikzcd}
		\end{equation*}
		\item É sufficiente scrivere le ipotesi in forma di successioni esatte
		\[ 0 \to \ker f \to A \to Q \to 0 \qquad\text{ e } \qquad 0 \to Q \to B \to \coker f \to 0, \]
		applicare il primo punto per dedurre la finitezza dei gruppi di Tate, il secondo per ottenere l'uguaglianza desiderata.
	\end{enumerate}
\end{proof}

L'ultimo punto della proposizione sembra, più che una proprietà elementare del quoziente di Herbrand, un lemma tecnico e particolarmente specifico: è così, ci servirà in una dimostrazione particolarmente impegnativa più avanti.

\section{p - gruppi}

\section{Gruppi profiniti}

\section{dimensione coomologica}