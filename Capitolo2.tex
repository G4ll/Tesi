\chapter{Primi risultati}
In questo capitolo giocheremo con il nostro nuovo strumento, esplorandone le potenzialità e gettando delle solide basi sui cui poggeranno i calcoli successivi. Il primo obbiettivo sarà studiare la coomologia dei gruppi ciclici, che si rivelerà piuttosto semplice e di evidente importanza in tutto il seguito. Vorremmo poi capire come calcolare operativamente la coomologia di un gruppo, per esempio riconducendola a quella dei suoi Sylow. \\

Sistemati i giocattoli, sarà il momento di concentrarci sul vero obbiettivo del nostro lavoro: i gruppi di Galois. Per affrontare la Teoria di Galois avremo bisogno di rimuovere l'ipotesi di finitezza su cui abbiamo fondato la nostra teoria, in modo da comprendere quantomeno i gruppi profiniti.

\section{Gruppi Ciclici}
Sia nel seguito $ G $ un gruppo ciclico di ordine $ n $ e $ \sigma $ un generatore fissato. La coomologia di $ G $ è particolarmente semplice da descrivere: qualunque sia il modulo $ A $, i gruppi di Tate sono tutti uguali a $ \Hh^0(G, \, A) $ oppure a $ \Hh^1(G, \, A) $, per entrambi i quali conosciamo una descrizione esplicita! \todo[questa definizione dei gruppi di Tate mi sposta le righe! Sistemare!]

\begin{theorem}[dei Gruppi Ciclici]\label{ciclici}
	Per un gruppo ciclico finito $ G $ e un suo modulo $ A $, i gruppi di Tate sono 2-periodici nel grado:
	\[ \Hh^{i+2}(G, \, A) = \Hh^{i}(G, \, A). \]
\end{theorem}

\begin{proof}
	Siano $ N = 1 + \sigma + \sigma^2 + \dots + \sigma^{n-1} $ e $ D = 1 - \sigma $ due elementi di $ \Z[G] $. Il teorema si basa interamente sullo scrivere la risoluzione proiettiva
	\[ \begin{tikzcd}[column sep = small]
	\dots \rar["N"]
	& \Z[G] \rar["D"]
	& \Z[G] \rar["N"]
	& \Z[G] \rar["D"]
	& \Z[G] \rar
	& \Z \rar & 0,
	\end{tikzcd} \]
	da cui segue la 2-periodicità sia in coomologia che in omologia. 
	
	\todo[finire: come ci appiccichiamo i gradi bassi?]
\end{proof}

Per i moduli i cui gruppi di Tate sono finitamente generati abbiamo scoperto due importanti invarianti: la cardinalità dei gruppi in grado pari e dei gruppi in grado dispari, che chiamiamo $ h_0(A) $ e $ h_1(A) $.

\begin{definition}(Quoziente di Herbrand)
	Se $ A $ ha gruppi di Tate finiti chiamiamo \emph{quoziente di Herbrand} il numero naturale
	\[ {h}(A) := \frac{h_0(A)}{h_1(A)} = \frac{|\Hh^0(G, \, A)|}{|\Hh^1(G, \, A)|}. \]
\end{definition}

\begin{proposition}[Herbrand]
	Alcune proprietà del quoziente:
	\begin{enumerate}
		\item Data una successione esatta corta
		\[ 0 \to A \to B \to C \to 0, \]
		se il quoziente $ h $ è definito per due moduli su tre, è definito anche per il terzo. In questo caso vale $ h(A)h(C) = h(B) $.
		\item Se $ A $ è finito, $ h(A) = 1 $.
		\item Se $ f \colon A \to B $ ha nucleo e conucleo finiti e uno tra $ h(A) $ e $h(B) $ è definito, allora lo è anche l'altro e sono uguali.
	\end{enumerate}
\end{proposition}

\begin{proof}
	\begin{enumerate}
		\item La successione esatta lunghissima associata può essere ririassunta in una successione \leftquote esagonalmente esatta"
		\[\begin{tikzcd}[column sep={1cm,between origins}, row sep={1.732050808cm,between origins}]
			& \Hh^0(G, \, A) \arrow[rr] && \Hh^0(G, \, B) \arrow[rd] &  \\
			\Hh^1(G, \, C) \arrow[ru]&  &&  & \Hh^0(G, \, C) \arrow[ld] \\				& \Hh^1(G, \, B) \arrow[lu] && \Hh^1(G, \, A) \arrow[ll], & 
		\end{tikzcd}\]
		che possiamo slacciare in un punto a piacere \todo[su]
		\[ 0 \to Q \to \Hh^0(G, \, A) \to \Hh^0(G, \, B) \to \Hh^0(G, \, C) \to \Hh^1(G, \, A) \to \Hh^1(G, \, B) \to \Hh^1(G, \, C) \to Q \to 0, \]
		aggiungendo un opportuno modulo $ Q = \ker \left[\Hh^0(G, \, A) \to \Hh^0(G, \, B)\right] = \left[ \Hh^1(G, \, B) \to \Hh^1(G, \, C) \right] $.
		Dobbiamo pertanto avere, per esattezza, che
		\[  h_0(B) h_1(A) h_1(C) \cdot |\,Q\,| = h_0(A) h_0(C) h_1(B) \cdot |\,Q\,|, \]			da cui la tesi.
		
		\item Per come abbiamo definito i gruppi di Tate abbiamo la successione esatta
		\begin{equation*}			\begin{tikzcd}[column sep = small]
			0 \rar& \Hh^{-1}(G, \, A) \rar& A_G \rar["N"]& A^G \rar&\Hh^0(G, \, A) \rar& 0,
			\end{tikzcd}
		\end{equation*}
		da cui deduciamo che $ h_1(A) \cdot |\, A^G\, | =  h_0(A) \cdot |\, A_G\, | $; siccome il gruppo è ciclico siamo inoltre in grado di ottenere la desiderata relazione fra invarianti e co-invarianti:
		\begin{equation*}
		\begin{tikzcd}[column sep = small]
			0 \rar& A_G \rar& A \rar["D"]& A \rar& A^G \rar& 0.
			\end{tikzcd}
		\end{equation*}
		\item É sufficiente scrivere le ipotesi in forma di successioni esatte
		\[ 0 \to \ker f \to A \to Q \to 0 \qquad\text{ e } \qquad 0 \to Q \to B \to \coker f \to 0, \]
		applicare il primo punto per dedurre la finitezza dei gruppi di Tate, il secondo per ottenere l'uguaglianza desiderata.
	\end{enumerate}
\end{proof}

L'ultimo punto della proposizione sembra, più che una proprietà elementare del quoziente di Herbrand, un lemma tecnico e particolarmente specifico: è così, ci servirà in una dimostrazione particolarmente impegnativa più avanti.

\section{Banalità}
Lo studio della coomologia di un generico gruppo si fa troppo complessa per poter sperare di enunciare teoremi di portata analoga a quelli validi per i gruppi ciclici. Si riesce però, con non poca fatica, a dire qualcosa sui gruppi che hanno tutti i gruppi di coomologia banali.

\begin{definition}
	Diciamo che uno $ G $ modulo $ A $ è \emph{coomologicamente banale} se
	$$  \HH^i(H, \, A) = 0 \qquad \forall \, i > 0, \quad  \forall \, H < G.  $$
\end{definition}

\begin{remark}
	Abbiamo già mostrato che i moduli indotti sono coomologicamente banali (Lemma \ref{indotti}).
\end{remark}

Il lettore più accorto potrebbe trovarsi perplesso da questa scelta: perché, dopo i generosi risultati dei capitoli precedenti, abbiamo abbanondato i gruppi di Tate? Scopriremo nel seguito che le due definizioni sono equivalenti. \\

Una delle ragioni per cui siamo interessati ai moduli coomologicamente banali è che possiamo spezzarne lo studio in parti più semplici. Per ogni primo $ p $ possiamo scegliere un $ p $-Sylow $ G_p $ di $ G $, il gruppo $ \Hh^i(G_p, \, A) $ non dipende dalla scelta del Sylow, poiché ogni mappa di coniugio $ \sigma_t $ che li permuta induce un isomorfismo in coomologia. Fissiamo dunque un Sylow $ G_p $ per ogni primo. Non è facile, in generale, risalire dalla coomologia dei Sylow a quella del gruppo originario, il seguente risultato ci convincerà però a continuare lo studio sui Sylow.

\begin{lemma}
	Uno $ G $-modulo $ A $ è coomologicamente banale se e solo se è coomoligcamente come $ G_p $-modulo per ogni $ p $ primo.
\end{lemma}
\begin{proof}
	Supponiamo che $ A $ sia $ G_p $-coomologicamente banale per ogni $ p $ (l'altra implicazine è ovvia). Fissiamo un sottogruppo $ H < G $. Vogliamo mostrare che
	$  \HH^i(H, \, A) = 0 $ per ogni $ i > 0 $. Scelto un primo $ p $, prendiamo un Sylow $ H_p < H $ che possiamo supporre, senza perdita di generalità, incluso in $ G_p $: per ipotesi $ \HH^i(H_p, \, A) $ è nullo.
	Sapendo ora che l'omomofismo di restrizione $ \Res \colon \HH^i(H, \, A) \to \HH^i(H_p, A) $ è iniettivo sulla componente $ p $-primaria (lemma \ref{injp}), scopriamo che questa è nulla. La tesi segue dall'arbitrarietà di $ p $.
	
\end{proof}

Covinti ora a studiare l'omologia dei $ p $-gruppi, 
cominciamo con un lemmino tecnico.

\begin{lemma}\label{ban1}
	Sia $ G $ un $ p $-gruppo finito, ogni modulo $ A $ un di torsione $ p $-primario ha sottomodulo degli invarianti non banale. Detto altrimenti: se $ A^G = 0 $, allora $ A = 0 $.
\end{lemma}
\begin{proof}
	Per ogni elemento $ a \in A $ consideriamo il sottomodulo finito $ M $ generato da $ a $, questo è spaccato dall'azione di $ G $ in orbite, fornendoci un'equazione di classe:
	\[ |\,M\,| = |\, M^G\,| +\sum \frac{|\,G\,|}{|Stab(x)|},  \]
	da cui concludiamo che, dividendo tutti gli altri addendi, $ p $ deve dividere anche $ |\, M^G \, | $.
\end{proof}

Questo lemma ci tornerà particolarmente utile nella dimostrazione del risultato seguente: un criterio per stabilire la banalità dei moduli di $ p $-torsione.

\begin{proposition}
	Sia $ A $ uno $ G_p $-modulo di $ p $-torsione. Se esiste un indice $ q $ per cui $ \Hh^q(G, \, A) = 0 $, allora $ A $ è indotto.
\end{proposition}

\begin{proof}
	Dimostreremo che $ A $ è uno $ \F_p[G] $-modulo libero; è equivalente alla tesi perché $ \F_p[G] = \Z[G] \otimes \F_p $ è indotto. \\
	
	Cominciamo osservando che $ A^G $ è un $ \F_p $-modulo libero per ipotesi. Scelta una base base di $ A^G $, costruiamo $ F $ come l'$ \F_p[G] $-modulo libero generato sulla stessa base, per cui abbiamo un isomorfismo
	\[ j \colon A^G \to F^G. \]
	
	Mostriamo che eiste un sollevamento di $ j $ a un isomorfismo tra $ F $ ed $ A $. Consideriamo la successione esatta
	\[ 0 \to A^G \to A \to A/A^G \to 0 \]
	e prendiamone gli $ \Hom_\Z(\, \bullet, \, F) $
	\begin{equation}\label{homsucc1}
		0 \to \Hom_\Z(A/A^G, \, F) \to \Hom_\Z(A, \, F) \to \Hom_\Z(A^G, \, F).
	\end{equation}
	Per un lemma mistico che ci siamo dimenticati di dimostrare, $ \Hom_Z(A/A^G) $ è indotto. \todo[dimostarre il lemma mistico nell'opportuna sezione] Nella la successione esatta lunga assocciata alla $ \ref{homsucc1} $, poiché $ \HH^1(G, \, \Hom_\Z(A/A^G, \, F)) = 0 $, troviamo dunque la suriezione
	\[ \Hom_\Z(A, \, F)^G \to \Hom_\Z(A^G, \, F)^G \to 0, \]
	che possiamo riscrivere più chiaramente come
	\[ \Hom_G(A, \, F) \to \Hom_G(A^G, \, F^G) \to 0; \]
	troviamo pertanto un sollevamento $ J \colon A \to F $.\\
	
	Non ci resta che mostrare che anche $ J $ è bigettivo. Supponiamo per il momento che $ q = 1 $, ovvero che $ \HH^1(G, \, A) = 0 $. L'iniettività è immediata: $ \ker J $ è uno $ G $ di modulo di $ p $-torsione senza invarianti, infatti
	\[ (\ker J)^G = \ker j = 0, \]
	dunque è banale per il Lemma \ref{ban1} appena mostrato. Rimaniamo con la successione esatta
	\[ \begin{tikzcd}[column sep = small]
	0 \rar& A \rar["J"]& F \rar& Q \rar& 0,
	\end{tikzcd} \]
	la cui successione esatta lunga associata comincia con
	\[ \begin{tikzcd}[column sep = small]
	0 \rar & A^G \rar["j"] & F^G \rar & Q^G \rar & 0,
	\end{tikzcd} \]
	da cui deduciamo che, essendo $ j $ un isomorfismo, $ Q^G \to 0 $ è iniettiva e la tesi segue dal Lemma \ref{ban1}.\\
	
	Se $ q \neq 1 $ abbiamo bisogno di ingegnarci altrimenti. Quando $ q > 1 $, definiamo $ A_1 $ come il conucleo dell'iniezione di $ A $ nel suo indotto:
	\[ 0 \to A \to \Ind^G(A) \to A_1 \to 0. \]
	Nella successione esatta lunghissima associata troviamo
	\[ \Hh^{i-1}(G, \, \Ind^G(A)) \to \Hh^{i-1}(G, \, A_1) \to \Hh^{i}(G, \, A) \to \Hh^{i}(G, \, \Ind^G(A)),  \]
	da cui deduciamo che $ \Hh^{i-1}(G, \, A_1) = \Hh^{i}(G, \, A) $ e, ragionando per induzione su $ r $, che
	\[ \Hh^{i-r}(G, \, A_r) = \Hh^{i}(G, \, A). \] 
	Abbiamo quindi la tesi per $ A_{q-1} $, che è pertanto coomologicamente banale e ci forza così l'ipotesi 
	\[ \Hh^1(G, \, A) = \Hh^{2-q}(G, \, A_{q-1}) = 0 \]
	di cui avevamo bisogno per concludere.
	Analogamente, quando $ q < 1 $, consideriamo la successione esatta corta della proiezione:
	\[ 0 \to A_{-1} \to \Ind^G(A) \to A \to 0. \]
\end{proof}
\todo[in realtà non mi dispiacerebbe mettere quest'ultima parte in un posto migliore]

Con ancora un piccolo sforzo riusciamo a rimuovere l'ipotesi di $ p $-torsione.

\begin{theorem}
	Sia $ A $ uno $ G_p $-modulo. Se esiste un indice $ q $ per cui
	\[ \Hh^q(G_p, \, A) = \Hh^{q+1}(G_p, \, A) = 0, \]
	allora $ A $ è coomologicamente banale.
\end{theorem}

\begin{proof}
	La dimostrazione è un grande trucco di prestidigitazione.
\end{proof}

\begin{corollary}
	Se $ A $ è coomologicamente banale, allora esistono $ F $ libero e $ P $ proiettivo per cui
	\[ 0 \to P \to F \to A \to 0 \]
	è esatta.
\end{corollary}

\section{Gruppi profiniti}
Non tutti i gruppi in Teoria di Galois sono finiti. Per poterla affrontare avremo dunque bisogno di estendere le capacità del nostro linguaggio dai gruppi finiti a quelli profiniti. Una prima difficoltà è data dalla comparsa in scena della topologia: la categoria dei $ G $-moduli non è adatta ai nostri scopi, perché non tiene conto della geometria del gruppo. Chiediamo quindi il minimo possibile, ovvero che i $ G $-moduli siano muniti della topologia discreta. In queste nuove ipotesi si riesce a definire la coomologia per gruppi profiniti; per nostra fortuna, questa si riconduce subito a quella per gruppi finiti:

\begin{proposition}
	Sia $ G $ un gruppo profinito che agisce su un modulo discreto $ A $, allora
	\[ \HH^n \left(G, \, A \right)  = \varinjlim_{U < G} \HH^n(G/U, \, A^U). \]
\end{proposition}

\section{dimensione coomologica}