\chapter{Dualità}

\section{Tazza Prodotto}
Costruiremo ora un'applicazione bilineare tra i gruppi di coomologia, donando al nostro strumento un prodotto e, dunque, una struttura di algebra. La costruzione del \leftquote tazza-prodotto", così come la presentazione delle relative proprietà, è fondamentalmente una boriosa, lunghissima, verifica: dovremo riesumare la presentazione in cocatene. Come ogni buon libro di algebra, presenteramo dettagliatamente la costruzione del prodotto e la lista delle proprietà che useremo in seguito, lasciando decidere al lettore se fidarsi ciecamente di quanto presentato oppure dedicare un paio di pomeriggi alla verifica diretta. \\

Poniamoci nella dovuta generalità. Siano $ A, \, B $ due $ G $-moduli e $ K_{\mathtt{om}}(A) $ e $ K_{\mathtt{om}}(B) $ i corrisponendti complessi di cocatene omogenee; che ricordiamo si ottengono omogeneizzando le usuali cocatene $ K(A) $ e $ K(B) $: presa $ f \colon G^i \to A $, definiamo $ F\colon G^{i+1} \to A  $ in modo che
\[ F(g_0, \, g_1, \, \dots, \, g_i) = g_0 f(g_0^{-1}g_1, \, g_1^{-1}g_2, \, \dots, \, g_{i-1}^{-1}g_i). \]
Questa presentazione è molto più comoda perchè l'applicazione bilineare
\begin{align*}
	\cup \colon K^p_{\mathtt{om}}(A) \times K^q_{\mathtt{om}}(B) &\to K_{\mathtt{om}}^{p+q}(A \otimes B)\\
	(a, \, b) & \mapsto a(g_0,\, \dots, \, g_p) \otimes b(g_p,\, \dots, \, g_{p+q})
\end{align*}
si comporta bene con i differenziali
\[ d(a \cup b) = (da) \cup b + (-1)^p a \cup (db), \]
permettendoci di passare in coomologia.

\begin{definition}[Tazza prodotto]
	Chiamiamo tazza prodotto l'applicazione $ \Z $-bilineare indotta da dal prodotto appena definito
	\[ \cup \colon \HH^p(G, \, A) \times \HH^q(G, \, B) \to \HH^{p+q}(G, \, A \otimes B)  \]
\end{definition}

Il prodotto è ovviamente compaitibile con le successioni esatte lugnhe:

\begin{proposition}
	Se entrambe le successioni
	\[ 0 \to A \to A' \to A'' \to 0, \qquad 0 \to A \otimes B \to A'\otimes B \to A''\otimes B \to 0 \]
	sono esatte, allora il tazza prodotto, fissato $ \beta \in \HH^q(G, \, B) $, in induce un morfismo di complessi
	\[\begin{tikzcd}[column sep = small]
	\dots \rar
	& \Hh^{p}(G, \, A') \dar["\cup\beta"] \rar 
	& \Hh^{p}(G, \, A'') \rar["\delta"] \dar["\cup\beta"]
	& \Hh^{p+1}(G, \, A) \rar \dar["\cup\beta"]
	&\dots  \\
	\dots \rar
	& \Hh^{p+q}(G, \, A'\otimes B) \rar
	& \Hh^{p+q}(G, \, A''\otimes B) \rar["\delta"]
	& \Hh^{p+q+1}(G, \, A\otimes B) \rar &\dots \, , \end{tikzcd}\]
	equivalentemente $ (\delta\alpha'') \cup \beta = \delta (\alpha'' \cup \beta) $.
\end{proposition}

Presa una qualunque applicazione $ \Z[G] $-bilineare $ \varphi\colon A \times B \to C $, componendo all'omomorfismo $ \varphi\colon A \otimes B \to C  $ con il tazza prodotto otteniamo qualunque sorta di prodotto in coomologia. Dalla proposizione di compatibilità precedente possiamo dedurne una versione appena più generale, che ci tornerà utile in futuro.

\begin{lemma}
	Date due successioni esatte
	\[ 0 \to A' \to A \to A'' \to 0, \qquad 0 \to B' \to B \to B'' \to 0 \]
	e un'applicazione bilineare $  \varphi\colon A \times B \to C $ nulla su $ A'\times B' $. Troviamo due applicazioni bilineari
	\[ \varphi'\colon A' \times B'' \to C, \qquad \varphi''\colon A'' \times B' \to C, \]
	che inducono prodotti in omologia
	\begin{align*}
		\HH^{p+1}(G, \, A'') \times \HH^{q}(G, \, B') &\to \HH^{p+q}(G, \, C)\\
		\HH^{p}(G, \, A') \times \HH^{q-1}(G, \, B'') &\to \HH^{p+q}(G, \, C)
	\end{align*}
	compatibili con le frecce di bordo delle successioni esatte lunghe associate:  $ \delta \alpha \cup \beta + (-1)^p \, a \cup \delta \beta $.
\end{lemma}

\section{Dualità di Tate-Nakayama}

Questo lemma è difficile da digerire tutto assieme, ma sta fondamnetalmente usando il lemma di coomologia banale per generare isomorfismi. Già che c'è, usa tutta la potenza per mescolare tensori e sottogruppi.

\begin{proposition}
	Sia $ f \colon B \to C $ una mappa tra $ G $ moduli. Supponiamo di conoscere, per ogni primo $ p $, un indice $ n_p $ per cui
	\begin{align*}
		\HH^{n_p-1}(G_p,\, B) &\twoheadrightarrow \HH^{n_p-1}(G_p,\, C) & \text{ sia suriettivo,} \\
		\HH^{n_p}(G_p,\, B) &\to \HH^{n_p}(G_p,\, C) & \text{ sia un isomorfismo,} \\
		\HH^{n_p+1}(G_p,\, B) &\hookrightarrow \HH^{n_p+1}(G_p,\, C) & \text{ sia iniettivo.}
	\end{align*}
	Possiamo allora conlcudere che, per ogni modulo $ D $ piatto e ogni sottogruppo $ H < G $, la mappa indotta dal tensore
	\[ \Hh^n(H, \, B \otimes D) \to \Hh^n(H, \, C \otimes D) \qquad\text{ è un isomorfismo, per ogni } n. \]
\end{proposition}

Ora vogliamo metterci il prodotto, che è tipo la stessa cosa, ma aggiunge uno shift degli indici!! Non è sorprendente pensare che andremo d decalage...

\begin{proposition}
	Sia $ \varphi \colon A \times B \to C $ una mappa bilineare tra $ G $ moduli; fissiamo un elemento $ a \in \HH^q(G, \, A) $. Supponiamo di conoscere, per ogni primo $ p $, un indice $ n_p $ per cui l'omomorfismo indotto dal tazza prodotto per $ \Res a $
	\begin{align*}
	\HH^{n_p-1}(G_p,\, B) &\twoheadrightarrow \HH^{n_p+q-1}(G_p,\, C) & \text{ sia suriettivo,} \\
	\HH^{n_p}(G_p,\, B) &\to \HH^{n_p+q}(G_p,\, C) & \text{ sia un isomorfismo,} \\
	\HH^{n_p+1}(G_p,\, B) &\hookrightarrow \HH^{n_p+q+1}(G_p,\, C) & \text{ sia iniettivo.}
	\end{align*}
	Possiamo allora conlcudere che, per ogni modulo $ D $ piatto e ogni sottogruppo $ H < G $, la mappa indotta dal tazza prodotto per $ \Res a $
	\[ \Hh^n(H, \, B \otimes D) \to \Hh^{n+q}(H, \,  C\otimes D) \qquad\text{ è un isomorfismo, per ogni } n. \]
\end{proposition}

\begin{theorem}[Dualità di Tate-Nakayama]
	Sia $ A $ un $ G $-modulo. Fissiamo un elemento $ a \in A $. Supponiamo di sapere , per ogni numero primo $ p $, ch
	\begin{enumerate}
		\item $ \HH^1(G_p, \, A) = 0 $;
		\item il gruppo $ \HH^2(G_p, \, A) $ ha la stessa cardinalità di $ G_p $ ed è generato da $ \Res a $.
	\end{enumerate}
	Possiamo allora concludere che l'omomorfismo indotto dal tazza prodotto con $ \Res a $ induce, per ogni sottogruppo $ H < G $, un isomorfismo
	\[ \Hh^n(H, \, \Z) \to \Hh^{n+2}(H, \,  A) \qquad \text{ per ogni } n.\]
\end{theorem}


\section{Applicazione di reciprocità locale}

% Costruzione della Reciprocità e... poco altro?

\section{Dualità di Tate}

% MIsteriosa esistenza del mdulo dualizzante

% Lemmino

% Trovare il dualizzante

% Dualità di Tate

% Calcolo del gruppo di Galois assoluto


