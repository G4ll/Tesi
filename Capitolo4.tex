\chapter{Dualità}
Ci addentreremo ora 

\section{Tazza Prodotto}
Costruiremo ora un'applicazione bilineare tra i gruppi di coomologia, donando al nostro strumento un prodotto e, dunque, una struttura di algebra. La costruzione del \leftquote tazza-prodotto", così come la presentazione delle relative proprietà, è fondamentalmente una boriosa, lunghissima, verifica: dovremo riesumare la presentazione in cocatene. Come ogni buon libro di algebra, presenteremo dettagliatamente la costruzione del prodotto e la lista delle proprietà che useremo in seguito, lasciando decidere al lettore se fidarsi ciecamente di quanto presentato oppure dedicare un paio di pomeriggi alla verifica diretta. \\

Poniamoci nella dovuta generalità. Siano $ A, \, B $ due $ G $-moduli e $ K_{\mathtt{om}}(A) $ e $ K_{\mathtt{om}}(B) $ i corrispondenti complessi di cocatene omogenee; che ricordiamo si ottengono omogeneizzando le usuali cocatene $ K(A) $ e $ K(B) $: presa $ f \colon G^i \to A $, definiamo $ F\colon G^{i+1} \to A  $ in modo che
\[ F(g_0, \, g_1, \, \dots, \, g_i) = g_0 f(g_0^{-1}g_1, \, g_1^{-1}g_2, \, \dots, \, g_{i-1}^{-1}g_i). \]
Questa presentazione è molto più comoda perché l'applicazione bilineare
\begin{align*}
	\cup \colon K^p_{\mathtt{om}}(A) \times K^q_{\mathtt{om}}(B) &\to K_{\mathtt{om}}^{p+q}(A \otimes B)\\
	(a, \, b) & \mapsto a(g_0,\, \dots, \, g_p) \otimes b(g_p,\, \dots, \, g_{p+q})
\end{align*}
si comporta bene con i differenziali
\[ d(a \cup b) = (da) \cup b + (-1)^p a \cup (db), \]
permettendoci di passare in coomologia.

\begin{definition}[Tazza prodotto]
	Chiamiamo tazza-prodotto l'applicazione $ \Z $-bilineare indotta da dal prodotto appena definito
	\[ \cup \colon \HH^p(G, \, A) \times \HH^q(G, \, B) \to \HH^{p+q}(G, \, A \otimes B)  \]
\end{definition}

Il prodotto è ovviamente compatibilità con le successioni esatte lunghe:

\begin{proposition}\label{cup1}
	Se entrambe le successioni
	\[ 0 \to A \to A' \to A'' \to 0, \qquad 0 \to A \otimes B \to A'\otimes B \to A''\otimes B \to 0 \]
	sono esatte, allora il tazza-prodotto, fissato $ \beta \in \HH^q(G, \, B) $, in induce un morfismo di complessi
	\[\begin{tikzcd}[column sep = small]
	\dots \rar
	& \HH^{p}(G, \, A') \dar["\cup\beta"] \rar 
	& \HH^{p}(G, \, A'') \rar["\delta"] \dar["\cup\beta"]
	& \HH^{p+1}(G, \, A) \rar \dar["\cup\beta"]
	&\dots  \\
	\dots \rar
	& \HH^{p+q}(G, \, A'\otimes B) \rar
	& \HH^{p+q}(G, \, A''\otimes B) \rar["\delta"]
	& \HH^{p+q+1}(G, \, A\otimes B) \rar &\dots \, , \end{tikzcd}\]
	equivalentemente $ (\delta\alpha'') \cup \beta = \delta (\alpha'' \cup \beta) $.
\end{proposition}

Presa una qualunque applicazione $ \Z[G] $-bilineare $ \varphi\colon A \times B \to C $, componendo all'omomorfismo $ \varphi\colon A \otimes B \to C  $ con il tazza-prodotto otteniamo qualunque sorta di prodotto in coomologia. Dalla proposizione di compatibilità precedente possiamo dedurne una versione appena più generale, che ci tornerà utile in futuro.

\begin{lemma}\label{cup2}
	Date due successioni esatte
	\[ 0 \to A' \to A \to A'' \to 0, \qquad 0 \to B' \to B \to B'' \to 0 \]
	e un'applicazione bilineare $  \varphi\colon A \times B \to C $ nulla su $ A'\times B' $. Troviamo allora due accoppiamenti
	\[ \varphi'\colon A' \times B'' \to C, \qquad \varphi''\colon A'' \times B' \to C, \]
	che inducono prodotti in omologia
	\begin{align*}
		\HH^{p+1}(G, \, A'') \times \HH^{q}(G, \, B') &\to \HH^{p+q}(G, \, C)\\
		\HH^{p}(G, \, A') \times \HH^{q-1}(G, \, B'') &\to \HH^{p+q}(G, \, C)
	\end{align*}
	compatibili con le frecce di bordo delle successioni esatte lunghe associate:  $ \delta \alpha \cup \beta + (-1)^p \, a \cup \delta \beta $.
\end{lemma}

\section{Dualità di Tate-Nakayama}
Ci addentreremo ora nell'argomento più tecnico di tutta la tesi: il teorema di dualità id Tate-Nakayama sarà preceduto da alcuni risultati di natura puramente algebrica e carattere incredibilmente generale, per questo difficili da digerire. L'obbiettivo dei risultati seguenti è di trasformare i criteri di banalità, introdotti qualche pagine indietro, in strumenti per costruire isomorfismi tra gruppi di coomologia con coefficienti diversi. Per fare questo assumeremo delle ipotesi ad hoc, della cui ragionevolezza ci occuperemo solo in futuro. 

\begin{proposition}
	Sia $ f \colon B \to C $ una mappa tra $ G $ moduli. Supponiamo di conoscere, per ogni primo $ p $, un indice $ n_p $ per cui
	\begin{align*}
		\Hh^{n_p}(G_p,\, B) &\twoheadrightarrow \Hh^{n_p}(G_p,\, C) & \text{ sia suriettivo,} \\
		\Hh^{n_p+1}(G_p,\, B) &\to \Hh^{n_p+1}(G_p,\, C) & \text{ sia un isomorfismo,} \\
		\Hh^{n_p+2}(G_p,\, B) &\hookrightarrow \Hh^{n_p+2}(G_p,\, C) & \text{ sia iniettivo.}
	\end{align*}
	Possiamo allora concludere che, per ogni modulo $ D $ piatto e ogni sottogruppo $ H < G $, la mappa indotta dal tensore
	\[ \Hh^n(H, \, B \otimes D) \to \Hh^n(H, \, C \otimes D) \qquad\text{ è un isomorfismo, per ogni } n. \]
\end{proposition}

La tesi è assolutamente sopraffacente. È utile pensare al risultato prima nella forma più semplice, con $ H=G $ e $ D = \Z $, seguito da due piccole aggiunte che, essendo gratuite, abbiamo deciso di includere per completezza: che sia valida per ogni sottogruppo non è sorprendente, data la definizione di \leftquote coomologicamente banale"; che si possano estendere i coefficienti, segue invece dal fatto che i moduli piatti preservano gli indotti. 

\begin{proof}
	Le ipotesi sono scelte in modo da funzionare magnificamente una volta assunto $ f $ iniettiva: in questo caso otteniamo una successione esatte corta
	\[ \begin{tikzcd}[column sep = small]
	0 \rar
	& B \rar["f"]
	& C \rar
	& Q \rar
	& 0 \end{tikzcd} \]
	la cui successione esatta lunga associata, per ogni primo $ p $, restituisce
	\[ \Hh^{n_p}(G_p, \, Q) = \Hh^{n_p+1}(G_p, \, Q), \]
	da cui, per il criterio di banalità $ (\ref{ban2}) $, deduciamo che $ Q $ è coomologicamente banale. Abbiamo già osservato che $ Q \otimes D $ rimane tale (\ref{tensor magic}), da cui la tesi. \\
	
	Per concludere è pertanto sufficiente mostrare che è ci si può ricondurre al caso precedente. Per fare questo consideriamo la classica immersione di $ B $ nel suo modulo indotto $ j \colon B \to \Ind_1^G(B) $ e dunque la mappa iniettiva
	\[ (f, j)\colon B \to C \oplus \Ind_1^G(B), \]
	la tesi segue dunque dalla banalità dell'indotto, che trasforma la successione esatta corta
	\[ \begin{tikzcd}[column sep = small]
	0 \rar
	& C \rar["f"]
	& C \oplus \Ind_1^G(B) \rar
	& \Ind_1^G(B) \rar
	& 0 \end{tikzcd} \]
	in degli isomorfismi
	\[ \Hh^n(H, \, C \otimes D ) = \Hh^n \left(H, \, \left(C \oplus \Ind_1^G(B)\right) \otimes D \right). \]
\end{proof}

Ripreso il fiato, concessoci il tempo di capire quanto attentamente sono state scelte le ipotesi, procediamo verso una generalizzazione ancora più spinta. Vorremmo un risultato analogo per mappe indotte dal tazza-prodotto, che come unica differenza producono un cambio di dimensione in arrivo.

\begin{proposition}
	Sia $ \varphi \colon A \times B \to C $ una mappa bilineare tra $ G $ moduli; fissiamo un elemento $ a \in \HH^q(G, \, A) $. Supponiamo di conoscere, per ogni primo $ p $, un indice $ n_p $ per cui l'omomorfismo indotto dal tazza prodotto per $ \Res a $
	\begin{align*}
	\Hh^{n_p}(G_p,\, B) &\twoheadrightarrow \Hh^{n_p+q}(G_p,\, C) & \text{ sia suriettivo,} \\
	\Hh^{n_p+1}(G_p,\, B) &\to \Hh^{n_p+q+1}(G_p,\, C) & \text{ sia un isomorfismo,} \\
	\Hh^{n_p+2}(G_p,\, B) &\hookrightarrow \Hh^{n_p+q+2}(G_p,\, C) & \text{ sia iniettivo.}
	\end{align*}
	Possiamo allora concludere che, per ogni modulo $ D $ piatto e ogni sottogruppo $ H < G $, la mappa indotta dal tazza-prodotto per $ \Res a $
	\[ \Hh^n(H, \, B \otimes D) \to \Hh^{n+q}(H, \,  C\otimes D) \qquad\text{ è un isomorfismo, per ogni } n. \]
\end{proposition}

\begin{proof}
	Giocando a trovare le differenze con la proposizione precedente, si scopre che queste coincidono quando $ a \in \HH^0(G, \, A) $, ovvero assumendo $ q = 0. $ Per concludere è pertanto sufficiente mostrare che ci si può ricondurre al caso precedente. Il modo naturale di farlo è passare ai moduli con coomologia traslata: sostituendo $ A, \, C $ con $ A_1 ,\,  C_1 $, otteniamo un'applicazione bilineare $ \varphi_1\colon A_1 \times B \to C_1 $.
	\[ \begin{tikzcd}
	content
	\end{tikzcd} \]
	
	 \todo[scrivere questa parte]
\end{proof}

Raggiunto il desiderato livello di astrazione, possiamo occuparci di riformulare le particolari ipotesi in una forma più accessibile, se non proprio comprensibile.

\begin{theorem}[Dualità di Tate-Nakayama]
	Sia $ A $ un $ G $-modulo. Fissiamo un elemento $ a \in \HH^2(G, \, A) $. Supponiamo di sapere , per ogni numero primo $ p $, che
	\begin{enumerate}
		\item $ \HH^1(G_p, \, A) = 0 $;
		\item il gruppo $ \HH^2(G_p, \, A) $ ha la stessa cardinalità di $ G_p $, diciamo $ m_p $, ed è generato da $ \Res a $.
	\end{enumerate}
	Possiamo allora concludere che l'omomorfismo indotto dal tazza-prodotto con $ \Res a $ induce, per ogni sottogruppo $ H < G $, un isomorfismo
	\[ \Hh^n(H, \, \Z) \to \Hh^{n+2}(H, \,  A) \qquad \text{ per ogni } n.\]
\end{theorem}

\begin{proof}
	Applichiamo la proposizione precedente con $ B = \Z $, $ C = A $ e $ D = \Z $. Verifichiamo che le ipotesi siano verificate per $ n_p = -1 $:
	\begin{align*}
		\Hh^{-1}(G_p,\, \Z) &\twoheadrightarrow \Hh^{1}(G_p,\, A) = 0, \\
		\Z/m_p\Z=\Hh^{0}(G_p,\, \Z) &\to \Hh^{2}(G_p,\, A) = \Z/m_p\Z, \\
		0 = \Hh^{1}(G_p,\, \Z) &\hookrightarrow \Hh^{3}(G_p,\, A).
	\end{align*}
\end{proof}

\section{Reciprocità Locale}


% Costruzione della Reciprocità e... poco altro?
Continuando nella nostra discesa dall'astrazione verso la Teoria del Corpo di Classe, applichiamo subito il teorema al caso dei campi locali: siano $ K $ un campo locale e $ L / K  $ un'estensione finita di Galois di grado $ n = [L\,\colon K] $. Poniamo $ A = L^\times $ e $ G $ il gruppo di Galois dell'estensione. Per Hilbert 90 l'ipotesi 1 del teorema di Tate-Nakayama è soddisfatta e, grazie al nostro studio del gruppo di Brauer, anche l'ipotesi 2: per uno dei lemmi comparsi nel calcolo del Brauer (in particolare \ref{br2}) sappiamo che $ \HH^2(G, \, L^\times) $ coincide con il sottogruppo di $ n $-torsione di $ \Br K = \Q/\Z $, possiamo pertanto scegliere $ u \in \HH^2(G, \, L^\times) $ per cui $ \inv_K(u) = \frac{1}{n} $. Il teorema produce allora un isomorfismo
\[ \theta\colon \Hh^{-2}(G, \, \Z) \to \Hh^{0}(G, \,  L^\times), \]
che coincide con il tazza-prodotto per $ u. $

\begin{theorem}[Reciprocità Locale]
	Per ogni estensione finita abeliana $ L/K $, abbiamo un isomorfismo
	\[ \omega_L\colon \frac{K^\times}{NL^\times} \to \Gal{L/K}, \]
	noto come \emph{isomorfismo di reciprocità}.
\end{theorem}

\begin{proof}
	Consideriamo l'isomorfismo mappa $ \theta^{-1} $ definito sopra e calcoliamo i due gruppi di coomologia coinvolti. Per definizione
	\[ \Hh^{0}(G, \,  L^\times) = K^\times / NL^\times. \]
	Il primo gruppo di omologia $ \Hh^{-2}(G, \, \Z) = \HH_1(G, \, \Z) $ è l'abelianizzato di $ G $: la successione esatta corta
	\[ 0 \to I_G \to \Z[G] \to \Z \to 0 \]
	produce in omologia, poiché $ \Z[G] $ è indotto, la successione esatta
	\[ 0 \to \HH_1(G,\, \Z) \to I_G/I_G^2 \to \Z \to \Z \to 0. \]
	Da cui deduciamo che $ \HH_1(G,\, \Z) = I_G/I_G^2 = G^\mathtt{ab} $.
\end{proof}


Questa mappa fornisce una descrizione, in qualche senso esplicita, di tutti i gruppi di Galois di estensioni abeliane, fornendo uno strumento dalla evidente importanza che conclude in un qualche senso lo studio delle estensioni abeliane. L'ultimo obiettivo rimane riassumere tutto in un unico risultato: il calcolo del gruppo di Galois della massima estensione abeliana, ovvero l'abelianizzato del gruppo di Galois assoluto $ \Gamma_K $. Passando al limite l'isomorfismo di reciprocità, sulle sottoestensioni finite, otteniamo una mappa
\[ \omega\colon K^\times \to \Gamma_K^{\texttt{ab}}, \]
detta \emph{applicazione di reciprocità}, che è un isomorfismo sul quoziente $ \varprojlim_L K^\times/NL^\times $.

