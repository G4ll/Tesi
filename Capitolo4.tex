\chapter{Risultati di dualità}
Il capitolo seguente è sicuramente il più tecnico, ma non poteva essere diversamente: fino ad ora non abbiamo fatto altro che introdurre il linguaggio dei funtori coomologici e tradurvici quanto già sapevamo, è giunto il momento di mettere in azione questo misterioso macchinario e vedere cosa ne possiamo ricavare.\\

Il grande vantaggio di una teoria coomologica è che possiamo definirci sopra un prodotto. Uno dei fondamentali problemi, è che dovremo effettivamente definirlo. Ci lanceremo dunque in un tentativo di presentare con cura la definizione, in modo che si riesca a credere ai risultati tecnici di buon comportamento del prodotto che seguiranno. Al lettore affamato di dettagli consigliamo \cite{Schmidt}. \\

Seguirà la dimostrazione del teorema di dualità di Tate-Nakayama, risultato che, tramite il prodotto appena definito (dunque in maniera piuttosto oscura), mette in comunicazione gruppi di coomologia con coefficienti in moduli diversi. Questo ci permetterà di reinterpretare la coomologia di un campo spostando i coefficienti su moduli più semplici, $ \Z $ per esempio, e ricavare di conseguenza qualche informazione sul gruppo di Galois in questione. \\

L'ultimo paragrafo sarà il momento in cui tutti i nostri sforzi saranno ripagati: otterremo una descrizione del gruppo di Galois della massima estensione abeliana di un campo locale in termini della struttura del campo stesso, l'applicazione di reciprocità locale.

\section{Tazza Prodotto}
Costruiremo ora un'applicazione bilineare tra i gruppi di coomologia, donando al nostro strumento un prodotto e, dunque, una struttura di algebra. La costruzione del \leftquote tazza-prodotto", così come la presentazione delle relative proprietà, è fondamentalmente una tediosa, lunghissima, verifica: preannuncio solo che dovremo riesumare la descrizione in cocatene. Come ogni buon libro di algebra, presenteremo dettagliatamente la costruzione del prodotto e la lista delle proprietà che useremo in seguito, lasciando decidere al lettore se fidarsi ciecamente di quanto presentato oppure dedicare un paio di pomeriggi alla verifica diretta. \\

Poniamoci nella dovuta generalità. Siano $ A, \, B $ due $ G $-moduli e $ K_{\mathtt{om}}(A) $ e $ K_{\mathtt{om}}(B) $ i corrispondenti complessi di cocatene omogenee, che si ottengono omogeneizzando le usuali cocatene $ K(A) $ e $ K(B) $: presa $ f \colon G^i \to A $, definiamo $ F\colon G^{i+1} \to A  $ in modo che
\[ F(g_0, \, g_1, \, \dots, \, g_i) = g_0 f(g_0^{-1}g_1, \, g_1^{-1}g_2, \, \dots, \, g_{i-1}^{-1}g_i). \]
Questa presentazione è molto più comoda perché l'applicazione bilineare
\begin{align*}
	\cup \colon K^p_{\mathtt{om}}(A) \times K^q_{\mathtt{om}}(B) &\to K_{\mathtt{om}}^{p+q}(A \otimes B)\\
	(a, \, b) & \mapsto a(g_0,\, \dots, \, g_p) \otimes b(g_p,\, \dots, \, g_{p+q})
\end{align*}
si comporta bene con i differenziali
\[ d(a \cup b) = (da) \cup b + (-1)^p a \cup (db), \]
permettendoci di passare in coomologia.

\begin{definition}[Tazza prodotto]
	Chiamiamo tazza-prodotto l'applicazione $ \Z $-bilineare indotta dal prodotto appena definito
	\[ \cup \colon \HH^p(G, \, A) \times \HH^q(G, \, B) \to \HH^{p+q}(G, \, A \otimes B)  \]
\end{definition}

Il prodotto che abbiamo ottenuto è ovviamente funtoriale e compatibile con le successioni esatte lunghe:

\begin{proposition}\label{cup1}
	Se entrambe le successioni
	\[ 0 \to A \to A' \to A'' \to 0, \qquad 0 \to A \otimes B \to A'\otimes B \to A''\otimes B \to 0 \]
	sono esatte, allora il tazza-prodotto, fissato $ \beta \in \HH^q(G, \, B) $, induce un morfismo di complessi
	\[\begin{tikzcd}[column sep = small]
	\dots \rar
	& \HH^{p}(G, \, A') \dar["\cup\beta"] \rar 
	& \HH^{p}(G, \, A'') \rar["\delta"] \dar["\cup\beta"]
	& \HH^{p+1}(G, \, A) \rar \dar["\cup\beta"]
	&\dots  \\
	\dots \rar
	& \HH^{p+q}(G, \, A'\otimes B) \rar
	& \HH^{p+q}(G, \, A''\otimes B) \rar["\delta"]
	& \HH^{p+q+1}(G, \, A\otimes B) \rar &\dots \, , \end{tikzcd}\]
	equivalentemente $ (\delta\alpha'') \cup \beta = \delta (\alpha'' \cup \beta) $.
\end{proposition}

Segue ovviamente una proposizione duale scambiando i ruoli di $ A $ e $ B $. Queste due proprietà, assieme alla funtorialità, determinano in modo unico il prodotto trovato, fissata un'applicazione bilineare in grado 0. È pertanto possibile dare una definizione di tazza-prodotto per i gruppi di Tate \cite[proposizione 1.4.7]{Schmidt}. \\

Presa una qualunque applicazione $ \Z[G] $-bilineare $ \varphi\colon A \times B \to C $, componendo l'omomorfismo $ \varphi\colon A \otimes B \to C  $ con il tazza-prodotto otteniamo una più generale nozione di prodotto in coomologia. Dalla proposizione di compatibilità precedente possiamo dedurre una versione appena più generale, che ci tornerà utile in futuro.

\begin{lemma}\label{cup2}
	Fissiamo due successioni esatte
	\[ 0 \to A' \to A \to A'' \to 0, \qquad 0 \to B' \to B \to B'' \to 0 \]
	e un'applicazione bilineare $  \varphi\colon A \times B \to C $ nulla su $ A'\times B' $. Troviamo allora due accoppiamenti
	\[ \varphi'\colon A' \times B'' \to C, \qquad \varphi''\colon A'' \times B' \to C, \]
	che inducono prodotti in coomologia
	\begin{align*}
		\HH^{p}(G, \, A'') \times \HH^{q}(G, \, B') &\to \HH^{p+q}(G, \, C)\\
		\HH^{p+1}(G, \, A') \times \HH^{q-1}(G, \, B'') &\to \HH^{p+q}(G, \, C)
	\end{align*}
	compatibili a meno del segno con le frecce di bordo delle successioni esatte lunghe associate:  $$  \delta \alpha \cup \beta + (-1)^p \, a \cup \delta \beta = 0.  $$
\end{lemma}

\section{Dualità di Tate-Nakayama}
Ci addentreremo ora nell'argomento più tecnico di tutta la tesi: il teorema di dualità di Tate-Nakayama, che sarà preceduto da alcuni risultati di natura puramente algebrica e dal carattere incredibilmente generale. L'obbiettivo dei risultati seguenti è di trasformare i criteri di banalità, introdotti qualche capitolo addietro, in strumenti per costruire isomorfismi tra gruppi di coomologia con coefficienti diversi. Per fare questo cominceremo assumendo ipotesi ad hoc, della cui ragionevolezza ci preoccuperemo solo in futuro. 

\begin{proposition}
	Sia $ f \colon B \to C $ una mappa tra $ G $ moduli. Supponiamo di conoscere per ogni primo $ p $ un indice $ n_p $ per cui
	\begin{align*}
		\Hhnp(G_p,\, B) &\twoheadrightarrow \Hhnp(G_p,\, C) & \text{ sia suriettivo,} \\
		\Hhnpp(G_p,\, B) &\to \Hhnpp(G_p,\, C) & \text{ sia un isomorfismo,} \\
		\Hhnppp(G_p,\, B) &\hookrightarrow \Hhnppp(G_p,\, C) & \text{ sia iniettivo.}
	\end{align*}
	Possiamo allora concludere che, per ogni modulo $ D $ piatto e ogni sottogruppo $ H < G $, la mappa indotta dal tensore
	\[ \Hhn(H, \, B \otimes D) \to \Hhn(H, \, C \otimes D) \qquad\text{ è un isomorfismo, per ogni } n. \]
\end{proposition}

La tesi è sopraffacente. È utile pensare al risultato prima nella forma più semplice, con $ H=G $ e $ D = \Z $, in cui sono stati semplicemente prodotti degli isomorfismi infilando abbastanza zeri nella successione esatta lunghissima associata; a cui seguono due piccole aggiunte che, essendo gratuite, abbiamo deciso di includere per completezza: che sia valida per ogni sottogruppo non è sorprendente, dato che la definizione di \leftquote coomologicamente banale" è una condizione rispetto ai sottogruppi; che si possano estendere i coefficienti, segue invece dal fatto che i moduli piatti preservano gli indotti $ (\ref{tensor magic}) $. 

\begin{proof}
	Le ipotesi sono scelte in modo da funzionare magnificamente una volta assunto che $ f $ sia iniettiva: in questo caso otteniamo una successione esatta corta
	\[ \begin{tikzcd}[column sep = small]
	0 \rar
	& B \rar["f"]
	& C \rar
	& Q \rar
	& 0 \end{tikzcd} \]
	la cui successione esatta lunga associata, per ogni primo $ p $, restituisce
	\[ \Hhnp(G_p, \, Q) = \Hhnpp(G_p, \, Q)=0, \]
	da cui, per il criterio di banalità $ (\ref{ban2}) $, deduciamo che $ Q $ è coomologicamente banale. Abbiamo già osservato che $ Q \otimes D $ rimane tale (\ref{tensor magic}), da cui la tesi. \\
	
	Per concludere è pertanto sufficiente mostrare che  ci si può ricondurre al caso precedente. Per fare questo consideriamo la classica immersione di $ B $ nel suo modulo indotto $ j \colon B \hookrightarrow \Ind_1^G(B) $ e dunque la mappa iniettiva
	\[ (f, j)\colon B \hookrightarrow C \oplus \Ind_1^G(B), \]
	la tesi segue dunque dal caso precedente e dalla banalità dell'indotto, che trasforma la successione esatta corta
	\[ \begin{tikzcd}[column sep = small]
	0 \rar
	& C \rar["f"]
	& C \oplus \Ind_1^G(B) \rar
	& \Ind_1^G(B) \rar
	& 0 \end{tikzcd} \]
	in degli isomorfismi
	\[ \Hhn(H, \, C \otimes D ) = \Hhn \left(H, \, \left(C \oplus \Ind_1^G(B)\right) \otimes D \right).  \qedhere \]
\end{proof}

Ripreso il fiato, concessoci il tempo di capire quanto attentamente sono state scelte le ipotesi, procediamo verso una generalizzazione ancora più spinta. Vorremmo un risultato analogo per mappe indotte dal tazza-prodotto, che come unica differenza producono un cambio di dimensione in arrivo.

\begin{proposition}
	Sia $ \varphi \colon A \times B \to C $ una mappa bilineare tra $ G $ moduli; fissiamo un elemento $ a \in \HH^q(G, \, A) $. Supponiamo di conoscere, per ogni primo $ p $, un indice $ n_p $ per cui l'omomorfismo indotto dal tazza prodotto per $ \Res a $
	\begin{align*}
	\Hhnp(G_p,\, B) &\twoheadrightarrow \Hhnpq(G_p,\, C) & \text{ sia suriettivo,} \\
	\Hhnpp(G_p,\, B) &\to \Hhnppq(G_p,\, C) & \text{ sia un isomorfismo,} \\
	\Hhnppp(G_p,\, B) &\hookrightarrow \Hhnpppq(G_p,\, C) & \text{ sia iniettivo.}
	\end{align*}
	Possiamo allora concludere che, per ogni modulo $ D $ piatto e ogni sottogruppo $ H < G $, la mappa indotta dal tazza-prodotto per $ \Res a $
	\[ \Hhn(H, \, B \otimes D) \to \Hhnq(H, \,  C\otimes D) \qquad\text{ è un isomorfismo, per ogni } n. \]
\end{proposition}

\begin{proof}
	Giocando a trovare le differenze con la proposizione precedente, si scopre che queste coincidono quando $ a \in \HH^0(G, \, A) $, ovvero assumendo $ q = 0. $ Per concludere è pertanto sufficiente mostrare che ci si può ricondurre al caso precedente. Il modo naturale di farlo è passare ai moduli con coomologia traslata: cominciamo dal caso $ q = 1 $. Sostituendo $ A, \, C $ con $ A_1 ,\,  C_1 $, otteniamo un'applicazione bilineare $ \varphi_1\colon A_1 \times B \to C_1 $ abbastanza naturalmente
	\[ \begin{tikzcd}[column sep = small]
	0 \rar
	& A \rar \dar["\cdot_\varphi\beta"]
	& \Ind_1^G(A) \rar \dar["\cdot_\varphi\beta"]
	& A_1 \rar \dar["\cdot_{\varphi_1}\beta", dashed]
	& 0\\
	0 \rar & C \rar & \Ind_1^G(C) \rar & C_1 \rar & 0.  \end{tikzcd} \]
	Prendiamo l'elemento $ a_1 \in \Hhz(G, \, A_1) $ corrispondente ad $ a $: l'unico per cui $ \delta(a_1) = a $. La tesi è vera per il prodotto indotto da $ \varphi_1 $ e l'elemento $ a_1 $, abbiamo pertanto dei meravigliosi isomorfismi
	\[ \begin{tikzcd}
	\Hhn(H, \, B \otimes D) \rar["a_1\cup"] & \Hhn(H, \,  C_1 \otimes D)
	\end{tikzcd}  \]
	a cui possiamo aggiungere il classico isomorfismo di collegamento
	\[ \begin{tikzcd}
	\Hhn(H, \, B \otimes D) \rar["a_1\cup"] & \Hhn(H, \,  C_1 \otimes D) \rar["\delta"] & \Hhnu(G, \, C \otimes D).
	\end{tikzcd}  \]
	Deduciamo la tesi grazie al risultato di compatibilità del tazza-prodotto (\ref{cup1}): l'isomorfismo trovato è infatti $$  x \mapsto \delta(\Res a_1 \cup x) = \delta(\Res a_1) \cup x = \Res a \cup x.  $$ Analogamente si procede per ogni $ q $.
\end{proof}

Raggiunto il desiderato livello di astrazione, possiamo occuparci di riformulare le particolari ipotesi in una forma più accessibile, se non proprio comprensibile.

\begin{theorem}[Dualità di Tate-Nakayama]
	Sia $ A $ un $ G $-modulo. Fissiamo un elemento $ a \in \HH^2(G, \, A) $. Supponiamo di sapere, per ogni numero primo $ p $, che
	\begin{enumerate}[label = \roman*.]
		\item $ \HH^1(G_p, \, A) = 0 $;
		\item il gruppo $ \HH^2(G_p, \, A) $ ha la stessa cardinalità di $ G_p $, diciamo $ m_p $, ed è generato da $ \Res a $.
	\end{enumerate}
	Possiamo allora concludere che l'omomorfismo indotto dal tazza-prodotto con $ \Res a $ induce, per ogni sottogruppo $ H < G $, un isomorfismo
	\[ \Hhn(H, \, \Z) \to \Hhnd(H, \,  A) \qquad \text{ per ogni } n.\]
\end{theorem}

\begin{proof}
	Applichiamo la proposizione precedente con $ B = \Z $, $ C = A $ e $ D = \Z $. Le ipotesi sono verificate per $ n_p = -1 $:
	\begin{align*}
		\Hhm(G_p,\, \Z) &\twoheadrightarrow \Hhu(G_p,\, A) = 0 & \text{per l'ipotesi } i, \\
		\Z/m_p\Z=\Hhz(G_p,\, \Z) &\to \Hhd(G_p,\, A) = \Z/m_p\Z & \text{per l'ipotesi } ii,\\
		0 = \Hhu(G_p,\, \Z) &\hookrightarrow \Hht(G_p,\, A) & \text{per l'ipotesi } i. 
	\end{align*}
	\qedhere
\end{proof}
Al contrario delle proposizioni precedenti, le ipotesi del teorema di dualità sembrano avere qualcosa a che fare con quanto visto fin'ora. Il risultato dovrebbe però chiarire in parte perché abbiamo deciso di percorrere questa strada: cambiando i coefficienti da un modulo qualunque a $ \Z $, riusciamo a spostare risultati specifici, per esempio le proprietà aritmetiche di un campo, a gruppi che invece conosciamo esplicitamente, per dire $ \HH_1(G, \, \Z) $ che, classicamente, coincide con l'abelianizzato $ G^\texttt{ab} $. Nella sezione seguente ci addentreremo più approfonditamente nell'argomento.
\section{Reciprocità Locale}

% Costruzione della Reciprocità e... poco altro?
Continuando nella nostra discesa dall'astrazione verso la Teoria del Corpo di Classe, applichiamo il teorema di dualità di Tate-Nakayama al caso dei campi locali: ne otterremo una descrizione dei gruppi di Galois in funzione della struttura del campo stesso. Siano $ K $ un campo locale e $ L / K  $ un'estensione finita di Galois di grado $ n = [L\,\colon K] $; chiamiamo $ G $ il gruppo di Galois dell'estensione e poniamo $ A = L^\times $. Per Hilbert 90 la prima ipotesi del teorema di Tate-Nakayama è soddisfatta 
\[ i. \; \HH^1(G_p,\, L^\times) = 0 \]
e, grazie al nostro studio del gruppo di Brauer, anche la seconda: per uno dei lemmi comparsi nel calcolo del Brauer (in particolare \ref{br2}) sappiamo che $ \HH^2(G, \, L^\times) $ coincide con il sottogruppo di $ n $-torsione di $ \Br K = \Q/\Z $, possiamo pertanto scegliere l'unico elemento $ u \in \HH^2(G, \, L^\times) $ per cui $ \inv_K(u) = \frac{1}{n} $; per il lemma di compatibilità tra invarianti e restrizione (\ref{inv}), si ha che
\[ ii. \; \HH^2(G_p, \, L^\times) = T_p \Br(L/K) \;\text{ è generato da } \Res u. \]
Il teorema di dualità produce allora un isomorfismo
\[ \theta\colon \Hhmm(G, \, \Z) \to \Hhz(G, \,  L^\times), \]
che coincide con il tazza-prodotto per $ u. $ Il primo gruppo è proprio con $ G^\text{ab} $: la successione esatta lunga in omologia associata a 
\[ 0 \to I_G \to \Z[G] \to \Z \to 0 \]
fornisce infatti l'isomorfismo richiesto: vi troviamo infatti $ \HH^1(G, \, \Z) \to I_G/I_G^2 $, che ci permette di concludere perché la mappa
\begin{align*}
	G^\texttt{ab} &\to I_G/I_G^2 \\
	g& \mapsto g-1
\end{align*}
è un isomorfimo; per esplicitare il secondo gruppo è invece sufficiente andare a riprendere la definizione \eqref{tatedef}. Ne segue dunque che

\begin{theorem}[Reciprocità Locale]
	Ogni estensione abeliana finita $ L/K $ è accompagnata da un isomorfismo
	\[ \omega_L\colon \frac{K^\times}{NL^\times} \to \Gal{L/K}, \]
	dove $ N $ è, come al solito, la norma.
\end{theorem}

%\begin{proof}
%	Consideriamo l'isomorfismo $ \theta^{-1} $ definito sopra e calcoliamo i due gruppi di coomologia coinvolti. Per definizione
%	\[ \Hh^{0}(G, \,  L^\times) = K^\times / NL^\times. \]
%	Il primo gruppo di omologia $ \Hh^{-2}(G, \, \Z) = \HH_1(G, \, \Z) $ è l'abelianizzato di $ G $: la successione esatta corta
%	\[ 0 \to I_G \to \Z[G] \to \Z \to 0 \]
%	produce in omologia, poiché $ \Z[G] $ è indotto, la successione esatta
%	\[ 0 \to \HH_1(G,\, \Z) \to I_G/I_G^2 \to \Z \to \Z \to 0. \]
%	Da cui deduciamo che $ \HH_1(G,\, \Z) = I_G/I_G^2 = G^\mathtt{ab} $.
%\end{proof}

Questo conclude in un qualche senso lo studio delle estensioni abeliane di un campo locale. Per esempio, possiamo annunciare trionfanti di essere riusciti a calcolare il gruppo di Galois della massima estensione abeliana, ovvero l'abelianizzato del gruppo di Galois assoluto $ \Gamma_K $: prendendo il limite inverso di $ \omega_L $ sulle estensioni abeliane finite, otteniamo una mappa
\[ \omega\colon K^\times \to \Gamma_K^{\texttt{ab}}, \]
detta \emph{applicazione di reciprocità}, che su $ \varprojlim_L K^\times / NL^\times  $ è un isomorfismo:
\begin{equation}\label{lim}
	\Gamma_K^{\texttt{ab}} = \varprojlim_L \frac{K^\times}{NL^\times}.
\end{equation}
Tornando con i piedi per terra, soffermiamoci su quella fastidiosa $ N $ al denominatore: come speriamo di calcolare esplicitamente questo gruppo? Come si calcola l'immagine tramite la norma di una data estensione? Studiando più approfonditamente l'applicazione di reciprocità è possibile raffinare il risultato, arrivando a mostrare che il limite in questione coincide proprio con la definizione di completamento profinito: sarà sufficiente mostrare che $ NL^\times $ spazia fra tutti e soli i sottogruppi aperti (e di indice finito) di $ K^\times $ (\cite[XIV.\S 6.1]{SerreFields}). Nel paragrafo successivo ci concentreremo sul caso dei campi $ p $-adici, per cui aggireremo il problema mostrando che
\[ \Gamma^\text{ab}_K = \varprojlim_n \frac{K^\times}{{K^\times}^n}, \]
limite che a sua volta coincide con il completamento profinito $ \hat{K}^\times $.
% assicurarsi della presenza dell'applicazione di reciprocità nella dualità di Tate

\section{Dualità di Tate}
Ci concentriamo ora sui campi $ p $-adici, principalmente perché la struttura aritmetica presenta la seguente proprietà, di cui ometteremo la dimostrazione, essendo un risultato classico in teoria dei numeri \cite[proposizione 4.11]{Harari}.

% Lemmino dei p-adici
\begin{proposition}
	Sia $ K $ un campo $ p $-adico, i sottogruppi $ {K^\times}^n $ sono aperti.
\end{proposition}

Quest'osservazione è fondamentale qualunque strada si cerchi di intraprendere per calcolare il limite desiderato \eqref{lim}, stiamo infatti mostrando che i sottogruppi $ {K^\times}^n $ sono aperti con l'obbiettivo di trovare una successione cofinale su cui calcolare sia il limite del completamento che il limite \eqref{lim}: che questa sia cofinale tra gli aperti è immediato, infatti ogni sottogruppo $ U $ aperto ha indice finito $ n $ e pertanto deve contenere $ {K^\times}^n $. Se riuscissimo a mostrare che sono cofinali anche negli $ NL^\times $ avremmo concluso, ma non è il percorso che intraprenderemo.

\begin{corollary} \label{finiti}
	Sia $ A $ un $ \Gamma_K $-modulo finito. I gruppi $ \HH^i(K, \, A) $ sono finiti per ogni $ i > 0 $.
\end{corollary}
\begin{proof}
	Avendo ogni campo locale dimensione coomologica $ 2 $, sarà sufficiente concentrarsi sui gradi bassi $ i = 0, \, 1, \, 2 $. Osserviamo che per le radici $ n $-esime dell'unità $ A = \mu_n $ la tesi è già stata mostrata: in $ i = 0 $ per definizione, in $ i = 1 $ è la proposizione appena enunciata, in $ i = 2 $ si tratta della $ n $-torsione del Brauer. Preso un modulo $ A $ di cardinalità $ n $, troviamo un sottogruppo normale e aperto di $ U < \Gamma_K $ che agisce banalmente sia su $ A $ che su $ \mu_n $, per esempio dentro l'intersezione degli stabilizzatori di tutti gli elementi. Pensato come $ U $-modulo, $ A $ ha solo la struttura di gruppo abeliano, possiamo pertanto pensarlo come somma diretta di finiti $ \mu_{n_i} $, per qualche $ n_i \mid n $. Avendo già mostrato il teorema per le radici dell'unità, scopriamo che $ \HH^i(U, \, A) $ è finito per ogni $ i \geq 0 $. Concludiamo sfruttando la successione spettrale di \HS, che in pagina due presenta solo moduli finiti
	\[ \HH^p(G/U, \, \HH^q(U, \, A)), \]
	i cui quozienti, necessariamente finiti, filtrano $ \HH^{p+q}(G, \, A) $, che risulterà pertanto finito a sua volta.
\end{proof}

Prima di poter enunciare il teorema di dualità di Tate, dobbiamo introdurre una serie di dualità, diverse ma dallo stesso sapore: una per i gruppi, classica, e una meno nota per i moduli.

\begin{definition}[Duale di Pontryagin]
	Per ogni gruppo $ G $, definiamo il duale secondo Pontryagin come
	\[ G^\vee = \Hom_\mathsf{Gr}(G, \, \Q/\Z). \]
\end{definition}

Osserviamo innanzitutto che, in analogia alla classica dualità tra spazi vettoriali, la dualità tra due gruppi abeliani finiti $ A, \, B $ è equivalente all'esistenza di un accoppiamento perfetto
\[ A \times B \to \Q/\Z, \]
ovvero una mappa bilineare non degenere.

\begin{proposition}
	Il biduale di un gruppo profinito è canonicamente isomorfo all'abelianizzato del gruppo stesso. 
\end{proposition}
\begin{proof}
	Il risultato è evidente per i gruppi abeliani finiti, per cui possiamo costruire esplicitamente l'isomorfismo che associa ad ogni elemento la corrisponendte valutazione
	\begin{align*}
		G & \to \Hom_\Z(G^\vee, \, \Q/\Z) \\
		g & \mapsto \psi_g \colon \varphi \mapsto \varphi(g).
	\end{align*}
	Passando al limite raggiungiamo tutti i gruppi abeliani profiniti. Infine, osserviamo che
	\[ \Hom_\mathsf{Gr}(G, \, \Q/\Z)  = \Hom_\Z(G^\texttt{ab}, \, \Q/\Z) \]
	per la proprietà universale dell'abelianizzato, da cui la tesi.
\end{proof}

Passiamo ora alla dualità dei moduli: vorremo definire la dualità nello stesso modo, ma abbiamo bisogno quantomeno di una topologia su $ \Q/\Z $. Poiché la scelta non è univoca, ci limitiamo a farlo in un caso in cui è quantomeno naturale: quando il gruppo in questione è un gruppo di Galois $ \Gamma $, pensiamo $ \Q/\Z $ come il gruppo di tutte le radici dell'unità $ \mu_\infty < \bar{K}^\times $. Equivalentemente, possiamo pensare al duale come alle mappe in $ \K^\times $, che avranno però necessariamente immagine nel suo sottogruppo di torsione: $ \mu_\infty. $

\begin{definition}[Duale di Cartier]
	Per ogni $ \Gamma $-modulo $ A $, definiamo il duale secondo Cartier come il gruppo
	\[ A^* = \Hom_\Gamma(A, \, \mu_\infty) \]
	munito della sua naturale struttura di $ \Gamma $-modulo, quella che si ottiene scegliendo la topologia della convergenza puntuale e lasciando agire $ \Gamma $ a sinistra sulle mappe: $ g \cdot f \colon x \mapsto f(gx) $.
\end{definition}

\begin{proposition}
	Il biduale di un modulo finito è canonicamente isomorfo al modulo stesso.
\end{proposition}

% Misteriosa esistenza del modulo dualizzante
Presentiamo ora un risultato generale di algebra omologica, nella cui dimostrazione, pavidamente, non ci addentreremo \cite[I.3.6]{SerreGalois}.
\begin{proposition}
	Sia $ G $ un gruppo profinito di dimensione coomologica finita $ \cd(G) = d $. Se il funtore $ \HH^d(G,\, \bullet\,) $ manda moduli finiti in gruppi finiti, il suo duale $ \HH^d(G,\, \bullet\,)^\vee $ è rappresentabile sulla sottocategoria dei moduli finiti: ovvero esiste un modulo $ I $ di torsione per cui
	\begin{equation}\label{du}
		\HH^d(G, \, A )^\vee = \Hom_G(A , \, I) \qquad \text{ per ogni } A \text{ finito.}
	\end{equation}
\end{proposition}

\begin{definition}
	Non contenti della quantità di dualità introdotto finora, chiamiamo \emph{modulo dualizzante} di $ G $ uno $ G $-modulo $ I $ per cui valga la $ \eqref{du} $.
\end{definition}

% Dire che possiamo passare au moduli di torsione con passaggio al limite, dire che su Hom c'è una topologia!
L'ipotesi di finitezza è soffocante. Il meglio che riusciamo a fare è estendere \eqref{du} a moduli di torsione: scrivendo $ A $ di torsione come unione di moduli finiti $ \varinjlim_i A_i $, tramite la solita proposizione $ \ref{limite} $ di passaggio al limite otteniamo infatti
\[ \HH^d(G, \, \varinjlim_i A_i )^\vee = \varprojlim_i \HH^d(G, \, A_i )^\vee = \varprojlim_i \Hom_G(A_i , \, I) = \Hom_G(\varinjlim_i A_i , \, I). \]

% Intro
Fissiamo ora un campo $ p $-adico $ K $ e il suo gruppo di Galois assoluto $ \Gamma $. Abbiamo calcolato, nel capitolo scorso (\ref{cdim2}), che la dimensione coomologica di $ \Gamma $ è 2: troviamo quindi un modulo dualizzante $ I $ per cui 
\[ \HH^2(G, \, \bullet \, )^\vee = \Hom_G(\,\bullet\, , \, I). \]


% Lemmino


\begin{lemma}
	Sia $ G $ un gruppo di dimensione coomologica $ n $. Detto $ I $ un suo modulo dualizzante, questo è dualizzante per ogni sottogruppo $ U < G $ aperto.
\end{lemma}
\begin{proof}
	Il risultato segue immediatamente dal Lemma di Shapiro (\ref{Shapiro}):
	\[ \HH^n(U, \, A)^\vee = \HH^n(G, \, \Ind_U^G(A))^\vee = \Hom_G(\Ind_U^G(A), \, I) = \Hom_U(A, \, \f I). \qedhere \]
\end{proof}

% Dire che iniziamo a lavorare coi p-adici

% Trovare il dualizzante
\begin{proposition}
	Il modulo dualizzante di $ \Gamma $ è canonicamente isomorfo al modulo di tutte le radici dell'unità $ \mu_\infty $.
\end{proposition}
\begin{proof}
	Sia $ I $ un modulo dualizzante e $ I[\, n\, ] $ il sottomodulo di $ n $-torsione, nucleo della moltiplicazione per $ n $:
	\[ \begin{tikzcd}[column sep = small]
	0 \rar & I[\, n \,] \rar & I \rar["\cdot n"] & I \rar & 0.
	\end{tikzcd} \]
	Per definizione di modulo dualizzante, essendo $ I $ dualizzante per ogni sottgruppo aperto $ U $ di $ \Gamma $, abbiamo un isomorfismo
	\begin{equation}\label{dualiz}
	\Hom_U(\mu_n, \, I) \to \HH^2(U, \, \mu_n)^\vee;
	\end{equation}
	avendo impiegato un'intero capitolo per calcolarlo, conosciamo ora il gruppo di Brauer dell'estensione finita relativa a $ U $ e sappiamo dunque che $$  \HH^2(U, \, \mu_n) = (\Br \bar{K}^U)[\, n\, ] = \Z/n\Z.  $$
	Osservando infine che essendo $ \mu_n $ di $ n $-torsione così sarà la sua immagine, possiamo riscrivere l'isomorfismo \eqref{dualiz} di sopra come
	\[ \Hom_U(\mu_n, \, I_n) \to \Hom_U(\mu_n, \, I) \to \HH^2(U, \, \mu_n)^\vee \to \Z/n\Z. \]
	Ne deduciamo che l'azione di $ \Gamma $ su $ \Hom_\Z(\mu_n, \, I_n) $ è sempre banale: se così non fosse lo stabilizzatore di un qualche elemento sarebbe un sottogruppo proprio aperto $ U $, il che ci porterebbe a concludere che questo elemento non può appartenere a $ \Hom_U(\mu_n, \, I_n) $, contraddicendo l'isomorfismo appena ottenuto. Ne segue che un isomorfismo di gruppi $ \mu_n \to I_n $ è automaticamente $ \Gamma $-equivariante: possiamo prendere, per esempio, l'isomorfismo $ f_n \in \Hom_U(\mu_n, \, I_n) $ che corrisponde a $ 1 $ in $ \Z/n\Z $ (è chiaro che sia un isomorfismo non appena ci si dimentica della struttura di modulo su $ \mu_n $ e $ I_n $). Infine, assembliamo l'isomorfismo
	\[ \left(f = \bigcup f_n\right) \colon \left(\mu_\infty = \bigcup \mu_n\right) \to \left(I = \bigcup I_n\right). \qedhere \]
\end{proof}

% Dualità di Tate
Siamo finalmente pronti a enunciare il risultato principale di questo capitolo.
\begin{theorem}[Dualità di Tate]
	Sia $ K $ un campo $ p $-adico e $ M $ un $ \Gamma $-modulo finito. Il tazza-prodotto induce un accoppiamento di dualità tra gruppi finiti
	\[ \HH^i(\Gamma, \, A) \times \HH^{2-i}(\Gamma, \, A^*) \to \HH^2(\Gamma, \, \bar{K}^\times) = \Q / \Z \]
	per $ i = 0, \, 1, \, 2 $.
\end{theorem}

% commento?

\begin{proof}
	Cominciamo osservando che, a rigore, l'immagine del tazza-prodotto dovrebbe avere i coefficienti in $ A \otimes A^* $, che però con la mappa data dalla dualità di Cartier:
	\[ A \otimes \Hom(A,\, \mu_\infty) \to \mu_\infty. \]
	Partiamo dal caso $ i = 0 $, da cui segue $ i = 2 $ per simmetria. Tutto il lavoro necessario è in realtà contenuto nella proposizione precedente, otteniamo infatti la tesi da
	\begin{align*}
	\HH^2(\Gamma, \, A)^\vee & = \Hom_\Gamma(A, \, \mu_\infty) & \text{per via del dualizzante,} \\
	& = \Hom_\Z(A, \, \mu_\infty)^\Gamma & \text{per definizione,} \\
	& = \HH^0(\Gamma, \, A^*) & \text{per definizione.}
	\end{align*}
	Affrontiamo ora il caso $ i = 1 $. Essendo i gruppi in questione finiti per quanto visto a inizio capitolo (\ref{finiti}), è sufficiente mostrare che l'omomorfismo di dualità
	\[ \HH^1(\Gamma, \, A) \to \HH^1(\Gamma, \, A^*)^\vee \]
	è iniettivo: per simmetria lo sarà anche nell'altra direzione. Abbiamo bisogno di applicare una delle misteriose proprietà del tazza-prodotto: la \ref{cup2}, in particolare. Le ipotesi sono facilmente verificate avendo, poiché $ \f\mu_\infty $ è un modulo iniettivo, due successioni esatte corte
	\[ 0 \to A \to \Ind_1^\Gamma(A) \to A_1 \to 0, \qquad 0 \to A_1^* \to \Ind_1^\Gamma(A)^* \to A^* \to 0 \]
	sui cui moduli iniziali il tazza-prodotto è evidentemente nullo:
	\[ A \otimes A_1^* = A \otimes \Hom(\Ind_1^\Gamma(A)/A, \, \mu_\infty) = 0. \]
	Otteniamo dunque due accoppiamenti di dualità 
	\begin{align*}
	\HH^{1}(\Gamma, \, A_1) \times \HH^{1}(\Gamma, \, A_1^*) &\to \HH^{2}(\Gamma, \, \bar{K}^\times),\\
	\HH^{0}(\Gamma, \, A) \times \HH^{2}(G, \, A^*) &\to \HH^{2}(\Gamma, \, \bar{K}^\times)
	\end{align*}
	compatibili con le mappe di transizione, che quindi possiamo incastrare nel diagramma commutativo a meno del segno
	\[\begin{tikzcd}[column sep = small]
	\HH^0(\Gamma, \, \Ind_1^\Gamma(A)) \rar\dar
	& \HH^0(\Gamma, \, A_1) \rar\dar
	& \HH^1(\Gamma, \, A) \rar\dar
	& 0 \\
	\HH^2(\Gamma, \, \Ind_1^\Gamma(A)^*)^\vee \rar
	& \HH^2(\Gamma, \, A_1^*)^\vee \rar
	& \HH^1(\Gamma, \, A^*)^\vee,
	&
	\end{tikzcd}  \]
	in cui le mappe orizzontali sono parte delle solite successioni esatte lunghe associate alle due corte riportate sopra (quella sotto, dualizzata), mentre le mappe verticali sono i morfismi di dualità indotti dal tazza-prodotto; per la parte di teorema già dimostrata, i primi due omomorfismi verticali sono isomorfismi: se i due moduli in questione non fossero finiti, potremmo comunque estendere l'isomorfismo passando al limite, essendo i due duali di torsione. Segue la tesi per diagram-chasing.
\end{proof}


% Calcolo del gruppo di Galois assoluto
Con un piccolo trucco, riusciamo a sfruttare questo risultato di dualità per calcolare il gruppo $ \Gamma^\texttt{ab} $, completando così lo studio della teoria del campo di classe per campi $ p $-adici.

\begin{theorem}
	$ \Gamma^\mathtt{ab} $ è isomorfo al completamento profinito $ \hat{K}^\times $.
\end{theorem}
\begin{proof}
	\begin{align*}
	\Gamma^\mathtt{ab}
	& = \Hom_\mathsf{Gr}(\Gamma,\, \Q/\Z)^\vee & \text{per dualità di Pontryagin} \\[0.1em]
	& = \Hom_\mathsf{Gr}(\Gamma,\, \textstyle\varinjlim_n \Z/n\Z)^\vee & \text{piccolo trucco}\\ 
	& = \textstyle\varprojlim_n \Hom_\mathsf{Gr}(\Gamma,\, \Z/n\Z)^\vee & \text{per abstract nonsense} \\
	& = \textstyle\varprojlim_n \HH^1(\Gamma,\, \Z/n\Z)^\vee & \text{perché l'azione è banale} \\
	& = \textstyle\varprojlim_n \HH^1(\Gamma,\, \mu_n) & \text{per dualità di Tate} \\
	& = \textstyle\varprojlim_n \K^\times / {\K^\times}^n & \text{dalla successione esatta lunga} \\
	& = \hat{K}^\times. & \text{per cofinalità di } {\K^\times}^n \text{ negli aperti.}
	\end{align*}
\end{proof}
