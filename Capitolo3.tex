\chapter{Campi locali}
In questo capitolo faremo fruttare quanto preparato in precedenza: cominceremo a parlare di estensioni di campi e a studiare la coomologia dei relativi gruppi di Galois. Inizieremo con qualche risultato di carattere generale riguardo le estensioni di Galois, dalle nozioni di base alla teoria di Kummer, per restringere poi la nostra attenzione ai campi locali. \\

% Discorso su perché i campi locali.
Per campo locale intendiamo un campo completo rispetto a una valutazione discreta di rango 1. Si può pensare, senza perdita di generalità, a estensioni finite dei $ p $-adici $ \Q_p $ o delle serie Laurent su campi finiti $ \F_p((t)) $. Studiamo questi campi per almeno due buone ragioni; in primis, perché presentano una struttura relativamente semplice: i relativi anelli degli interi sono locali (da cui il nome) e le unità sono di facile descrizione; in secondo luogo, perché sono un ottimo punto di partenza per dedurre risultati analoghi anche nel caso globale, trattazione che però esula dagli scopi di questa tesi. \\

% Cos'è la Class Field Theory?
Il capitolo sarà dunque rivolto a calcolare la coomologia dei moduli intrinsecamente associati a un'estensione di Galois di un campo locale, come il gruppo additivo e moltiplicativo del campo soprastante: mostreremo che in grado uno è sempre nulla, la calcoleremo il grado due e mostreremo che dal grado tre in poi torna a essere nulla.

\section{Teoria di Galois}
Sia $ k $ un campo. Fissiamo una chiusura separabile $ \bar{k} $. D'ora in avanti denoteremo, per brevità, con $ \Gamma_k $ il gruppo di Galois assoluto $ \Gal{\bar{k} / k} $, sostituendolo talvolta con $ k $ stesso: in questo senso parlaremo impropriamente di coomologia di $ k $, per esempio scrivendo
\[ \HH^i(k, \, A) := \HH^i(\Gamma_k, \, A). \]
Osserviamo che, presa una diversa chiusura separabile $ \bar{k}' $ e un sollevamento di $ \id_{k} $ a un isomorfismo $ f\colon \bar{k} \to \bar{k}' $, questo induce un isomorfismo in coomologia
\[ f^* \colon \HH^i(\Gal{\bar{\K} / \K}, \, A) \to \HH^i(\Gal{\bar{\K}' / \K}, \, A) \]
che non dipende  dalla scelta di $ f $: possiamo cambiare l'immersione $ f $ solo agendo in partenza o in arrivo con un elemento del rispettivo gruppo di Galois, il che equivale a permutare il gruppo stesso per coniugio, isomorfismo che, per la proposizione \ref{coniugio}, è l'identità in coomologia. Scelte comunque due chiusure separabili, i relativi gruppi di coomologia sono canonicamente isomorfi. Possiamo quindi parlare senza remore della \leftquote coomologia di $ k $", della sua dimensione coomologica e dei suoi gruppi di coomologia. \\

% Hilbert 90 additivo
Fissiamo un'estensione di Galois $ L/k $ e cominciamo ad analizzare la coomologia del modulo più naturale del relativo gruppo $ \Gal{L/k} $: il campo $ L $. Il gruppo di Galois è composto dagli automorfismi del campo, per definizione, quindi è ben definita un'azione tanto sul gruppo additivo $ L $, quanto sul gruppo moltiplicativo $ L^\times $.

\begin{theorem} \label{Hadd}
	Sia $ L / k $ un'estensione di Galois. Il gruppo additivo di $ L $ è coomologicamente banale.
\end{theorem}
\begin{proof}
	Nel caso di estensioni finite, la tesi segue dal Teorema della Base Normale: questo afferma che $ L $ è un $ k[G] $ modulo libero. Per estensioni infinite è sufficiente passare al limite sulle sottoestensioni finite (tramite la solita proposizione \ref{limite}).
\end{proof}

% Hilbert 90 moltiplicativo
Non potevamo chiedere niente di meglio. La coomologia del gruppo moltiplicativo non è altrettanto semplice: lo studio di quest'ultima è il cuore della Teoria. La banalità del primo di gruppo di coomologia è già parecchio interessante, si tratta infatti di una riscrittura del Teorema di Indipendenza dei Caratteri di Artin nel nostro linguaggio:

\begin{theorem}[Hilbert 90]\label{H90}
	Sia $ L/k $ un'estensione di Galois. Si ha
	\[ \HH^1(\Gal{ L / k}, \, L^\times) = 0. \]
\end{theorem}
\begin{proof}
	Assumiamo, per il momento, che $ L/k $ sia finita. Mostreremo che tutti i cocicli sono cobordi: prendiamo un cociclo $ a \colon G \to L^\times $. Gli elementi di $ G $ sono linearmente indipendenti su $ L $, quindi troviamo $ x \in L^\times $ per cui
	\[ y = \sum_{g \in G} a_g \, g(x) \neq 0. \]
	Ma, traslando questo elemento di $ h $ e ricordando che $ a_{hg} = a_h \, ha_g $ per definizione, otteniamo $$  hy = \sum_{g \in G} ha_g  \, hg(x)  = a_h^{-1} \sum_{hg \in G} a_{hg} \,  hg(x) = a_h^{-1}y  $$ per ogni $ h \in G $, da cui la tesi $ a_h = (h-1)y^{-1} $.
	Ancora una volta, passando al limite (con \ref{limite}) otteniamo la tesi anche per le estensioni infinite.
\end{proof}

% Gruppo di Brauer
Il grado $ i = 2 $ si fa ancora più interessante, tanto da meritarsi un nome proprio.
\begin{definition}[Gruppo di Brauer]
	Il gruppo di Brauer di $ k $ è $$  \Br k = \HH^2(k, \, \bar{k}^\times).  $$
\end{definition}

% mi serve dire qualcosa sul gruppo di Brauer? Tipo perché ci piace? Sì dai!
Il gruppo di Brauer compare in diverse aree della matematica ed è di particolare importanza in algebra non commutativa, dove si scopre classificare le algebra centrali semplici sul campo in oggetto. Non entreremo nel dettaglio di questa corrispondenza, invitando però il lettore esperto a interpretare i risultati dei capitoli seguenti anche in quest'ottica: se ne ricava una buona intuizione dei risultati.\\

% Metto un paio di proposizioni per stare tranquilli; è importante dire che possiamo pensare al Brauer come contenete i sottobrauerini, che direi essere il lemma qua sotto
Nel seguito sarà comodo avere della notazione per indicare $ \HH^2(\Gal{L/k}, \, L^\times) $, che possiamo pensare come il gruppo di Brauer \leftquote relativo" all'estensione $ L/k $: lo chiameremo $ \Br(L/k) $.

\begin{lemma}\label{stronzetto}
	Per ogni estensione $ L/k $ finita di Galois, troviamo una successione esatta
	\[ 0 \to \Br(L/k) \to \Br k \to \Br L.  \]
\end{lemma}
\begin{proof}
	Possiamo riscrivere la tesi come
	\[ 0 \to \HH^2(\Gal{L/k}, \, L^\times) \to \HH^2(\Gal{\bar{k}, \, k}, \, \bar{k}^\times) \to \HH^2(\Gal{\bar{k}, \, L}, \, \bar{k}^\times), \]
	che è proprio una delle successioni di termini di grado basso della successione spettrale di \HS (corollario \ref{boo2}).
\end{proof}

Grazie a questo risultato, ci è concesso pensare al gruppo di Brauer di un campo come l'unione dei gruppi $ \HH^2(G, \, L^\times) $, che siamo interessati a calcolare, su tutte le estensioni di Galois finite. Enunciamo ora un'osservazione molto comoda per maneggiare il gruppo di Brauer di un campo, che quindi useremo spesso in futuro.

\begin{proposition} \label{torsione del brauer}
	Il sottogruppo di $ n $-torsione $ (\Br k) [\,n\,] $ coincide con $ \HH^2(k, \mu_n) $.
\end{proposition}
\begin{proof}
	La successione esatta lunga associata a 
	\[\begin{tikzcd}[column sep = small]
	0 \rar
	& \mu_n \rar
	& \bar{k}^\times \rar["\cdot n"]
	& \bar{k}^\times \rar
	& 0,
	\end{tikzcd}  \]
	grazie a Hilbert 90, contiene il segmento
	\[\begin{tikzcd}[column sep = small]
	0 \rar
	& \HH^2(G, \, \mu_n) \rar
	& \HH^2(G, \, \bar{\K}^\times) \rar["\cdot n"]
	& \HH^2(G, \, \bar{\K}^\times).
	\end{tikzcd} \qedhere \]
\end{proof}

% Verso il profinito
\begin{Profinite}
	È incredibilmente rassicurante osservare che, nonostante sia indispensabile aver definito la coomologia per gruppi profiniti, tutto si riconduca senza problemi alla coomologia di gruppi finiti. Tiriamo un sospiro di sollievo e procediamo con serenità.
\end{Profinite}

\section{Teoria di Kummer}
Questa sezione è dedicata a mostrare l'efficacia del linguaggio coomologico sviluppato fin'ora nel trattare alcuni problemi classici in teoria dei campi.
Sia $ \Gamma_k = \Gal{\bar{k}/ k} $ il gruppo di Galois assoluto di $ k $. Fissiamo un gruppo abeliano $ A $, finito e munito della topologia discreta; scegliendo un omomorfismo continuo
\[ \varphi \colon \Gamma_k \to A, \]
produciamo come nucleo un sottogruppo chiuso e normale. Per il teorema di corrispondenza di Galois, vi è associata un'estensione $ L $ di gruppo
\[ \Gal{L/k} = \frac{\Gal{\bar{k}/ k}}{\ker \varphi} \hookrightarrow A. \]
Al fine di classificare le estensioni di gruppo di Galois $ A $ abeliano è pertanto sufficiente trovare tutti gli omomorfismi $ \Hom_{\textsf{Gr}}(\Gamma_k, \, A) $, in realtà a meno della relazione che identifica due omomorfismi con lo stesso nucleo. Poiché $ G $ non agisce su $ A $, siamo in grado di trasportare il problema nel linguaggio della coomologia: siamo interessati a calcolare $ \HH^1(\Gamma_k, \, A) = \Hom_{\textsf{Gr}}(\Gamma_k, \, A) $. Possiamo quindi sfoggiare il nostro arsenale per produrre un piacevole risultato.

\begin{theorem}[di Kummer]
	Sia $ \K $ un campo e $ p $ un numero primo. Supponiamo che $ \K $ contenga le radici $ p $-esime dell'unità $ \mu_p $. In questo caso, c'è una corrispondenza tra le estensioni cicliche di gruppo $ \Z/p\Z $ (o banale) di $ K $ e $ \K^\times / {\K^\times}^p $.
\end{theorem}
\begin{proof}
	Calcoleremo $ \HH^1(\Gamma_k, \, \Z/p\Z) $. Prendere la $ p $-esima potenza $ x \mapsto x^p $ è un'operazione suriettiva nella chiusura algebrica, da cui l'esattezza di
	\[\begin{tikzcd}[column sep = small]
	0 \rar
	& \mu_p \rar
	& \bar{\K}^\times \rar["\cdot p"]
	& \bar{\K}^\times \rar
	& 0.
	\end{tikzcd}  \]
	La successione esatta lunga associata comincia, applicando Hilbert 90, con
	\[\begin{tikzcd}[column sep = small]
	0 \rar
	& \mu_p \rar
	& \K^\times \rar["\cdot p"]
	& \K^\times \rar
	& \HH^1(\Gamma_k, \, \mu_p) \rar
	& 0;
	\end{tikzcd}  \]
	che è proprio quello che cercavamo: poiché le radici $ p $-esime dell'unità vivono in $ k $ per ipotesi, $ \Gamma_k $ vi agisce banalmente e pertanto
	\[ \HH^1(\Gamma_k, \, \Z/p\Z) = \HH^1(\Gamma_k, \, \mu_p) =\K^\times / {\K^\times}^p . \qedhere \]
\end{proof}

Il Teorema di Kummer si colloca appena all'inizio di una teoria più ampia, piuttosto che esserne il culmine. La dimostrazione mostra la potenza e la comodità del linguaggio coomologico per affrontate lo studio delle estensioni di campi: in poche righe siamo riusciti a smontare un teorema per nulla banale, avendo costruito in precedenza tutta la rete di mappe e isomorfismi che questa dimostrazione avrebbe altrimenti richiesto di esplicitare. A titolo informativo presentiamo il seguente risultato, che permette sia di chiarire quanto esplicita si possa rendere la corrispondenza di Kummer sia, al contempo, di mostrare il tipo di risultati che si può sperare di ottenere procedendo in questa direzione.

\begin{proposition}
	Sia $K/\mathbb{Q}$ un'estensione ciclica di grado 3. Allora $ K $ è il campo di spezzamento di un polinomio della forma $$ x^3-3x-\frac{(2(a^2-3b^2))}{(a^2+3b^2)} $$ per certi $ a,\, b $ razionali. Viceversa, ogni tale polinomio o è riducibile o ha gruppo di Galois ciclico.
\end{proposition} 

È possibile utilizzare questo arsenale anche in modo più ricreativo.

\begin{theorem}
	Le terne pitagoriche primitive sono della forma $ (a^2-b^2,\, 2ab,\, a^2+b^2) $.
\end{theorem}

\begin{proof}
	Le terne pitagoriche primitive sono in bigezione con i punti di $ \Q(i) $ sulla circonferenza unitaria, ovvero di norma 1. Per la definizione del gruppo di Tate di grado $ i = -1 $, la ciclicità di $ G = \Gal{\Q(i)/\Q} $ e Hilbert 90, abbiamo che
	\[  \left(\ker N\right) / I_G \Q(i)^\times = \Hhm(G, \, \Q(i)^\times) = \HH^1(G, \, \Q(i)^\times) = 0, \]
	ovvero $ \ker N = I_G \Q(i)^\times $: ogni elemento di norma unitaria si scrive nella forma
	\[ (1-g) (a+ib) = \frac{a+ib}{a-ib} = \frac{a^2 -b^2}{a^2 + b^2} + i \, \frac{2ab}{a^2 + b^2}. \qedhere \]
\end{proof}

\section{Struttura di un Campo Locale}
% struttura
Ricordiamo brevemente tutto quello che è necessario sapere sui campi locali per poter procedere. Chiamato $ \K $ il nostro campo locale, possiamo definire l'anello degli interi su $ \K $ come gli elementi di valutazione non negativa
\[ \mathcal{O}_{\K} = \{ x \in \K \mid v(x)\geq 0 \}. \]
Questo è naturalmente un anello a valutazione discreta, pertanto locale: chiamiamo $ \pi $ un generatore dell'ideale primo e $ \kappa = \mathcal{O}_{\K}/(\pi) $ il campo residuo. Le unità di quest'anello sono il nucleo dell'omomorfismo di valutazione
\begin{equation}\label{local1}
	\begin{tikzcd}[column sep = small]
	1 \rar
	& \mathcal{O}_{\K}^\times \rar
	& \K^\times \rar["v"]
	& \Z \rar
	& 0.
	\end{tikzcd}
\end{equation}
La valutazione fornisce una piacevole filtrazione delle unità  principali, ovvero quelle nella forma $ x = \pi^k +1 $, permettendoci per esempio di distinguerle in classi  $$  U_{\K}^i :=\{x \in \mathcal{O}_{\K}^\times \mid v(x-1) \geq i \}.  $$ Si può mostrare che l'omomorfismo $ U_{\K}^i \to \kappa $ che manda $ x \mapsto (x-1)/p^n $ è suriettivo e genera della successioni esatte corte
\begin{equation}\label{local2}
\begin{tikzcd}[column sep = small]
1 \rar
& U_K^{i+1} \rar
& U_{K}^i \rar
& \kappa \rar
& 0.
\end{tikzcd}
\end{equation}
Abbiamo così ottenuto una filtrazione $ \mathcal{O}_{\K}^\times \supseteq U_{\K}^1 \supseteq U_{\K}^2 \supseteq \dots $ 
i cui quozienti $ U_{\K}^i / U_{\K}^{i+1} $ sono tutti isomorfi a $ \kappa $.
Infine troviamo, a collegare la prima successione \eqref{local1} con quelle sottostanti, 
\begin{equation}\label{local3}
	\begin{tikzcd}[column sep = small]
	1 \rar
	& U_K^{1} \rar
	& \mathcal{O}_{\K}^\times \rar
	& \kappa^\times \rar
	&1,
	\end{tikzcd}
\end{equation}
che, per il Lemma di Hensel, spezza.\\

Procediamo ora nell'enunciare qualche risultato che, sfruttando la relativa semplicità di questa struttura, ci fornisce qualche informazione aggiuntiva sulla coomologia delle estensioni di questi campi locali. Cominciamo dalle estensioni più facili da controllare: quelle non ramificate.

% coomologia dei gruppi delle unità
\begin{theorem}
	Sia $ L/\K $ un'estensione di Galois, finita e non ramificata, di gruppo $ G $. I gruppi $ \mathcal{O}_{\K}^\times $ e $ U_L^1 $ sono coomologicamente banali.
\end{theorem}
\begin{proof}
	Sia $ \lambda $ il campo residuo di $ L $. Aver assunto l'estensione non ramificata ci concede la comodità di poter identificare $ G = \Gal{L/\K} = \Gal{\lambda / \kappa} $. Poiché il gruppo additivo di $ \lambda $ è coomologicamente banale (teorema \ref{Hadd}), dalla successione esatta
	\[\begin{tikzcd}[column sep = small]
	1 \rar
	& U_L^{i+1} \rar
	& U_{L}^i \rar
	& \lambda \rar
	& 0
	\end{tikzcd}\]
	la stessa proprietà segue per tutti gli $ U^i_{L} / U^{i+1}_{L} $.
	Poiché la coomologia di un gruppo finito commuta con il limite proiettivo nel secondo argomento se tutti i moduli in questione sono finiti \cite[proposizione 1.12]{Harari}, segue per induzione che $ U^1_{L} / U^{i}_{L} $, e passando al limite anche tutto $ U^1_L = \varprojlim U^1_{L} / U^{i}_{L} $, è coomologicamente banale. \\
	
	Per estendere la proprietà a tutto $ \mathcal{O}_{L}^\times $ riscriviamo la successione esatta \eqref{local3}
	\begin{equation*}
	\begin{tikzcd}[column sep = small]
	1 \rar
	& U_L^{1} \rar
	& \mathcal{O}_{L}^\times \rar
	& \lambda^\times \rar
	&1
	\end{tikzcd}
	\end{equation*}
	e osserviamo che non solo $  U^1_L $ è coomologicamente banale, ma anche $ \lambda^\times $: infatti il gruppo $ G $ è ciclico e in grado $ i = 1 $ è banale per Hilbert 90, in grado $ i = 2 $ è banale perché $ \Br \lambda $ è nullo, perché $ \hat{\Z} $ ha dimensione coomologica 1. La tesi segue prendendo la successione esatta lunga associata.
\end{proof}

% Assioma del corpo di classe
Per estensioni ramificate la situazione non è così semplice, anche rimanendo interessati alle sole estensioni cicliche; tant'è che abbiamo bisogno di premettere un lemma tecnico alla generalizzazione del risultato appena ottenuto.

\begin{lemma}[del sottomodulo banale]
	Sia $ L/\K $ un'estensione finita di Galois di gruppo $ G $. Esiste un sottomodulo $ V \subseteq U_L^1 $ d'indice finito e coomologicamente banale.
\end{lemma}

\begin{proof}
	Per il teorema della Base Normale esiste un elemento $ \alpha \in L $ per cui $ \{ g\alpha \mid g \in G \} $ sia una base di $ L $ come spazio vettoriale su $ K $. Poiché l'estensione è finita, troviamo un elemento $ a \in K^\times $ di valutazione abbastanza grande da far cadere tutti i $ a (g\alpha) $ negli interi $ \mathcal{O}_L $: chiamiamo $ M $ l'$ \mathcal{O}_K $-sottomodulo di $ \mathcal{O}_L $ generato da questi elementi; $ M $ è isomorfo a $ \mathcal{O}_K[G] $ per costruzione.  
	Per di più, $ M $ è un sottogruppo aperto di $ \mathcal{O}_L $ (che, equipaggiato con la topologia indotta da $ L $, è un gruppo additivo profinito): dette $ p_i $ le proiezioni sulle coordinate, otteniamo infatti $ M $ come intersezione di aperti della forma $ \{ x \in \mathcal{O}_L \mid v_K p_i (x) \geq m_i\} $, per opportuni interi $ m_i $. Ne segue che $ M $ è un sottogruppo di indice finito e dunque che esiste un intero $ m $ per cui
	\[  \pi^m\mathcal{O}_L\subseteq M \subseteq \mathcal{O}_L. \]
	Traslando opportunamente $ M $, costruiamo una serie di sottomoduli di $ U_L^1 $ 
	\[ V_i = 1 + \pi^{m+i}M, \]
	con la piacevole proprietà di filtrare $ V_1 $, presentando quozienti coomologicamente banali: abbiamo infatti un isomorfismo $ V_i/V_{i+1} \to M/\pi M $ dato da $$  v_i \mapsto \pi^{-m-i}(v_i-1) \; +\pi M,  $$ che possiamo riassumere in diverse successioni esatte corte
	\[ 0 \to V_{i+1} \to V_i \to \kappa [G] \to 0; \]
	da cui deduciamo, analogamente a quanto fatto nella proposizione precedente, che il sottomodulo $ V= V_1 $, di indice finito in $ U_L^1 $ per costruzione, è anch'esso coomologicamente banale ed è quindi il modulo desiderato.
\end{proof}

\begin{theorem}[Assioma del Campo di Classe] \label{assioma}
	Sia $ L/\K $ un'estensione di Galois, finita e di gruppo $ G = \Gal{L/\K} $ ciclico. Allora $ \HH^1(G, \, L^\times) = 0 $ e $ \Hhz(G, \, L^\times) $ ha cardinalità $ [L \, \colon \K] $.
\end{theorem}
\begin{proof}
	La prima asserzione segue da Hilbert 90. Applichiamo il lemma del sottomodulo banale e scriviamo la successione esatta 
	\[ 1 \to V \to \mathcal{O}_L^\times \to \mathcal{O}_L^\times/V \to 1; \]
	il primo modulo è coomologicamente banale per costruzione, pertanto il suo quoziente di Herbrand è $ h(V) = 1 $, e anche l'ultimo modulo ha $ h = 1 $: infatti è finito, perché $$  [\mathcal{O}_L^\times \, \colon V] = [\mathcal{O}_L^\times \,\colon U_L^1]\cdot[U_L^1 \,\colon V],  $$
	dove il secondo indice è finito per costruzione e il primo perché il quoziente di quei due gruppi è $ \lambda^\times $ (si guardi la successione esatta $ \eqref{local3} $). Segue dalle proprietà del quoziente di Herbrand (prima \ref{Herb2} e poi $ \ref{Herb1} $) che $ h(\mathcal{O}_L^\times) = 1 $. Passiamo ora alla successione esatta $ \eqref{local1} $
	\[ \begin{tikzcd}[column sep = small]
	1 \rar
	& \mathcal{O}_{L}^\times \rar
	& L^\times \rar["v"]
	& \Z \rar
	& 0,
	\end{tikzcd} \]
	della quale conosciamo il quoziente di Herbrand del primo e dell'ultimo gruppo: sappiamo infatti che
	$$  \HH^1(G, \, \Z) = \Hom_\mathsf{Gr}(G, \, \Z) = \Hom_\Z(G^\texttt{ab}, \, \Z) = 0  $$ perché $ G $ è finito e agisce banalmente su $ \Z $ e che
	$$  \Hhz(G, \, \Z) = \frac{\Z^G}{ N\Z} = \frac{\Z}{ |G|\, \Z}  $$ per definizione. Infine, per le proprietà del quoziente di Herbrand e Hilbert 90
	\[ h(L^\times) = h(\Z) \qquad\Longrightarrow\qquad |\Hhz(\K, L^\times)\, | = |G| = [L \, \colon \K]. \qedhere \]
\end{proof}

Questi due risultati, fondamentalmente, traducono nel linguaggio coomologico le proprietà aritmetiche di un campo locale. Sembra superfluo aggiungere che non è posibile percorrere una strada simile per ottenere risultati analoghi nel caso di campi globali. Nel prossimo paragrafo sfrutteremo quanto appena scoperto per calcolare la coomologia in grado 2.

\section{Calcolo del gruppo di Brauer}
% Calcolo del BRKnr e mappa inv
Ogni campo locale ha esattamente un'estensione non ramificata per ogni grado $ \knrr $, il cui gruppo di Galois coincide con la corrispondente estensione del campo residuo dello stesso grado: $ \Gal{\knrr / \K} = \Gal{\kappa^n / \kappa} = \Z / n\Z $. Ne segue che la massima estensione non ramificata $ \knr = \varinjlim \knrr $ ha gruppo di Galois
\[ G = \Gal{\knr / \K} = \Gal{\bar{\kappa} / \kappa} = \hat{\Z}. \]
Siamo interessati a calcolare $ \HH^2(\Gamma, \, \knr^\times) $. Dalla successione esatta corta
\[ \begin{tikzcd}[column sep = small]
1 \rar
& \mathcal{O}_{\knr}^\times \rar
& \knr^\times \rar["v"]
& \Z \rar
& 0,
\end{tikzcd} \]
ricordando che $ \mathcal{O}_{\knr}^\times $ è coomologicamente banale (o meglio, osservando che tutti gli oggetti del sistema induttivo di cui dobbiamo prendere il limite per calcolarlo lo sono), otteniamo un isomorfismo
\[ v^* \colon \HH^2(G, \, \knr^\times) \to \HH^2(G, \, \Z). \]
Un po' a sorpresa, ripetiamo il ragionamento sulla successione esatta
\[ 0 \to \Z \to \Q \to \Q/\Z \to 0, \]
in cui troviamo $ \Q $ iniettivo, per ottenere un secondo isomorfismo
\[ \delta \colon \HH^2(G, \, \Z) \to \HH^1(G, \, \Q/\Z). \]
Poiché $ \Q/\Z $ è munito dell'azione banale, siamo in realtà interessati a calcolare $ \Hom(G, \, \Q/\Z) $, morfismi che ovviamente coincidono con le possibili immagini del generatore topologico $ 1 $ di $ G = \hat{\Z} $: abbiamo un'identificazione formale $ \gamma \colon \HH^1(G, \, \Q/\Z) \to \Q/\Z.  $

\begin{definition}
	Chiamiamo $ \inv_\K $ l'isomorfismo che otteniamo componendo le tre mappe di sopra
	\[ \begin{tikzcd}[column sep = small]
	\inv_\K \colon \HH^2(G, \, \knr^\times)  \rar["v^*"]
	& \HH^2(G, \, \Z) \rar["\delta"]
	& \HH^1(G, \, \Q/\Z) \rar["\gamma"]
	& \Q/\Z.
	\end{tikzcd} \]
\end{definition}

Avendo calcolato questo gruppo, abbiamo svolto in scioltezza metà del lavoro. Le buone proprietà di questo isomorfismo saranno fondamentali per affrontare il conto rimanente.

\begin{proposition}\label{inv}
	Sia $ L/\K $ un'estensione finita di grado $ n = [L\, \colon \K] $. Il seguente diagramma commuta
	\[ \begin{tikzcd}
	\HH^2(G_\K, \, \knr^\times) \rar["\inv_\K"] \dar["\Res"]
	& \Q/\Z \dar["\cdot n"] \\
	\HH^2(G_L, \, \knr^\times) \rar["\inv_L"]
	& \Q/\Z.
	\end{tikzcd} \]
\end{proposition}
\begin{proof}
	Analizziamo come commuta la $ \Res $ con i singoli isomorfismi di cui è composto l'isomorfismo $ \inv $. Siano $ e $ l'indice di ramificazione e $ f $ l'indice d'inerzia dell'estensione. Quando c'è ramificazione la valutazione viene rinormalizzata in modo che $ (v_L)_{\mid K} = e \cdot v_\K $, dunque
	\[ v_L^* \Res = e \cdot v_\K^* \Res  = e \cdot  \Res v_\K^*, \]
	dove possiamo effettuare l'ultima commutazione perché $ \Res $ è un morfismo di complessi per costruzione. 
	Analogamente, la restrizione commuta anche con l'omomorfismo di collegamento $ \delta $. L'indice $ f $ coincide con il grado della relativa estensione del campo residuo, che dunque è proprio il generatore di $ G_L = f\hat{\Z} < \hat{\Z} = G_\K $; abbiamo così, ricordando che $ \gamma $ coincide con la valutazione nel generatore, che
	$$  \gamma(\Res h) = (\Res h)(1) = h(\Res(1)) = h(f) = f\cdot h(1) = f \cdot  \gamma(h).  $$
	Componendo i tre risultati, si ottiene la tesi.
	\[ \begin{tikzcd}
	\HH^2(G_\K, \, \knr^\times) \rar["v_\K^*"] \dar["\Res"]
	& \HH^2(G_\K, \, \Z) \dar["\cdot e\Res"] \rar["\delta"]
	& \HH^1(G_\K, \, \Q/\Z) \rar["\gamma"] \dar["\cdot e\Res"]
	& \Q/\Z \dar["\cdot ef"] \\
	\HH^2(G_L, \, \knr^\times) \rar["v_L^*"]
	& \HH^2(G_L, \, \Z) \rar["\delta"]
	& \HH^1(G_L, \, \Q/\Z) \rar["\gamma"]
	& \Q/\Z.
	\end{tikzcd} \]
\end{proof}

% BR(Knr/K) = Br(K)
Siamo ora pronti per dimostrare che il gruppo di Brauer relativo all'estensione non ramificata massimale è in realtà già tutto $ \Br \K $.

\begin{theorem}
	Il gruppo $ \Br K $ è isomorfo a $ \Br (\knr/K) = \Q/\Z $.
\end{theorem}

\begin{proof}
Dimostreremo che il gruppo di Brauer relativo ad ogni estensione di Galois di grado $ n $ coincide, come sottogruppo di $ \Br \K $, con il Brauer relativo all'unica estensione non ramificata di grado $ n $: $ \K_\mathtt{nr}^n $. È conveniente dividere la dimostrazione in due lemmi;
fissiamo un'estensione $ L/\K $ di Galois di grado $ n = [L\,\colon \K] $ e gruppo $ G $.

\begin{lemma}\label{Br1}
	La cardinalità del gruppo $ \HH^2(G,\, L^\times ) $ divide $ n $.
\end{lemma}

\begin{proof}
	Arriveremo alla tesi per approssimazione. % fuck yeah
	Il caso in cui $ G $ sia ciclico segue immediatamente dall'Assioma del Campo di Classe (\ref{assioma}). Affrontiamo ora il problema per $ G $ $ p $-gruppo: grazie alla formula delle classi sappiamo che il centro è non banale, questo è abeliano e ha pertanto un sottogruppo di ordine $ p $ che chiamiamo $ H $, normale in $ G $. A questo gruppo sarà associata un'estensione intermedia $ L^H $, anch'essa di Galois. Dalla successione spettrale di Hoschild-Serre, o meglio per quanto dice il corollario \ref{boo2}, ricaviamo la successione esatta
	\[\begin{tikzcd}
	0 \rar & \HH^2(G/H, \, \left({L^H}\right)^\times) \rar["\Inf"]
	& \HH^2(G, \, L^\times) \rar["\Res"]
	& \HH^2(H, \, L^\times),
	\end{tikzcd} \qquad  \]
	da cui deduciamo la tesi per induzione. Segue immediatamente il caso generale: scelto un Sylow per ogni primo $ p $, la restrizione produce degli omomorfismi iniettivi
	\[ \begin{tikzcd}
	0 \rar & T_p\HH^2(G, \, L^\times) \rar["\Res"] & \HH^2(G_p, L^\times),
	\end{tikzcd}  \]
	grazie ai quali deduciamo che la cardinalità del gruppo di destra divide $ |G_p| $, ovvero la massima potenza di $ p $ che compare nella fattorizzazione di $ n $.
\end{proof}

\begin{lemma}\label{br2}
	$ \HH^2(G,\, L^\times ) $ e $ \HH^2(\Gal{\knrr/\K},\, \knrr^{\times}) $ sono lo stesso sottogruppo di $ \Br \K $.
\end{lemma}

\begin{proof}
	La cardinalità di $ \HH^2(\Gal{\knrr/\K},\, \knrr^{\times}) $ è esattamente $ n $: per la ciclicità del gruppo di Galois è la stessa del gruppo di Tate in grado zero, che è quella voluta per l'Assioma del Corpo di Classe. È quindi sufficiente mostrare che $ \HH^2(\Gal{\knrr/\K},\, \knrr^{\times}) $ è un sottogruppo di $ \HH^2(G,\, L^\times ) $, grazie al risultato sulla cardinalità appena stabilito.
	Prendiamo un elemento $ x \in \HH^2(\Gal{\knrr/\K},\, \knrr^{\times}) < \Br \knr < \Br \K $. Questo vive in un gruppo di $ n $-torsione e pertanto viene mandato in $ 0 $ dalla restrizione
	\[ \begin{tikzcd}
	\Br \K \rar["\Res"]
	& \Br L \\
	\HH^2(G_\K, \, \knr^\times) \rar["\Res"] \dar["\inv_\K"] \arrow[hook]{u}
	& \HH^2(G_L, \, \knr^\times) \dar["\inv_L"] \arrow[hook]{u} \\
	\Q/\Z \rar["\cdot n"]
	& \Q/\Z,
	\end{tikzcd} \]
	il cui nucleo è proprio $ \HH^2(G, \, L^\times) $ per la solita successione esatta dei Brauer relativi (\ref{stronzetto}):
	\[ 0 \to \HH^2(G, \, L^\times) \to \Br \K \to \Br L. \qedhere \]
\end{proof}

Torniamo ora alla dimostrazione del teorema. Abbiamo già osservato che $ \Br \K = \HH^2(\Gamma_K, \, \bar{\K}^\times) $ è unione dei Brauer relativi alle sottoestensioni di Galois finite, che per il secondo lemma coincidono con i soli Brauer relativi alle estensioni non ramificate:
\begin{align*}
	\Br \K &= \textstyle\varinjlim_{L/K} \HH^2(\Gal{L/K}, \, L^\times)\\
	&= \textstyle\varinjlim_{n} \HH^2(\Gal{\K_\mathtt{nr}^n/\K},\, \K_\mathtt{nr}^{n\times})\\
	&= \Br (\knr /\K)\\
	&= \Q/\Z. \qedhere
\end{align*}

\end{proof}

\section{Tutto il resto}
In questa sezione concluderemo lo studio della coomologia di $ \Gamma_K $, mostrando che questo ha dimensione coomologica 2: tutti i gruppi di coomologia rimanenti saranno dunque nulli. Quasi. Ad essere precisi, solo quelli con coefficienti in moduli di torsione.

\begin{lemma}
	Sia $ k $ un campo e $ p $ un numero primo. I seguenti sono equivalenti:
	\begin{enumerate}[label= \roman*.]
		\item La $ p $-dimensione coomologica $ \cd_p(k) $ è al più 1.
		\item Per ogni estensione finita separabile $ L/k $, la $ p $-torsione del $ \Br L $ è nulla.
	\end{enumerate}
\end{lemma}

La $ p $-torsione è, per definizione, il nucleo della moltiplicazione per $ p $ e non l'intera componente $ p $-primaria, si proceda con attenzione: denoteremo la $ p $-torsione con delle parentesi quadre $ (\Br K)[\, p \, ]. $


\begin{proof}
	Assumiamo $ i $. Per ogni $ L $, la $ p $-dimensione coomologica del sottogruppo $ \Gal{\bar{k}/L} $ è al più quella del gruppo $ \Gal{\bar{k}/k} $ (per \ref{cd2}), ne segue che
	\[ (\Br L) [\, p \,] = \HH^2(L, \, \mu_p) = 0. \] 
	Assumiamo dunque $ ii $. Sia $ G_p $ un $ p $-Sylow di $ \Gal{\bar{k}/k} $ e $ L $ il sottocampo corrispondente. $ L $ contiene le radici dell'unità $ \mu_p $: infatti l'indice $ [L(\mu_p)\,\colon L] $ deve dividere sia $ p-1 $, perché campo di spezzamento di $ x^p-1 $, che $ p $, per costruzione. Ne segue che
	\[ \HH^2(L, \, \Z/p\Z) = \HH^2(L, \, \mu_p) = (\Br L)[\, p\,], \]
	che è nullo per ipotesi; da cui, sfoderando tutti i teoremi sulla dimensione coomologica a nostra conoscenza, deduciamo che $ \cd_p(k) = \cd_p(L) = 1 $.
\end{proof}


\begin{theorem}\label{cdim2}
	La dimensione coomologica di un campo locale è 2.
\end{theorem}
\begin{proof}
	Mostreremo preliminarmente che il gruppo di Galois assoluto $ \Gal{\bar{K}/\knr} $ dell'estensione non ramificata massimale $ \knr $ ha dimensione coomologica $ 1 $. Per il lemma appena enunciato sarà sufficiente dimostrare che, per ogni primo $ p $ e ogni estensione finita e separabile $ L/\knr $, il gruppo di Brauer $ \Br L $ non ha componente $ p $-primaria: questo gruppo è limite diretto dei gruppi di Brauer $ \Br E $ delle estensioni intermedie finite $ K \subseteq E \subseteq L $, per il solito teorema di passaggio al limite (\ref{limite}); mostreremo pertanto che ogni elemento di $ p^\alpha $-torsione viene eventualmente ucciso dalle mappe del sistema induttivo (che ricordiamo essere restrizioni). Nel sistema induttivo si trova infatti, sopra $ E $, un'estensione $ F $ per cui l'indice $ [F \,\colon E] = p^\alpha $, che produce la mappa assassina che cercavamo:
	\[ \begin{tikzcd}
	\Br E \rar["\inv_E"] \dar["\Res"]
	& \Q/\Z \dar["\cdot p^\alpha"] \\
	\Br F \rar["\inv_F"]
	& \Q/\Z.
	\end{tikzcd} \]
	Riusciamo ad ottenere l'estensione $ F $ richiesta, per esempio, componendo $ E $ con un'estensione di $ K $ non ramificata del giusto grado.\\
	
	Possiamo ora concludere: l'estensione $ \knr /K $ non ramificata massimale ha gruppo di Galois $ \hat{\Z} $, che è un sottogruppo di $ \Gal{\bar{\K}/K} $, il cui quoziente abbiamo appena mostrato avere dimensione coomologica 1; per il lemma \ref{quozienti}, concludiamo che 
	$$  \cd(K) \leq \cd(\knr) + \cd(\hat{\Z}) = 1+1.  $$
	Osserviamo infine che, per un primo $ p $ eventualmente diverso dalla caratteristica di $ K $, il gruppo $ \HH^2(K, \, \mu_p) = (\Br K)[\, p\, ] $ non è nullo; otteniamo così la disuguaglianza opposta.
\end{proof}

% Sarebbe ragionevole avere un'intuzione di cosa signifa e cosa ce ne facciamo... in particolare ci stiamo per lanciare in un capitolo follemente tecnico.
Questo teorema conclude il calcolo della coomologia di un campo locale, ma abbiamo tutt'altro che concluso lo studio. Che cosa ce ne facciamo di aver calcolato questi misteriosi gruppi di Tate? Ne ricaviamo qualcosa di comprensibile a livello di estensioni di campi? Il prossimo capitolo sarà dedicato all'interpretazione, o meglio alla decifrazione, di quanto calcolato in questo.