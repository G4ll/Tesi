\chapter{Teoria del Campo di Classe}
In questo capitolo faremo fruttare tutto quello che è stato preparato in precedenza: cominceremo a parlare di estensioni di campi e a studiare la coomologia dei relativi gruppi di Galois. Inizieremo con qualche risultato di carattere generale riguardo le estensioni di Galois, dalle nozioni di base alla teoria di Kummer, per restringere poi la nostra attenzione ai campi locali. \\

% Discorso su perché i campi locali.
Per campo locale intendiamo un campo completo rispetto a una valutazione discreta di rango 1. Si può pensare, senza perdita di generalità, a estensioni dei $ p $-adici $ \Q_p $ o delle serie Laurent su campi finiti $ \F_p((t)) $. Studiamo questi campi per almeno due buone ragioni; in primis, perché presentano una struttura relativamente semplice: i relativi anelli degli interi sono locali (da cui il nome) e le unità sono di facile descrizione; in secondo luogo, perché sono un ottimo punto di partenza per dedurre risultati analoghi anche nel caso globale.

% Cos'è la Class Field Theory?


\section{Teoria di Galois}
Sia $ \K $ un campo. Fissiamo una chiusura separabile $ \bar{\K} $. D'ora in avanti denoteremo
\[ \HH^i(\K, \, A) := \HH^i(\Gal{\bar{\K} / \K}, \, A). \]
Osserviamo che, presa una diversa chiusura separabile $ \bar{\K}' $ e un sollevamento di $ \id_{\K} $ a un isomorfismo $ f\colon \bar{\K} \to \bar{\K}' $, questo induce un isomorfismo in coomologia
\[ f^* \colon \HH^i(\Gal{\bar{\K} / \K}, \, A) \to \HH^i(\Gal{\bar{\K}' / \K}, \, A) \]
che non dipende affatto dalla scelta di $ f $: possiamo cambiare l'immersione $ f $ solo agendo in partenza o in arrivo con un elemento del rispettivo gruppo di Galois, il che equivale a permutare il gruppo stesso per coniugio, isomorfismo che, per la proposizione \ref{coniugio}, è l'identità in coomologia. Scelte comunque due chiusure separabili, i relativi gruppi di coomologia sono canonicamente isomorfi.\\


% Hilbert 90 additivo
Fissiamo un'estensione di Galois $ L/\K $ e cominciamo ad analizzare la coomologia del modulo più naturale del relativo gruppo $ \Gal{L/\K} $: il campo $ L $. Il gruppo di Galois è composto dagli automorfismi del campo, per definizione, quindi è ben definita un'azione tanto sul gruppo additivo $ L $, quanto sul gruppo moltiplicativo $ L^\times $.

\begin{theorem} \label{Hadd}
	Sia $ L / \K $ un'estensione di Galois. Il gruppo additivo di $ L $ è coomologicamente banale.
\end{theorem}
\begin{proof}
	Nel caso di estensioni finite, la tesi segue dal Teorema della Base Normale: questo afferma che $ L $ è un $ K[G] $ modulo libero. Per estensioni infinite è sufficiente passare al limite sulle sottoestensioni finite (tramite la solita proposizione \ref{limite}).
\end{proof}

% Hilbert 90 moltiplicativo
Non potevamo chiedere niente di meglio. La coomologia del gruppo moltiplicativo non è altrettanto semplice: lo studio di quest'ultima è il cuore della Teoria. La banalità del primo di gruppo di coomologia è già parecchio interessante, si tratta infatti di una riscrittura del Teorema di Indipendenza dei Caratteri di Artin nel nostro linguaggio:

\begin{theorem}[Hilbert 90]\label{H90}
	Sia $ L/\K $ un'estensione di Galois. Si ha
	\[ \HH^1(\Gal{ L / \K}, \, L^\times) = 0. \]
\end{theorem}
\begin{proof}
	Assumiamo, per il momento, che $ L/\K $ sia finita. Mostreremo che tutti i cocicli sono cobordi: prendiamo un cociclo $ a \colon G \to L^\times $, gli elementi di $ G $ sono linearmente indipendenti su $ L $, quindi troviamo $ x \in L^\times $ per cui
	\[ y = \sum_{g \in G} a_g g(x) \neq 0. \]
	Ma $ hy = a_h^{-1}y $ per ogni $ h \in G $, da cui la tesi.
	Ancora una volta, passando al limite otteniamo la tesi anche per le estensioni infinite. \todo[bruttino]
\end{proof}

% Gruppo di Brauer
Il grado $ i = 2 $ si fa ancora più interessante, tanto da meritarsi un nome proprio.
\begin{definition}[Gruppo di Brauer]
	Il gruppo di Brauer di $ \K $ è $$  \Br \K = \HH^2(\K, \, \bar{\K}^\times).  $$
\end{definition}

% mi serve dire qualcosa sul gruppo di Brauer? Tipo perché ci piace? Sì dai!
Il gruppo di Brauer compare in diverse aree della matematica ed è di particolare importanza in algebra non commutativa, dove si scopre classificare le algebra centrali semplici sul campo in oggetto. Non entreremo nel dettaglio di questa corrispondenza, invitando però il lettore che ne avesse sentito parlare a interpretare i risultati dei capitoli seguenti anche in quest'ottica: se ne ricava più che altro una buona intuzione dei risultati.

% Metto un paio di proposizioni per stare tranquilli; è importante dire che possiamo pensare al Brauer come contenete i sottobrauerini, che direi essere il lemma qua sotto
Nel seguito sarà comoda avere della notazione per indicare $ \HH^2(\Gal{L/k}, \, L^\times) $, che possiamo pensare come il gruppo di Brauer relativo all'estensione $ L/k $: lo chiameremo $ \Br(L/\K) $.

\begin{lemma}\label{stronzetto}
	Se $ L/k $ è un'estensione finita di Galois
	\[ 0 \to \Br(L/k) \to \Br k \to \Br L.  \]
\end{lemma}
\begin{proof}
	Possiamo riscrivere la tesi come
	\[ 0 \to \HH^2(\Gal{L/k}, \, L^\times) \to \HH^2(\Gal{\bar{k}, \, k}, \, \bar{k}^\times) \to \HH^2(\Gal{\bar{k}, \, L}, \, \bar{k}^\times), \]
	la cui esattezza segue da una delle successioni di termini di grado basso della successione spettrale di Hoschild-Serre (corollario \ref{boo2}).
\end{proof}

Grazie a questo risultato, ci è concesso pensare al gruppo di Brauer di un campo come l'unione dei misteriosi gruppi $ \HH^2(G, \, L^\times) $ di tutte le estensioni di Galois finite.


\begin{lemma} \label{torsione del brauer}
	$ (\Br k) [n] = \HH^2(k, \mu_n) $.
\end{lemma}
\begin{proof}
	Segue immediatamente osservando che la successione esatta lunga associata a 
	\[\begin{tikzcd}[column sep = small]
	0 \rar
	& \mu_p \rar
	& \bar{k}^\times \rar["\cdot n"]
	& \bar{k}^\times \rar
	& 0,
	\end{tikzcd}  \]
	grazie a Hilbert 90, contiene il segmento
	\[\begin{tikzcd}[column sep = small]
	0 \rar
	& \HH^2(G, \, \mu_n) \rar
	& \HH^2(G, \, \bar{\K}^\times) \rar["\cdot n"]
	& \HH^2(G, \, \bar{\K}^\times).
	\end{tikzcd}  \]
\end{proof}

% Verso il profinito
\begin{profinite}
	È incredibilmente rassicurante osservare che, nonostante sia indispensabile aver definito la coomologia per gruppi profiniti, tutto si riconduca senza problemi alla coomologia di gruppi finiti. Tiriamo un sospirio di sollievo e procediamo con serenità
\end{profinite}

\section{Teoria di Kummer}


Sia $ G = \Gal{\bar{\K}/ \K} $ il gruppo di Galois assoluto di $ \K $. Fissiamo un gruppo abeliano $ A $, finito e munito della topologia discreta; scegliendo un omomorfismo continuo
\[ \varphi \colon \Gal{\bar{\K}/ \K} \to A, \]
produciamo come nucleo $ \ker \varphi $ un sottogruppo chiuso e normale del gruppo di Galois assoluto. Per il teorema di corrispondenza di Galois, otteniamo un'estensione $ L $ di gruppo
\[ \Gal{L/K} = \frac{\Gal{\bar{\K}/ \K}}{\ker \varphi} \hookrightarrow A. \]
Al fine di classificare le estensioni di gruppo di Galois $ A $ abeliano è pertanto sufficiente trovare tutti gli omomorfimi $ \Hom_{\textsf{Gr}}(G, \, A) $, in realtà a meno della relazione che identifica due omomorfimi con lo stesso nucleo. Poiché $ G $ non agisce su $ A $, siamo in grado di trasportare il problema nel linguaggio della coomologia: siamo interessati a calcolare $ \HH^1(G, \, A) = \Hom_{\textsf{Gr}}(G, \, A) $. Possiamo quindi sfoggiare il notro arsenale per produrre un risultato piacevole.

\begin{theorem}[di Kummer]
	Sia $ \K $ un campo e $ p $ un numero primo. Supponiamo che $ \K $ contenga le radici $ p $-esime dell'unità $ \mu_p $. In questo caso, c'è una corrispondenza tra le estensioni cicliche di gruppo $ \Z/p\Z $ (o banale) e $ \K^\times / {\K^\times}^p $.
\end{theorem}
\begin{proof}
	Calcoleremo $ \HH^1(G, \, \Z/p\Z) $, dove $ G $ è come al solito il gruppo di Galois assoluto del campo $ \K $. Prendere la $ p $-esima potenza $ x \mapsto x^p $ è un'operazione suriettiva nella chiusura algebrica, da cui l'esattezza di
	\[\begin{tikzcd}[column sep = small]
	0 \rar
	& \mu_p \rar
	& \bar{\K}^\times \rar["\cdot p"]
	& \bar{\K}^\times \rar
	& 0.
	\end{tikzcd}  \]
	La successine esatta lunga associata comincia, applicando Hilbert 90, con
	\[\begin{tikzcd}[column sep = small]
	0 \rar
	& \mu_p \rar
	& \K^\times \rar["\cdot p"]
	& \K^\times \rar
	& \HH^1(G, \, \mu_p) \rar
	& 0;
	\end{tikzcd}  \]
	che è proprio quello che cercavamo: poiché le radici $ p $-esime dell'unità vivono in $ \K $ per ipotesi, $ G $ vi agisce banalmente e pertanto
	\[ \HH^1(G, \, \Z/p\Z) = \HH^1(G, \, \mu_p) =\K^\times / {\K^\times}^p . \]
\end{proof}

\section{Il gruppo di Brauer di un Campe Locale}
% struttura
Ricordiamo brevemente tutto quello che è necessario sapere sui campi locali per poter procedere. Chiamato il nostro campo locale $ \K $, possiamo definire l'anello degli interi di $ \K $ come gli elementi di valutazione non negativa
\[ \mathcal{O}_{\K} = \{ x \in \K \mid v(x)\geq 0 \}. \]
Questo è naturalmente un anello a valutazione discreta, pertanto locale: chiamiamo $ \pi $ il generatore dell'ideale primo e $ \kappa = \mathcal{O}_{\K}/(\pi) $ il campo residuo. Le unità di quest'anello sono il nucleo dell'omomorfismo di valutazione
\begin{equation}\label{local1}
	\begin{tikzcd}[column sep = small]
	1 \rar
	& \mathcal{O}_{\K}^\times \rar
	& \K^\times \rar["v"]
	& \Z \rar
	& 0,
	\end{tikzcd}
\end{equation}
per le quali il grado fornisce una piacevole filtrazione: le unità devono essere nella forma $ x = \pi^k +1 $, il che ci permette di ditinguerle in classi  $$  U_{\K}^i :=\{x \in \mathcal{O}_{\K}^\times \mid v(x-1) \geq i \}.  $$ Si può mostrare che l'omomorfismo $ U_{\K}^i \to \kappa $ che manda $ x \mapsto (x-1)/p^n $ è suriettivo e genera della successioni esatte corte
\begin{equation}\label{local2}
\begin{tikzcd}[column sep = small]
1 \rar
& U_k^{i+1} \rar
& U_{\K}^i \rar
& \kappa \rar
& 0.
\end{tikzcd}
\end{equation}
Abbiamo così ottenuto una filtrazione $ \mathcal{O}_{\K}^\times \supseteq U_{\K}^1 \supseteq U_{\K}^2 \supseteq \dots $ 
i cui quozienti $ U_{\K}^i / U_{\K}^{i+1} $ sono tutti isomorfi a $ \kappa $.
Infine, per il Lemma di Hensel, abbiamo che
\begin{equation}\label{local3}
	\begin{tikzcd}[column sep = small]
	1 \rar
	& U_k^{1} \rar
	& \mathcal{O}_{\K}^\times \rar
	& \kappa^\times \rar
	&1.
	\end{tikzcd}
\end{equation}

\todo[non sono completamente sicuro di quello che sto scrivendo]

Procediamo ora nell'enunciare qualche risultato che, sfruttando la relativa semplicità di questa struttura, ci fornisce qualche informazione aggiuntiva sulla coomologia delle estensioni di questi campi locali. Cominciamo dalle estensioni più facili da controllare: quelle non ramificate.

% coomologia dei gruppi delle unità
\begin{theorem}
	Sia $ L/\K $ un'estensione di Galois, finita e non ramificata, di gruppo $ G $. I gruppi $ \mathcal{O}_{\K}^\times $ e $ U_L^1 $ sono coomologicamente banali.
\end{theorem}
\begin{proof}
	Sia $ \lambda $ il campo residuo di $ L $. Aver assunto l'estensione non ramificata ci concede la comodità di poter identificare $ G = \Gal{L/\K} = \Gal{\lambda / \kappa} $. Poiché il gruppo additivo di $ \lambda $ è coomologicamente banale (teorema \ref{Hadd}), dalla successione esatta
	\[\begin{tikzcd}[column sep = small]
	1 \rar
	& U_L^{i+1} \rar
	& U_{L}^i \rar
	& \lambda \rar
	& 0
	\end{tikzcd}\]
	la stessa proprietà segue per tutti gli $ U^i_{L} / U^{i+1}_{L} $.
	\todo[aggiungere nel capitolo 1 l'osservazione seguente]
	Poiché la coomologia di un gruppo finito commuta con il limite proiettivo nel secondo argomento, segue per induzione che $ U^1_{L} / U^{i}_{L} $, e passando al limite che tutto $ U^1_L = \varprojlim U^1_{L} / U^{i}_{L} $, è coomologicamente banale. \\
	
	Per estendere la proprietà a tutto $ \mathcal{O}_{L}^\times $ riscriviamo la successione esatta \ref{local3}
	\begin{equation*}
	\begin{tikzcd}[column sep = small]
	1 \rar
	& U_L^{1} \rar
	& \mathcal{O}_{L}^\times \rar
	& \lambda^\times \rar
	&1
	\end{tikzcd}
	\end{equation*}
	e osserviamo che non solo $  U^1_L $ è coomologicamente banale, ma anche $ \lambda^\times $: infatti in grado $ i = 1 $ è banale per Hilbert 90, in grado $ i = 2 $ banale perché sottogruppo di $ \Br \kappa $ (volendo per \ref{stronzetto}), che è nullo perché $ \hat{\Z} $ ha dimensione coomologica 1, e il gruppo $ G $ è ciclico. Segue la tesi prendendo la successione esatta lunga associata.
\end{proof}

% Calcolo del BRKnr e mappa inv
Ogni campo locale ha esattamente un'estensione non ramificata per ogni grado $ \knr $, il cui gruppo di Galois coincide con la corrispondente estensione del campo residuo dello stesso grado: $ \Gal{\K^n_\texttt{nr} / \K} = \Gal{\kappa^n / \kappa} = \Z / n\Z $. Ne segue che la massima estensione non ramificata $ \knr = \varinjlim \K^n_\texttt{nr} $ ha gruppo di Galois
\[ G = \Gal{\knr / \K} = \Gal{\bar{\kappa} / \kappa} = \hat{\Z}. \]
Siamo interessati a calcolare $ \HH^2(\Gamma, \, \knr^\times) $. Dalla successione esatta corta
\[ \begin{tikzcd}[column sep = small]
1 \rar
& \mathcal{O}_{\knr}^\times \rar
& \knr^\times \rar["v"]
& \Z \rar
& 0,
\end{tikzcd} \]
ricordando che $ \mathcal{O}_{\knr}^\times $ è coomologicamente banale, otteniamo un isomorfismo
\[ v^* \colon \HH^2(G, \, \knr^\times) \to \HH^2(G, \, \Z). \]
Un po' a sorpresa, ripetiamo il ragionamento sulla successione esatta
\[ 0 \to \Z \to \Q \to \Q/\Z \to 0, \]
in cui troviamo $ \Q $ iniettivo, per ottenere un secondo isomorfismo
\[ \delta \colon \HH^2(G, \, \Z) \to \HH^1(G, \, \Q/\Z). \]
Poiché $ \Q/\Z $ è munito dell'azione banale, siamo in realtà interessati a calcolare $ \Hom(G, \, \Q/\Z) $, che ovviamente coincidono con le possibili immagini del generatore topologico $ 1 $ di $ G = \hat{\Z} $: abbiamo un'identificazione formale $ \gamma \colon \HH^1(G, \, \Q/\Z) \to \Q/\Z.  $

\begin{definition}
	Chiamiamo $ \inv_\K $ l'isomorfismo che otteniamo componendo le tre mappe di sopra
	\[ \begin{tikzcd}[column sep = small]
	\inv_\K \colon \HH^2(G, \, \knr^\times)  \rar["v^*"]
	& \HH^2(G, \, \Z) \rar["\delta"]
	& \HH^1(G, \, \Q/\Z) \rar["\gamma"]
	& \Q/\Z.
	\end{tikzcd} \]
\end{definition}

\begin{proposition}
	Sia $ L/\K $ un'estensione finita di grado $ n = [L\, \colon \K] $. Il seguente diagramma commuta
	\[ \begin{tikzcd}
	\HH^2(G_\K, \, \knr^\times) \rar["\inv_\K"] \dar["\Res"]
	& \Q/\Z \dar["\cdot n"] \\
	\HH^2(G_L, \, \knr^\times) \rar["\inv_L"]
	& \Q/\Z.
	\end{tikzcd} \]
\end{proposition}
\begin{proof}
	Analizziamo come commuta la $ \Res $ con i singoli isomorfimi di cui è composto $ \inv $. Siano $ e $ l'indice di ramificazione e $ f $ l'indice d'inerzia dell'estensione. Quando c'è ramificazione abbiamo che $ v_L = e \cdot v_\K $, dunque
	\[ \Res v_L^* = e \cdot \Res v_\K^* = e \cdot v_\K^* \Res, \]
	poiché $ \Res $ è un morfismo di complessi per costruzione. 
	Analogamente, la restrizione commuta anche con l'omomorfismo di collegamento $ \delta $. L'indice $ f $ coincide con il grado della relativa estensione del campo residuo, che dunque è proprio il generatore di $ G_L = f\hat{\Z} < \hat{\Z} = G_\K $; abbiamo così, ricordando che $ \gamma $ coincide con la valutazione nel generatore, che
	$$  \gamma(\Res h) = (\Res h)(1) = h(\Res(1)) = h(f) = f\cdot h(1) = f \cdot  \gamma(h).  $$
	Componendo i tre risultati, si ottiene la tesi.
	\[ \begin{tikzcd}
	\HH^2(G_\K, \, \knr^\times) \rar["v_\K^*"] \dar["\Res"]
	& \HH^2(G_\K, \, \Z) \dar["\cdot e\Res"] \rar["\delta"]
	& \HH^1(G_\K, \, \Q/\Z) \rar["\gamma"] \dar["\cdot e\Res"]
	& \Q/\Z \dar["\cdot ef"] \\
	\HH^2(G_L, \, \knr^\times) \rar["v_L^*"]
	& \HH^2(G_L, \, \Z) \rar["\delta"]
	& \HH^1(G_L, \, \Q/\Z) \rar["\gamma"]
	& \Q/\Z.
	\end{tikzcd} \]
\end{proof}

% Assioma del corpo di classe
Per estensioni ramificate la situazione non è così semplice, anche rimanendo interessati alle sole estensioni cicliche; tant'è che abbiamo bisogno di premettere un lemma tecnico alla generalizzazione del risultato appena ottenuto.

\begin{lemma}[del sottomodulo banale]
	Sia $ L/\K $ un'estensione finita di Galois di gruppo $ G $. Esiste un sottomodulo $ V \subseteq U_L^1 $ d'indice finito e coomologicamente banale.
\end{lemma}

\begin{theorem}[Assioma del Campo di Classe] \label{assioma} \todo[capire come si chiama davvero]
	Sia $ L/\K $ un'estensione di Galois, finita e di gruppo $ G = \Gal{L/\K} $ ciclico. Allora $ \HH^1(G, \, L^\times) = 0 $ e $ \Hh^0(G, \, L^\times) $ ha cardinalità $ [L \, \colon \K] $.
\end{theorem}
\begin{proof}
	La prima asserzione segue da Hilbert 90. Applichiamo il lemma del sottomodulo banale e scriviamo la successione esatta 
	\[ 1 \to V \to \mathcal{O}_L^\times \to \mathcal{O}_L^\times/V \to 1; \]
	il primo modulo è coomologicamente banale per costruzione, pertanto il suo quoziente di Herbrand è $ h(V) = 1 $, e anche l'ultimo modulo ha $ h = 1 $; infatti è finito, perché $$  [\mathcal{O}_L^\times \, \colon V] = [\mathcal{O}_L^\times \,\colon U_L^1]\cdot[U_L^1 \,\colon V],  $$
	dove il secondo indice è finito per costruzione e il primo perché il quoziente di quei due gruppi è $ \lambda^\times $ (si guardi la successione esatta $ \ref{local3} $). Segue dalle proprietà del quoziente di Herbrand che $ h(\mathcal{O}_L^\times) = 1 $. Passiamo ora alla successione esatta $ \ref{local1} $
	\[ \begin{tikzcd}[column sep = small]
	1 \rar
	& \mathcal{O}_{L}^\times \rar
	& L^\times \rar["v"]
	& \Z \rar
	& 0,
	\end{tikzcd} \]
	della quale conosciamo il quoziente del primo e dell'ultimo gruppo: sappiamo infatti che $$  \HH^1(G, \, \Z) = \Hom_\mathsf{Gr}(G, \, \Z) = \Hom_\Z(G^\texttt{ab}, \, \Z) = 0  $$ perché $ G $ è finito e agisce banalmente su $ \Z $ e che $$  \Hh^0(G, \, \Z) = \frac{\Z^G}{ N\Z} = \frac{\Z}{ |G|\, \Z}  $$ per definizione. Infine, per le proprietà del quoziente di Herbrand e l'Hilbert 90
	\[ h(L^\times) = h(\Z) \qquad\Longrightarrow\qquad |\Hh^0(\K, L^\times)\, | = |G| = [L \, \colon \K]. \]
\end{proof}


% BR(Knr/K) = Br(K)
Siamo ora pronti per dimostrare che il gruppo di Brauer relativo all'estensione non ramificata massimale è in realtà già tutto $ \Br \K $.

\begin{theorem}
	$ \Br \K = \Q/\Z $.
\end{theorem}

\begin{proof}
Dimostreremo che il gruppo di Brauer relativo ad ogni estensione di Galois di grado $ n $ coincide, come sottogruppo di $ \Br \K $, con il Brauer relativo all'unica estensione non ramificata di grado $ n $: $ \K_\mathtt{nr}^n $. È conveniente rompere la dimostrazione in due lemmi;
fissiamo un'estensione $ L/\K $ di Galois di grado $ n = [L\,\colon \K] $ e gruppo $ G $.

\begin{lemma}\label{Br1}
	La cardinalità del gruppo $ \HH^2(G,\, L^\times ) $ divide $ n $.
\end{lemma}

\begin{proof}
	Arriveremo alla tesi per approssimazione. % fuck yeah
	Il caso in cui $ G $ sia ciclico segue immediateente dall'Assioma del Campo di Classe (\ref{assioma}). Affrontiamo ora il problema per $ G $ $ p $-gruppo: grazie alla formula delle classi sappiamo che il centro è non banale, questo è abeliano e ha partanto un ottogruppo normale di ordine $ p $ che chiamiamo $ H $, il quale, essendo un sottogruppo normale di un gruppo caratteristico, è normale in $ G $. A questo gruppo sarà associata un'estensione intermedia $ L^H $, anch'essa di Galois. Dalla successione spettrale di Hoschild-Serre, o meglio per quanto dice il corollario \ref{boo2}, ricaviamo la successione esatta
	\[\begin{tikzcd}[column sep = small]
	0 \rar & \HH^q(G/H, \, {L^H}^\times) \rar["\Inf"]
	& \HH^q(G, \, L^\times) \rar["\Res"]
	& \HH^q(H, \, L^\times),
	\end{tikzcd} \qquad  \]
	da cui deduciamo la tesi per induzione. Segue immediatamente il caso generale: se per ogni primo $ p $ scegliamo un Sylow, la restrizione produce degli omomorfismi iniettivi
	\[ T_p\HH^2(G, \, L^\times) \to \HH^2(G_p, L^\times), \]
	da cui appunto, invocando il risultato sull'estensione $ L/L^{G_p} $, segue la tesi.
\end{proof}

\begin{lemma}
	$ \HH^2(G,\, L^\times ) $ e $ \HH^2(\Gal{\K_\mathtt{nr}^n/\K},\, \K_\mathtt{nr}^{n\times}) $ sono lo stesso sottogruppo di $ \Br \K $.
\end{lemma}

\begin{proof}
	La cardinalità di $ \HH^2(\Gal{\K_\mathtt{nr}^n/\K},\, \K_\mathtt{nr}^{n\times}) $ è esattamente $ n $: per la cilcicità del gruppo di Galois è la stessa del gruppo di Tate in grado zero, che è quella voluta per l'Assioma del Corpo di Classe. È quindi sufficiente mostrare che $ \HH^2(\Gal{\K_\mathtt{nr}^n/\K},\, \K_\mathtt{nr}^{n\times}) $ è un sottogruppo di $ \HH^2(G,\, L^\times ) $, grazie al risultato sulla cardinalità appena stabilito.
	Prendiamo un elemento $ x \in \HH^2(\Gal{\K_\mathtt{nr}^n/\K},\, \K_\mathtt{nr}^{n\times}) < \Br \knr < \Br \K $. Questo vive in un gruppo di $ n $-torsione e pertanto viene mandato in $ 0 $ dalla restrizione
	\[ \begin{tikzcd}
	\Br \K \rar["\Res"]
	& \Br L \\
	\HH^2(G_\K, \, \knr^\times) \rar["\Res"] \dar["\inv_\K"] \arrow[hook]{u}
	& \HH^2(G_L, \, \knr^\times) \dar["\inv_L"] \arrow[hook]{u} \\
	\Q/\Z \rar["\cdot n"]
	& \Q/\Z,
	\end{tikzcd} \]
	il cui nucleo è proprio
	\[ 0 \to \HH^2(G, \, L^\times) \to \Br \K \to \Br L \]
	per la solita successione esatta dei Brauer relativi (\ref{stronzetto}).
\end{proof}

Torniamo ora alla dimostrazione del teorema. Per il classico lemma di passaggio al limite, sappiamo che $ \Br \K = \HH^2(\Gal{\bar{\K}/\K}, \, \bar{\K}^\times) $ è limite induttivo sulle sottoestensioni di Galois finite, che per il secondo lemma possiamo calcolare sulle sole estensioni non ramificate:
\[ \Br \K = \varinjlim_{L/K} \HH^2(G, \, L^\times) = \varinjlim_{n} \HH^2(\Gal{\K_\mathtt{nr}^n/\K},\, \K_\mathtt{nr}^{n\times}) = \Br (\knr /\K) = \Q/\Z. \]

\end{proof}

\section{Tutto il resto}
In questa sezione concluderemo lo studio del gruppo di Galois assoluto di un campo locale, mostrando che questo ha dimensione coomoogica 2: tutto i gruppi di coomologia rimanenti saranno dunque nulli. Quasi. Ad essere precisi, solo quelli con coefficienti in moduli di torsione.

\begin{lemma}
	Sia $ k $ un campo e $ p $ un numero primo. I seguenti sono equivalenti:
	\begin{enumerate}
		\item La $ p $-dimensione coomologica $ \cd_p(k) \leq 1 $.
		\item Per ogni estensione algebrica separabile $ L/k $, la $ p $-torsione del Brauer è nulla: $ p \Br L = 0. $
		\item Per ogni estensione finita separabile $ L/k $, la $ p $-torsione del Brauer è nulla: $ p \Br L = 0. $
	\end{enumerate}
\end{lemma}

\begin{proof}
	Asuumiamo 1. La $ p $-dimensione coomologica del sottogruppo $ \Gal{\bar{k}/L} $ è al più quella del gruppo $ \Gal{\bar{k}/k} $, ne segue che
	\[ (\Br L) [p] = \HH^2(L, \, \mu_p) = 0. \]
	È evidente che 2 implica 3. Assumiamo dunque 3. Sia $ G_p $ un $ p $-Sylow di $ \Gal{\bar{k}/k} $ e $ L $ il sottocampo corrispondente. $ L $ contiene le radici dell'unità $ \mu_p $: infatti l'indice $ [L(\mu_p)\,\colon L] $ deve dividere sia $ p-1 $, perché campo di spezzamento di $ x^p-1 $, che $ p $, per costruzione. Ne segue che
	\[ \HH^2(L, \, \Z/p\Z) = \HH^2(L, \, \mu_p) = (\Br L)[p] = 0, \]
	per ipotesi; da cui $ \cd_p(k) \leq \cd_p(L) = 1 $.
\end{proof}

\begin{theorem}
	Ogni campo locale $ \K $ ha dimensione coomologica $ \cd(K) = 2 $.
\end{theorem}
\begin{proof}
	Mostreremo preliminarmente che l'estensione non ramificata massimale $ \knr $ ha dimensione coomologica $ 1 $. Per il lemma appena enunciato sarà sufficiente dimostrare che, per ogni primo $ p $ e ogni estensione finita e separabile $ L/\knr $, il gruppo di Brauer $ \Br L $ non ha componente $ p $-primaria: questo gruppo è limite diretto dei gruppi di Brauer $ \Br E $ delle estensioni intermedie finite $ K \subseteq E \subseteq L $, per il solito teorema di passaggio al limite (\ref{limite}); mostreremo pertanto che ogni elemento di $ p^\alpha $-torsione viene eventualmente ucciso dalle mappe di transizione (che ricordiamo essere restrizioni). Nel sistema induttivo si trova infatti, sopra $ E $, un'estensione $ F $ per cui l'indice $ [F \,\colon E] = p^\alpha $, che produce la mappa assassina che cercavamo:
	\[ \begin{tikzcd}
	\Br E \rar["\inv_E"] \dar["\Res"]
	& \Q/\Z \dar["\cdot p^\alpha"] \\
	\Br F \rar["\inv_F"]
	& \Q/\Z.
	\end{tikzcd} \]
	Riusciamo ad ottenere l'etensione $ F $ richiesta, per esempio, componendo $ E $ con un'estensione di $ K $ non ramificiata del giusto grado.\\
	
	Possiamo ora concludere: l'estensione non ramificata massimale ha gruppo di Galois $ \hat{\Z} $, che è un sottogruppo di $ \Gal{\bar{\K}/K} $, il cui quoziente abbiamo appena mostrato avere dimensione coomologica 1; per il lemma \ref{quozienti}, concludiamo che 
	$$  \cd(K) \leq \cd(\knr) + \cd(\hat{\Z}).  $$
\end{proof}