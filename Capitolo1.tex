\chapter{Lo Strumento}
\small

L'obiettivo di questo primo capitolo è di introdurre i gruppi della coomologia in questione, mostrarne le principali proprietà e presentare alcune generalizzazioni.\\

Poniamoci nella dovuta generalità. Sia $ G $ un gruppo \emph{finito}, consideriamo la categoria $ \Gmod $ costituita dai gruppi abeliani muniti di un'azione di $ G $, i cui morfismi siano le mappe che ne rispettano la struttura: gli omomorfismi di gruppo $ G $-equivarianti. Equivalentemente, possiamo pensare a $ \Gmod $ come la categoria dei moduli sull'anello di gruppo $ \Z[G] $. In quest'ottica, è naturale riferirsi agli oggetti della nostra categoria come \textquotedblleft$ G $ moduli".
Per fissare le idee si pensi, per esempio, al gruppo di Galois di un'estensione finita $ L/K $; questo agisce naturalmente su tutti i campi intermedi, rendendo sia i loro gruppo additivo che il loro gruppo moltiplicativo degli $ \Gal{L/K} $ moduli. \\

Siamo ora interessati al funtore $ \mathcal{F} $ che, preso un modulo, restituisce il sottomodulo costituito dagli elementi fissi rispetto  all'azione del gruppo
\begin{align*} 
\mathcal{F} \colon \Gmod &\to \mathcal{A}b \\
A &\mapsto A^G = \{ a \in A \mid ga = a \; \forall \, g \in G \},
\end{align*}
di evidente importanza in Teoria di Galois. Questo funtore è esatto a sinistra ma, in generale, non a destra. Partendo questo da una categoria con abbastanza iniettivi, ci è concesso prenderne il derivato destro $ R^i\mathcal{F} $, da cui la nostra prima definizione.

\begin{definition}
	Dato un gruppo finito $ G $ e un intero $ i $ non negativo, chiamiamo $ i $-esimo gruppo di coomologia di $ G $ il funtore
	\[ \HH^i(G, \, \bullet) \colon = R^i\mathcal{F}(\,\bullet\,). \]
\end{definition}

Attraverso questa definizione e la magia occulta dei funtori derivati, l'algebra omologica ci permette di assegnare a un $ G $-modulo $ A $ un'intera famiglia di invarianti $ \HH^i(G, \, A) $, il cui studio ci permetterà di descrivere meglio sia il modulo che il gruppo in questione. 

Dall'altro lato la definizione appena data sembra, al momento, un cambio di notazione completamente arbitrario: perché non continuare a chiamare i nostri oggetti funtori derivati per il resto della tesi? Limitiamoci ad osservare che la presentazione scelta mette in evidenza la proprietà fondamentale di questi oggetti.

\begin{theorem}\label{fond}
	Data una successione esatta corta di $ G $ moduli
	\[\begin{tikzcd}[column sep = small]
	0 \rar & A \rar & B \rar & C \rar & 0,
	\end{tikzcd}\]
	si ha una successione esatta lunga di gruppi abeliani
	\[\begin{tikzcd}[column sep = small]
	0 \rar & A^G \rar & B^G \rar & C^G \rar & \HH^1(G, \, A) \rar & \dots \, .
	\end{tikzcd}\]
\end{theorem}

Questa proposizione, come molte altre proprietà dei funtori derivati, non verrà dimostrata: ci limiteremo a ricordare alcuni risultati e lasceremo al corso di Istituzioni di Algebra il compito di indagare i misteri dell'algebra omologica. Per evidenziare però che non stiamo usando della teoria particolarmente esotica aggiungerei la seguente rassicurazione:

\begin{remark}
	Il funtore $ \mathcal{F} $ che abbiamo deciso di studiare non è altro che $ \Hom_G(\Z,\, \bullet\,) $: infatti, affinché un omomorfismo di gruppi da $ \Z $ in un modulo qualunque sia $ G $ equivariante, è necessario e sufficiente che l'immagine di $ 1 $ sia un punto fisso dell'azione. Il derivato destro è dunque il familiare $ \Ext^i(\Z, \, \bullet\,) $.
\end{remark}

Ricorderemo comunque velocemente la costruzione astratta del funtore derivato, cogliendo l'occasione per dare una qualche intuizione dietro alcuni dei risultati che enunceremo. Cominciamo però con una descrizione più esplicita dei nostri oggetti. \todo[questa è da sistemare]

\section{Una presentazione esplicita}
Vediamo ora la costruzione classica: quella omologica. Per costruire l'immagine di un $ G $ modulo $ A $ tramite $ R^i\mathcal{F} $, abbiamo bisogno di una risoluzione iniettiva
\[ 0 \to A \to I^{\,0} \to I^{\,1} \to I^{\,2} \to \dots, \]
che sappiamo sempre esistere in una categoria di moduli su un anello, come quella in cui troviamo. Applicando il funtore $ \mathcal{F} $, otteniamo il complesso
\[ 0 \to \mathcal{F}\left(I^{\,0}\right) \to \mathcal{F}\left(I^{\,1}\right) \to \mathcal{F}\left(I^{\,2}\right) \to \dots, \]
la cui omologia è, per definizione, $ R^i\mathcal{F}(A) = \HH^i(G, \, A) $. \\

Possiamo quindi affermare ora, con una certa convinzione, che tutti i moduli iniettivi hanno coomologia banale
\begin{equation} \label{injban}
	\HH^i(G, \, I) = 0 \quad \forall \, i > 0.
\end{equation}

Cosa ci abbiamo guadagnato?

Lavorare con delle risoluzioni iniettive ci offre il vantaggio di poter costruire facilmente omomorfismi tra i gruppi di coomologia: mentre la funtorialità in $ A $ è  




Affermiamo inoltre, in modo più ardito, che
\begin{proposition} 
	Se $ (A_j) $ è un sistema induttivo di $ G $ moduli, indicizzato su un insieme filtrante, allora
	\[ \HH^i(G, \, \lim_\rightarrow A_j) = \lim_\rightarrow \HH^i(G, \, A_j) \qquad \forall \, i \geq 0. \] 
\end{proposition}
\todo[definire i limiti iniettivi!]
\todo[come si dimostra?]


Una possibile costruzione dei gruppi $ \HH^i(G, \, A) $ si ottiene prendendo l'omologia dal complesso
\[ 0 \to K^0(A) \to K^1(A) \to K^2 (A) \to \dots, \]
dove l'oggetto $ K^i(A) = \{ f \colon G^i \to A\} $ è il gruppo (abeliano) delle applicazioni in $ i $-variabili dal gruppo $ G $ in $ A $; i differenziali $ \delta^i \colon K^i \to K^{i+1} $, nel classico stile della coomologia singolare, sono definiti in modo incomprensibile e stranamente intrigante:
\begin{align*}
	\delta f(g_1, \, \dots, \, g_{i+1}) = & \;  g_1 f(g_2, \, \dots, \, g_{i+1}) \\ & + \sum_{j = i}^{i} (-1)^j f(g_1, \, \dots, \, g_jg_{j+1}, \, \dots, \, g_{i+1}) \\ & + (-1)^{i+1} f(g_1, \dots, g_i).
\end{align*}

Questa definizione ha il vantaggio di fornirci una descrizione esplicita degli elementi del gruppo $ \HH^i(G, \, A) $; descrizione che, pur avendo una qualche utilità in grado basso ($ i = 0,\, 1 $), risulta troppo macchinosa per poterci effettivamente lavorare. Dunque, ricaviamone quanto possibile e dimentichiamocene in fretta. \\

È interessante osservare che, avendo una presentazione esplicita degli elementi, ci è concesso cominciare qualche ragionamento di cardinalità: se $ A $ è finito, allora ogni gruppo $ K^i(A) $ è finito e di conseguenza lo è anche ogni suo quoziente.
\begin{lemma}
	Se sia $ G $ che $ A $ sono finiti, allora tutti i gruppi di coomologia $ \HH^i(G, \, A) $ sono finiti.
\end{lemma}

Passiamo ora all'analisi della coomologia di grado basso: per definizione abbiamo
$$  \HH^0(G, \, A) = \ker \delta^0 = \{ f \colon G^0 \to A \mid gf-f = 0 \quad \forall \, g \in G \},  $$
identificando $ K^0(A) $ con $ A $, mandando $ f $ nell'unico elemento che compone la sua immagine, scopriamo che
\[ \HH^0(G, \, A) = \{ a \in A \mid ga = a \quad \forall \, g \in G \} = A^G. \]
Abbiamo ottenuto una descrizione piuttosto piacevole del gruppo di grado zero! Questo risultato è tutt'altro che inaspettato, è vero per ogni funtore derivato e si dimostra altrettanto facilmente andando a studiare la costruzione della risoluzione iniettiva di cui sopra. In grado uno il calcolo si fa appena più complesso: abbiamo bisogno di scriverci esplicitamente i cocicli
\[ Z^1(A) = \ker \delta^1 = \{ f \colon G \to A \mid f(gh) = gf(h) + f(g) \}, \]
poi i cobordi
\[ B^1(A) = \im \delta^0 = \{ ga - a \mid g \in G, \, a \in A  \}, \]
e infine il quoziente
\[ \HH^1(G, \, A) = \frac{Z^1(A)}{B^1(A)} = \frac{\{ f(gh) = gf(h) + f(g) \}}{\{ ga - a\}}. \]
Il risultato è meno chiaro del precedente, per non dire più deludente; tuttavia, nel caso in cui l'azione di $ G $ su $ A $ sia banale otteniamo una più profonda comprensione dell'oggetto, infatti il denominatore scompare e scopriamo che
\[ \HH^1(G, \, A) = \{ f(gh) = f(h) + f(g) \} = \Hom_\Z(G, \, A). \]
Accettiamo il risultato come vagamente interessante e procediamo.

\section{Funtorialità}
I gruppi di coomologia sono funtoriali nei coefficienti $ A $ per costruzione. Siamo ora interessati a capire che tipo di relazioni otteniamo al variare del gruppo $ G $ che agisce. Un problema che si pone immediatamente, seppur puramente formale, è che cambiando gruppo usciamo dalla categoria ambiente: abbiamo quindi bisogno del cugino del funtore dimenticante.

\begin{definition}
	Dati due gruppi finiti $ G $ e $ G' $, un omomorfismo $ f \colon G' \to G $ e uno $ G $ modulo $ A $, chiamiamo $ f^\times A $ il gruppo abeliano $ A $ equipaggiato dell'azione
	\[ g' \cdot_{G'} A := f(g') \cdot_{G} A \qquad \forall\, g' \in G'. \]
\end{definition}

A cui diamo questo simpatico appellattivo perché nel caso a cui saremo principalmente interessati, ovvero passare da $ G $ a un suo sottogruppo $ H $ tramite l'immersione naturale $ i \colon H \to G $, il funtore $ f^\times $ semplicemente \textquotedblleft dimentica" come gli elementi fuori da $ H $ agiscono sul modulo in questione. In queto particolare caso questa operazione è, in un qualche senso, parzialmente invertibile:

\begin{definition}
	Preso un sottogruppo $ H < G $ e un $ H $-modulo $ A $, chiamiamo modulo indotto il gruppo abeliano
	\[ \Ind_H^G(A) := \{ f \colon G \to A \mid f(hg) = hf(g) \; \forall\, h \in H \} = \Hom_H(G, \, A), \]
	munito dell'azione 
	\[g \cdot f(x) = f(xg) \qquad\forall\, g \in G. \]
	
\end{definition}

\begin{proposition}\label{aggiunzione}\todo[dimostrare, introdurre]
	I due funtori
	\[ f^\times \colon \Gmod \to \Gmod \qquad\text{ e }\qquad  \Ind_H^G \colon \Gmod \to \Gmod \]
	sono aggiunti. Ovvero, per ogni $ A \in \Gmod $ e $ B \in \Gmod $, soddisfano
	\[ \Hom_G(A,\, \Ind_H^G(B)) = \Hom_H(f^\times A, \, B). \]
\end{proposition}

Siamo ora pronti per affrontare il risultato principale, che ci permetterà di costruire tutte le mappe di cui avremo bisogno in seguito.
\begin{proposition}
	Siano  $ A $ uno $ G $ modulo e $ A' $ uno $ G' $ modulo. Dati $ f \colon G' \to G $ un omomorfismo di gruppi e $ u \colon f^\times A \to A' $ un omorfismo $ G' $-equivariante, otteniamo una mappa in coomologia
	\[ \HH^i(G, \, A) \to \HH^i(G', \, A') \qquad\forall i \geq 0. \]
\end{proposition}

\begin{proof}
	Osserviamo che, per definizione di $ f^\times A $, abbiamo un'inclusione
	\[ A^G \hookrightarrow \left(f^\times A\right)^{G'} \]
	che vogliamo estendere a un morfismo tra risoluzioni iniettive. Presa una risoluzione iniettiva di $ A $
	\[ 0 \to A \to I^{\,0} \to I^{\,1} \to I^{\,2} \to \dots, \]
	pocihé il funtore $ f^\times $ preserva gli iniettivi, anche
	\[ 0 \to f^\times A \to f^\times I^{\,0} \to f^\times I^{\,1} \to f^\times I^{\,2} \to \dots \]
	è una risoluzione iniettiva. Ci troviamo quindi nella seguente situazione:
	\[\begin{tikzcd}[column sep = small]
	0 \rar & A^G \rar \dar & \left(I^{\,0}\right)^G \rar & \left(I^{\,1}\right)^G \rar & \left(I^{\,2}\right)^G \rar & \dots \\
	0 \rar & \left(f^\times A\right)^{G'}  \rar & \left(f^\times I^{\,0}\right)^{G'} \rar & \left(f^\times I^{\,1}\right)^{G'} \rar & \left(I^{\, 2}\right)^{G'} \rar & \dots
	\end{tikzcd} \]
	Per definizione di modulo iniettivo otteniamo tutte le mappe successive
	\[\begin{tikzcd}[column sep = small]
	0 \rar
	& A^G \dar \rar \arrow[dashed, "comp." description]{dr}
	& \left(I^{\,0}\right)^G \rar \dar[dashed, "inj." description] \arrow[dashed, gray]{dr}
	& \left(I^{\,1}\right)^{G} \rar \dar[dashed, gray] \arrow[dashed, gray!50]{dr}
	& \left(I^{\, 2}\right)^{G} \dar[dashed, gray!50] \rar
	& \dots \\
	0 \rar
	& \left(f^\times A\right)^{G'} \rar
	& \left(f^\times I^{\,0}\right)^{G'} \rar
	& \left(f^\times I^{\,1}\right)^{G'} \rar
	& \left(f^\times I^{\,2}\right)^{G'} \rar
	& \dots
	\end{tikzcd} \]
	e un morfismo di complessi tra risoluzioni iniettive passa al quoziente, fornendoci una mappa
	\[ \HH^i(G, \, A) \to \HH^i(G', \, f^\times A) \qquad \forall i \geq 0. \]
	Abbiamo già osservato che $ \HH^i $ è funtoriale nel secondo argomento, dunque otteniamo la tesi componendo la mappa appena trovata con quella indotta da $ u $.
\end{proof}

Affrontato il problema nella dovuta generalità, decliniamo la soluzione nei casi a cui siamo interessati: fissiamo $ H $ un sottogruppo di $ G $.
\begin{definition}[Restrizione]
	Chiamiamo mappa di restrizione l'omomorfismo
	\[ \Res \colon \HH^i(G, \, A) \to \HH^i(H, \, f^\times A) \]
	indotto da $ i \colon H \to G $.
\end{definition}

\begin{definition}[Inflazione]
	Chiamiamo mappa d'inflazione l'omomorfismo
	\[ \Inf \colon \HH^i(G/H, \, A^H) \to \HH^i(G, \, A) \]
	indotto dalla proiezione natuarle $ \pi \colon G \to G/H $ e dall'inclusione $ u \colon A^H \to A $.
\end{definition}

Avendo sopra introdotto i moduli indotti, si sarebbe potuta fare un'altra scelta naturale nel tentativo di definire una mappa di restrizione, anch'essa di un certo interesse: abbiamo un omomorfismo \[ \HH^i(G, \, \Ind_H^G(B)) \to \HH^i(H, \, B) \] indotto dall'inclusione naturale $ i \colon H \to G $ e dal morfismo $ u \colon \Ind_H^G(B) \to B $ che valuta ogni mappa nell'elemento neutro $ f \mapsto f(1) $. 
\begin{proposition}[Lemma di Shapiro]
	L'omomorfismo sopra definito
	\[ \HH^i(G, \, \Ind_H^G(B)) \to \HH^i(H, \, B) \]
	è un isomorfismo.
\end{proposition}
\begin{proof}
	Per l'aggiunzione del funtore $ \Ind_H^G $ con il funtore dimenticante $ f^\times $ (\ref{aggiunzione}), abbiamo che
	\[ \Hom_G(\Z,\, \Ind_H^G(B)) = \Hom_H(f^\times \Z, \, B), \]
	ma, essendo $ \Z $ già munito dell'azione banale, scopriamo che i due funtori $ \HH^i(G, \, \Ind_H^G(\,\bullet\,)) $ e $ \HH^i(H, \, \bullet\,) $ sono il derivato dello stesso $ \Hom_H(\Z, \,\bullet \,) $ e pertanto coincidono.
\end{proof}

Portiamo all'attenzione un'ultima rilevante costruzione: lasciamo agire $ G $ su se stesso per coniugio, fissato un elmento $ t \in G $ otteniamo un automorfismo interno $ f_t \colon G \to G $. Questo induce naturalmente un automorfismo di $ \HH^i(G, \, A) $ che riusciamo a descrivere esplicitamente.

\begin{proposition}
	L'isomorfimo indotto dal coniugio per $ t $
	\[ \sigma_t \colon \HH^i(G, \, A) \to \HH^i(G, \, A) \]
	è l'identità.
\end{proposition}
\begin{proof}
	Dimostriamo il risultato per induzione su $ i $: per $ i = 0 $ non c'è nulla da dimostrare. Supponiamo ora la tesi vera fino al grado $ i $ per ogni $ G $ modulo. Preso ora un qualunque modulo $ A $, immergiamolo in modulo iniettivo $ I $:
	\[ 0 \to A \to I \to Q \to 0. \]
	Prendendo i fissati otteniamo una successione esatta lunga (per \ref{fond}) costellata di $ 0 $ ovunque si prenda la coomologia di $ I $ (per \ref{injban}), rimane solo il segmento iniziale
	\[\begin{tikzcd}
	0 \rar & A^G \rar\dar["\sigma_t = \id" description] & I^G \rar\dar["\sigma_t = \id" description] & Q^G \rar\dar["\sigma_t = \id" description] & \HH^1(G, \, A) \rar\dar["\sigma_t"] & 0 \\
	0 \rar & A^G \rar & I^G \rar & Q^G \rar & \HH^1(G, \, A) \rar & 0
	\end{tikzcd}\]
	e alcuni frammenti per $ i > 0 $
	\[\begin{tikzcd}
	0 \rar & \HH^i(G, \, Q) \rar\dar["\sigma_t = \id" description] & \HH^{i+1}(G, \, A) \rar\dar["\sigma_t"] & 0 \\
	0 \rar & \HH^i(G, \, Q) \rar & \HH^{i+1}(G, \, A) \rar & 0,
	\end{tikzcd}\]
	da cui deduciamo la tesi in grado $ i+1 $.
\end{proof}
\newpage
\section{Inflazione e Restrizione}

\section{Corestrizione}

\section{Gruppi Modificiati di Tate}