\chapter{Lo Strumento}

L'obiettivo di questo primo capitolo è di introdurre i gruppi della coomologia in questione, mostrarne le principali proprietà e presentare alcune generalizzazioni. Prima di cominciare vorrei però convincere il lettore dell'utilità del linguaggio che stiamo per costruire.\\

ESEMPIO ? \\

\section{Costruzione}

Poniamoci nella dovuta generalità. Fissiamo un gruppo finito $ G $ e consideriamo la categoria $ \mathcal{M}od_G $ costituita dai gruppi abeliani muniti di un'azione di $ G $, i cui morfismi sono le mappe che ne rispettano la struttura: gli omomorfismi di gruppo $ G $-equivarianti. Equivalentemente, possiamo pensare a $ \mathcal{M}od_G $ come la categoria dei moduli sull'anello $ \Z[G] $, che otteniamo ponendo sullo $ \Z $ modulo libero generato sugli elementi di $ G $ la moltiplicazione indotta da questi ultimi. In quest'ottica è naturale riferirsi agli oggetti della nostra categoria come "$ G $ moduli". \\

Siamo ora interessati al funtore che, preso un modulo, restituisce il sottomodulo fissato degli elementi fissati dall'azione del gruppo
\begin{align*}
	F \colon \mathcal{M}od_G &\to \mathcal{A}b \\
	 A &\mapsto A^G = \{ a \in A \mid ga = a \; \forall \, g \in G \},
\end{align*}
fondante in Teoria di Galois. Questo funtore è esatto a sinistra ma, in generale, non a destra. Vivendo questo nella categoria $ \mathcal{M}od_G $, che ha abbastanza iniettivi, ci è concesso prenderne il derivato destro $ R^iF $, da qui la nostra prima definizione.

\begin{definition}
	Dato un gruppo finito $ G $ e un intero $ i \geq 0 $, chiamiamo $ i $-esimo gruppo di coomologia di $ G $ il funtore
	\[ \HH^i(G, \, \bullet) \colon = R^iF(\,\bullet\,). \]
\end{definition}

Ricordiamo velocemente com'è costruito il funtore derivato: innanzitutto sostituiamo il modulo $ A $ di cui vogliamo prendere i fissati con una sua risoluzione iniettiva
\[\begin{tikzcd}[column sep = small]
0 \rar & A \rar & I^0 \rar & I^1 \rar & I^2 \rar & \dots
\end{tikzcd}\]
cosa che possiamo fare perché, appunto, $ \mathcal{M}od_G $ ha abbastanza iniettivi, essendo una categoria di moduli su un anello. Applichiamo ora il funtore
\[\begin{tikzcd}[column sep = small]
0 \rar & (I^0)^G \rar & (I^1)^G \rar & (I^2)^G \rar & \dots
\end{tikzcd}\]
e prendiamo la coomologia del complesso così ottenuto.

\section{Primissimi preliminari}

Come tutti ben sappiamo, esistono le definizioni
\begin{definition}
Una definizione è una cosa del genere.
\end{definition}

Poi ci sono i teoremi:
\begin{theorem}
Ogni intero positivo è somma di quattro quadrati.
\end{theorem}

E le dimostrazioni:
\begin{proof}
Le somme di quattro quadrati sono un monoide moltiplicativo (pensare ai quaternioni). Che ogni primo sia somma di quattro quadrati è ben noto. Una dimostrazione completa si può trovare in \cite[Teorema 2.3]{ArticoloFondamentale}.
\end{proof}

Poi lemmi, proposizioni, corollari, ed esempi, solo per citarne alcuni:

\begin{example}
\begin{equation}\label{eq:63}
63=7^2+3^2+2^2+1
\end{equation}
\end{example}
\begin{proposition}
Se $n \equiv 7 \pmod 8$, allora $n$ non si scrive come somma di tre quadrati di interi.
\end{proposition}
\begin{lemma}\label{lemma:Gamma}
\[
\lim_{n \to \infty} \left( \log(n) - \sum_{i=1}^n \frac{1}{i} \right)= -\gamma
\]
\end{lemma}
\begin{corollary}
\[
\zeta(s) = \frac{1}{s-1} + \gamma + O(s-1) \quad \text{ quando } s \to 1
\]
\end{corollary}

\section{Riferimenti}
Nel resto della tesi si può poi fare riferimento a risultati già dimostrati, come ad esempio l'utilissimo Lemma \ref{lemma:Gamma}, o l'ancora più fondamentale Equazione \eqref{eq:63}.