\chapter{Lo Strumento}
\small

L'obiettivo di questo primo capitolo è di introdurre i gruppi della coomologia in questione, mostrarne le principali proprietà e presentare alcune generalizzazioni.\\

Poniamoci nella dovuta generalità. Sia $ G $ un gruppo \emph{finito}, consideriamo la categoria $ \Gmod $ costituita dai gruppi abeliani muniti di un'azione di $ G $, i cui morfismi siano le mappe che ne rispettano la struttura: gli omomorfismi di gruppo $ G $-equivarianti. Equivalentemente, possiamo pensare a $ \Gmod $ come la categoria dei moduli sull'anello di gruppo $ \Z[G] $. In quest'ottica, è naturale riferirsi agli oggetti della nostra categoria come \textquotedblleft$ G $ moduli".
Per fissare le idee si pensi, per esempio, al gruppo di Galois di un'estensione finita $ L/K $; questo agisce naturalmente su tutti i campi intermedi, rendendo sia i loro gruppo additivo che il loro gruppo moltiplicativo degli $ \Gal{L/K} $ moduli. \\

Siamo ora interessati al funtore $ \mathcal{F} $ che, preso un modulo, restituisce il sottomodulo costituito dagli elementi fissi rispetto  all'azione del gruppo
\begin{align*} 
\mathcal{F} \colon \Gmod &\to \mathcal{A}b \\
A &\mapsto A^G = \{ a \in A \mid ga = a \; \forall \, g \in G \},
\end{align*}
di evidente importanza in Teoria di Galois. Questo funtore è esatto a sinistra ma, in generale, non a destra. Partendo questo da una categoria con abbastanza iniettivi, ci è concesso prenderne il derivato destro $ R^i\mathcal{F} $, da cui la nostra prima definizione.

\begin{definition}
	Dato un gruppo finito $ G $ e un intero $ i $ non negativo, chiamiamo $ i $-esimo gruppo di coomologia di $ G $ il funtore
	\[ \HH^i(G, \, \bullet) \colon = R^i\mathcal{F}(\,\bullet\,). \]
\end{definition}

Attraverso questa definizione e la magia occulta dei funtori derivati, l'algebra omologica ci permette di assegnare a un $ G $-modulo $ A $ un'intera famiglia di invarianti $ \HH^i(G, \, A) $, il cui studio ci permetterà di descrivere meglio sia il modulo che il gruppo in questione. 

Dall'altro lato la definizione appena data sembra, al momento, un cambio di notazione completamente arbitrario: perché non continuare a chiamare i nostri oggetti funtori derivati per il resto della tesi? Limitiamoci ad osservare che la presentazione scelta mette in evidenza la proprietà fondamentale di questi oggetti.

\begin{theorem}
	Data una successione esatta corta di $ G $ moduli
	\[\begin{tikzcd}[column sep = small]
	0 \rar & A \rar & B \rar & C \rar & 0,
	\end{tikzcd}\]
	si ha una successione esatta lunga di gruppi abeliani
	\[\begin{tikzcd}[column sep = small]
	0 \rar & A^G \rar & B^G \rar & C^G \rar & \HH^1(G, \, A) \rar & \dots \, .
	\end{tikzcd}\]
\end{theorem}

Questa proposizione, come molte altre proprietà dei funtori derivati, non verrà dimostrata: ci limiteremo a ricordare alcuni risultati e lasceremo al corso di Istituzioni di Algebra il compito di indagare i misteri dell'algebra omologica. Per evidenziare però che non stiamo usando della teoria particolarmente esotica aggiungerei la seguente osservazione:

\begin{remark}
	Il funtore $ \mathcal{F} $ che abbiamo deciso di studiare non è altro che il noto funtore esatto a sinistra $ \Hom_G(\Z,\, \bullet\,) $: infatti, affinché un omomorfismo di gruppi da $ \Z $ in un modulo qualunque sia $ G $ equivariante, è necessario e sufficiente che l'immagine di $ 1 $ sia un punto fisso dell'azione. Il derivato destro è dunque il familiare $ \Ext^i(\Z, \, \bullet\,) $.
\end{remark}

Ricorderemo comunque velocemente la costruzione astratta del funtore derivato, cogliendo l'occasione per dare una qualche intuizione dietro alcuni dei risultati che enunceremo. Mostreremo inoltre una costruzione dal taglio decisamente coomologico.

\section{Le costruzioni}
Per costruire l'immagine di un $ G $ modulo $ A $ tramite $ R^i\mathcal{F} $, abbiamo bisogno di una risoluzione iniettiva
\[ 0 \to A \to I^{\,0} \to I^{\,1} \to I^{\,2} \to \dots, \]
che sappiamo sempre esistere in una categoria di moduli su un anello, come quella in cui troviamo. Applicando il funtore $ \mathcal{F} $, otteniamo il complesso
\[ 0 \to \mathcal{F}\left(I^{\,0}\right) \to \mathcal{F}\left(I^{\,1}\right) \to \mathcal{F}\left(I^{\,2}\right) \to \dots, \]
la cui omologia è, per definizione, $ R^i\mathcal{F}(A) = \HH^i(G, \, A) $. \\

Possiamo quindi affermare ora, con una certa convinzione, che tutti i moduli iniettivi hanno coomologia banale
\[ \HH^i(G, \, I) = 0 \quad \forall \, i > 0. \]

Affermiamo inoltre, in modo più ardito, che
\begin{proposition} \todo[definire i limiti iniettivi!]
	Se $ (A_j) $ è un sistema induttivo di $ G $ moduli, indicizzato su un insieme filtrante, allora
	\[ \HH^i(G, \, \lim_\rightarrow A_j) = \lim_\rightarrow \HH^i(G, \, A_j) \qquad \forall \, i \geq 0. \] 
\end{proposition} 
Una costruzione alternativa dei gruppi $ \HH^i(G, \, A) $ si ottiene prendendo l'omologia dal complesso
\[ 0 \to K^0(A) \to K^1(A) \to K^2 (A) \to \dots, \]
dove l'oggetto $ K^i(A) = \{ f \colon G^i \to A\} $ è il gruppo (abeliano) delle applicazioni in $ i $-variabili dal gruppo $ G $ in $ A $; i differenziali $ \delta^i \colon K^i \to K^{i+1} $, nel classico stile della coomologia singolare, sono definiti in modo incomprensibile e stranamente intrigante:
\begin{align*}
	\delta f(g_1, \, \dots, \, g_{i+1}) = & \;  g_1 f(g_2, \, \dots, \, g_{i+1}) \\ & + \sum_{j = i}^{i} (-1)^j f(g_1, \, \dots, \, g_jg_{j+1}, \, \dots, \, g_{i+1}) \\ & + (-1)^{i+1} f(g_1, \dots, g_i).
\end{align*}

Questa definizione ha il vantaggio di fornirci una descrizione esplicita degli elementi del gruppo di coomologia $ \HH^i(G, \, A) $; descrizione che, pur avendo una qualche utilità in grado basso ($ i = 0,\, 1 $), risulta troppo macchinosa per poterci effettivamente lavorare. Dunque, ricaviamone quanto possibile e dimentichiamocene in fretta. \\

È interessante osservare che, avendo una presentazione esplicita degli elementi, ci è concesso cominciare qualche ragionamento di cardinalità: se $ A $ è finito, allora ogni gruppo $ K^i(A) $ è finito e di conseguenza lo è anche ogni suo quoziente. Abbiamo quindi il seguente risultato.
\begin{lemma}
	Se sia $ G $ che $ A $ sono finiti, allora tutti i gruppi di coomologia $ \HH^i(G, \, A) $ sono finiti.
\end{lemma}

Passiamo ora all'analisi della coomologia di grado basso: per definizione abbiamo
$$  \HH^0(G, \, A) = \ker \delta^0 = \{ f \colon G^0 \to A \mid gf-f = 0 \quad \forall \, g \in G \},  $$
identificando $ K^0(A) $ con $ A $, mandando $ f $ nell'unico elemento che compone la sua immagine, scopriamo che
\[ \HH^0(G, \, A) = \{ a \in A \mid ga = a \quad \forall \, g \in G \} = A^G. \]
Abbiamo ottenuto una descrizione piuttosto piacevole del gruppo di grado zero! Questo risultato è tutt'altro che inaspettato, è vero per ogni funtore derivato e si dimostra altrettanto facilmente andando a studiare la costruzione della risoluzione iniettiva di cui sopra. In grado uno il calcolo si fa appena più complesso: abbiamo bisogno di scriverci esplicitamente i cocicli
\[ Z^1(A) = \ker \delta^1 = \{ f \colon G \to A \mid f(gh) = gf(h) + f(g) \}, \]
poi i cobordi
\[ B^1(A) = \im \delta^0 = \{ ga - a \mid g \in G, \, a \in A  \}, \]
e infine il quoziente
\[ \HH^1(G, \, A) = \frac{Z^1(A)}{B^1(A)} = \frac{\{ f(gh) = gf(h) + f(g) \}}{\{ ga - a\}}. \]
Il risultato è meno chiaro del precedente, per non dire più deludente; tuttavia, nel caso in cui l'azione di $ G $ su $ A $ sia banale otteniamo una più profonda comprensione dell'oggetto, infatti il denominatore scompare e scopriamo che
\[ \HH^1(G, \, A) = \{ f(gh) = f(h) + f(g) \} = \Hom_\Z(G, \, A). \]
Accettiamo il risultato come vagamente interessante e procediamo.

\section{Moduli Indotti}


\newpage
\section{Inflazione e Restrizione}

\section{Corestrizione}

\section{Gruppi Modificiati di Tate}