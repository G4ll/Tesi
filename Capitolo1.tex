\chapter{Preliminari}

\section{Primissimi preliminari}

Come tutti ben sappiamo, esistono le definizioni
\begin{definition}
Una definizione è una cosa del genere.
\end{definition}

Poi ci sono i teoremi:
\begin{theorem}
Ogni intero positivo è somma di quattro quadrati.
\end{theorem}

E le dimostrazioni:
\begin{proof}
Le somme di quattro quadrati sono un monoide moltiplicativo (pensare ai quaternioni). Che ogni primo sia somma di quattro quadrati è ben noto. Una dimostrazione completa si può trovare in \cite[Teorema 2.3]{ArticoloFondamentale}.
\end{proof}

Poi lemmi, proposizioni, corollari, ed esempi, solo per citarne alcuni:

\begin{example}
\begin{equation}\label{eq:63}
63=7^2+3^2+2^2+1
\end{equation}
\end{example}
\begin{proposition}
Se $n \equiv 7 \pmod 8$, allora $n$ non si scrive come somma di tre quadrati di interi.
\end{proposition}
\begin{lemma}\label{lemma:Gamma}
\[
\lim_{n \to \infty} \left( \log(n) - \sum_{i=1}^n \frac{1}{i} \right)= -\gamma
\]
\end{lemma}
\begin{corollary}
\[
\zeta(s) = \frac{1}{s-1} + \gamma + O(s-1) \quad \text{ quando } s \to 1
\]
\end{corollary}

\section{Riferimenti}
Nel resto della tesi si può poi fare riferimento a risultati già dimostrati, come ad esempio l'utilissimo Lemma \ref{lemma:Gamma}, o l'ancora più fondamentale Equazione \eqref{eq:63}.