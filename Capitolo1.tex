\chapter{Lo Strumento}


L'obiettivo di questo primo capitolo è di introdurre la coomologia di gruppi, mostrarne le principali proprietà e presentare alcune generalizzazioni.\\

Poniamoci nella dovuta generalità. Sia $ G $ un gruppo \emph{finito}, consideriamo la categoria $ \Gmod $ costituita dai gruppi abeliani muniti di un'azione di $ G $, i cui morfismi siano le mappe che ne rispettano la struttura: gli omomorfismi di gruppo $ G $-equivarianti. Equivalentemente, possiamo pensare a $ \Gmod $ come la categoria dei moduli sull'anello di gruppo $ \Z[G] $. In quest'ottica, è naturale riferirsi agli oggetti della nostra categoria come \textquotedblleft$ G $ moduli".
Per fissare le idee si pensi, per esempio, al gruppo di Galois di un'estensione finita $ L/K $; questo agisce naturalmente su tutti i campi intermedi, rendendo sia i loro gruppo additivo che il loro gruppo moltiplicativo dei $ \Gal{L/K} $ moduli. \\

Siamo ora interessati al funtore $ \mathcal{F} $ che, preso un modulo, restituisce il sottomodulo costituito dagli elementi fissi rispetto  all'azione del gruppo
\begin{align*} 
\mathcal{F} \colon \Gmod &\to \mathcal{A}b \\
A &\mapsto A^G = \{ a \in A \mid ga = a \; \forall \, g \in G \},
\end{align*}
di evidente importanza in Teoria di Galois. Questo funtore è esatto a sinistra ma, in generale, non a destra. Partendo questo da una categoria con abbastanza iniettivi, ci è concesso prenderne il derivato destro $ R^i\mathcal{F} $, da cui la nostra prima definizione.

\begin{definition}
	Dato un gruppo finito $ G $ e un intero $ i $ non negativo, chiamiamo $ i $-esimo gruppo di coomologia di $ G $ il funtore
	\[ \HH^i(G, \, \bullet\,) : = R^i\mathcal{F}(\,\bullet\,). \]
\end{definition}

Attraverso questa definizione e la magia occulta dei funtori derivati, l'algebra omologica ci permette di assegnare a un $ G $-modulo $ A $ un'intera famiglia di invarianti $ \HH^i(G, \, A) $, il cui studio ci permetterà di descrivere meglio sia il modulo che il gruppo in questione. 

Dall'altro lato la definizione appena data sembra, al momento, un cambio di notazione completamente arbitrario: perché non continuare a chiamare i nostri oggetti funtori derivati per il resto della tesi? Limitiamoci ad osservare che la presentazione scelta mette in evidenza la proprietà fondamentale di questi oggetti.

\begin{theorem}\label{fond} \todo[questa si può fare meglio]
	Data una successione esatta corta di $ G $ moduli
	\[\begin{tikzcd}[column sep = small]
	0 \rar & A \rar & B \rar & C \rar & 0,
	\end{tikzcd}\]
	si ha una successione esatta lunga di gruppi abeliani
	\[\begin{tikzcd}[column sep = small, row sep = small]
	0 \rar & A^G \rar & B^G \arrow[d,phantom, ""{coordinate, name=Z}] \rar & C^G \arrow[dll, 
	rounded corners = 6 pt, 
	to path={ --([xshift=3ex]\tikztostart.east)
		|- (Z)[near end]\tikztonodes
		-| ([xshift=-3ex]\tikztotarget.west)
		-- (\tikztotarget)}] \\
	\qquad \quad &\HH^1(G, \, A) \rar & \HH^1(G, \, B) \arrow[d,phantom, ""{coordinate, name=W}] \rar & \HH^1(G, \, C) \arrow[dll, 
	rounded corners = 6 pt, 
	to path={ --([xshift=3ex]\tikztostart.east)
		|- (W)[near end]\tikztonodes
		-| ([xshift=-3ex]\tikztotarget.west)
		-- (\tikztotarget)}]  \\
	&\HH^2(G, \, A) \rar & \dots \, .
	& \end{tikzcd}\]
\end{theorem}

Questa proposizione, come molte altre proprietà dei funtori derivati, non verrà dimostrata: ci limiteremo a ricordare alcuni risultati e lasceremo al corso di Istituzioni di Algebra il compito di indagare i misteri dell'algebra omologica. Per evidenziare però che non stiamo usando della teoria particolarmente esotica aggiungerei la seguente rassicurazione:

\begin{remark}
	Il funtore $ \mathcal{F} $ che abbiamo deciso di studiare non è altro che $ \Hom_G(\Z,\, \bullet\,) $: infatti, affinché un omomorfismo di gruppi da $ \Z $ in un modulo qualunque sia $ G $ equivariante, è necessario e sufficiente che l'immagine di $ 1 $ sia un punto fisso dell'azione. Il derivato destro è dunque il familiare $ \Ext^i_G(\Z, \, \bullet\,) $.
\end{remark}

Ricorderemo comunque velocemente la costruzione astratta del funtore derivato, cogliendo l'occasione per dare una qualche intuizione dietro alcuni dei risultati che enunceremo. Cominciamo però con una descrizione più esplicita dei nostri oggetti. \todo[questa è da sistemare]

\section{Una presentazione esplicita}
Vediamo ora la costruzione classica: quella omologica. Per costruire l'immagine di un $ G $ modulo $ A $ tramite $ R^i\mathcal{F} $, abbiamo bisogno di una risoluzione iniettiva
\[ 0 \to A \to I^{\,0} \to I^{\,1} \to I^{\,2} \to \dots, \]
che sappiamo sempre esistere in una categoria di moduli su un anello, come quella in cui troviamo. Applicando il funtore $ \mathcal{F} $, otteniamo il complesso
\[ 0 \to \mathcal{F}\left(I^{\,0}\right) \to \mathcal{F}\left(I^{\,1}\right) \to \mathcal{F}\left(I^{\,2}\right) \to \dots, \]
la cui omologia è, per definizione, $ R^i\mathcal{F}(A) = \HH^i(G, \, A) $. \\

Possiamo quindi affermare ora, con una certa convinzione, che tutti i moduli iniettivi hanno coomologia banale
\begin{equation} \label{injban}
	\HH^i(G, \, I) = 0 \quad \forall \, i > 0.
\end{equation}

Cosa ci abbiamo guadagnato?

Lavorare con delle risoluzioni iniettive ci offre il vantaggio di poter costruire facilmente omomorfismi tra i gruppi di coomologia: mentre la funtorialità in $ A $ è  




Affermiamo inoltre, in modo più ardito, che
\begin{proposition} 
	Se $ (A_j) $ è un sistema induttivo di $ G $ moduli, indicizzato su un insieme filtrante, allora
	\[ \HH^i(G, \, \lim_\rightarrow A_j) = \lim_\rightarrow \HH^i(G, \, A_j) \qquad \forall \, i \geq 0. \] 
\end{proposition}
\todo[definire i limiti iniettivi!]
\todo[come si dimostra?]


Una possibile costruzione dei gruppi $ \HH^i(G, \, A) $ si ottiene prendendo l'omologia dal complesso
\[ 0 \to K^0(A) \to K^1(A) \to K^2 (A) \to \dots, \]
dove l'oggetto $ K^i(A) = \{ f \colon G^i \to A\} $ è il gruppo (abeliano) delle applicazioni in $ i $-variabili dal gruppo $ G $ in $ A $; i differenziali $ \delta^i \colon K^i \to K^{i+1} $, nel classico stile della coomologia singolare, sono definiti in modo incomprensibile e stranamente intrigante:
\begin{align*}
	\delta f(g_1, \, \dots, \, g_{i+1}) = & \;  g_1 f(g_2, \, \dots, \, g_{i+1}) \\ & + \sum_{j = i}^{i} (-1)^j f(g_1, \, \dots, \, g_jg_{j+1}, \, \dots, \, g_{i+1}) \\ & + (-1)^{i+1} f(g_1, \dots, g_i).
\end{align*}

Questa definizione ha il vantaggio di fornirci una descrizione esplicita degli elementi del gruppo $ \HH^i(G, \, A) $; descrizione che, pur avendo una qualche utilità in grado basso ($ i = 0,\, 1 $), risulta troppo macchinosa per poterci effettivamente lavorare. Dunque, ricaviamone quanto possibile e dimentichiamocene in fretta. \\

È interessante osservare che, avendo una presentazione esplicita degli elementi, ci è concesso cominciare qualche ragionamento di cardinalità: se $ A $ è finito, allora ogni gruppo $ K^i(A) $ è finito e di conseguenza lo è anche ogni suo quoziente.
\begin{lemma}
	Se sia $ G $ che $ A $ sono finiti, allora tutti i gruppi di coomologia $ \HH^i(G, \, A) $ sono finiti.
\end{lemma}

Passiamo ora all'analisi della coomologia di grado basso: per definizione abbiamo
$$  \HH^0(G, \, A) = \ker \delta^0 = \{ f \colon G^0 \to A \mid gf-f = 0 \quad \forall \, g \in G \},  $$
identificando $ K^0(A) $ con $ A $, mandando $ f $ nell'unico elemento che compone la sua immagine, scopriamo che
\[ \HH^0(G, \, A) = \{ a \in A \mid ga = a \quad \forall \, g \in G \} = A^G. \]
Abbiamo ottenuto una descrizione piuttosto piacevole del gruppo di grado zero! Questo risultato è tutt'altro che inaspettato, è vero per ogni funtore derivato e si dimostra altrettanto facilmente andando a studiare la costruzione della risoluzione iniettiva di cui sopra. In grado uno il calcolo si fa appena più complesso: abbiamo bisogno di scriverci esplicitamente i cocicli
\[ Z^1(A) = \ker \delta^1 = \{ f \colon G \to A \mid f(gh) = gf(h) + f(g) \}, \]
poi i cobordi
\[ B^1(A) = \im \delta^0 = \{ ga - a \mid g \in G, \, a \in A  \}, \]
e infine il quoziente
\[ \HH^1(G, \, A) = \frac{Z^1(A)}{B^1(A)} = \frac{\{ f(gh) = gf(h) + f(g) \}}{\{ ga - a\}}. \]
Il risultato è meno chiaro del precedente, per non dire più deludente; tuttavia, nel caso in cui l'azione di $ G $ su $ A $ sia banale otteniamo una più profonda comprensione dell'oggetto, infatti il denominatore scompare e scopriamo che
\[ \HH^1(G, \, A) = \{ f(gh) = f(h) + f(g) \} = \Hom_\Z(G, \, A). \]
Accettiamo il risultato come vagamente interessante e procediamo.

\section{Funtorialità}
I gruppi di coomologia sono funtoriali nei coefficienti $ A $ per costruzione. Siamo ora interessati a capire che tipo di relazioni otteniamo al variare del gruppo $ G $ che agisce. Un problema che si pone immediatamente, seppur puramente formale, è che cambiando gruppo usciamo dalla categoria ambiente: abbiamo quindi bisogno del cugino del funtore dimenticante.

\begin{definition}
	Dati due gruppi finiti $ G $ e $ G' $, un omomorfismo $ f \colon G' \to G $ e uno $ G $ modulo $ A $, chiamiamo $ f^\times A $ il gruppo abeliano $ A $ equipaggiato dell'azione
	\[ g' \cdot_{G'} A := f(g') \cdot_{G} A \qquad \forall\, g' \in G'. \]
\end{definition}

A cui diamo questo simpatico appellattivo perché nel caso a cui saremo principalmente interessati, ovvero passare da $ G $ a un suo sottogruppo $ H $ tramite l'immersione naturale $ i \colon H \to G $, il funtore $ f^\times $ semplicemente \textquotedblleft dimentica" come gli elementi fuori da $ H $ agiscono sul modulo in questione. In queto particolare caso l'operazione è, in un qualche senso, parzialmente invertibile:

\begin{definition}
	Preso un sottogruppo $ H < G $ e un $ H $-modulo $ A $, chiamiamo modulo indotto il gruppo abeliano
	\[ \Ind_H^G(A) := \{ f \colon G \to A \mid f(hg) = hf(g) \; \forall\, h \in H \} = \Hom_H(G, \, A), \]
	munito dell'azione 
	\[g \cdot f(x) = f(xg) \qquad\forall\, g \in G. \]
	
\end{definition}



\begin{proposition}\label{aggiunzione}\todo[dimostrare, introdurre]
	I due funtori
	\[ f^\times \colon \Gmod \to \Gmod \qquad\text{ e }\qquad  \Ind_H^G \colon \Gmod \to \Gmod \]
	sono aggiunti. Ovvero, per ogni $ A \in \Gmod $ e $ B \in \Gmod $, soddisfano
	\[ \Hom_G(A,\, \Ind_H^G(B)) = \Hom_H(f^\times A, \, B). \]
\end{proposition}

Per rimanere in tema, ribadiamo che $ f^\times $ è un'operazione puramente formale che usiamo per sottolineare quando stiamo pensando a un dato gruppo abeliano con un'azione diversa da quella naturale con il quale è stato introdotto, dunque inizieremo presto a omettere il funtore per sempplificare la notazione.

Siamo ora pronti per affrontare il risultato principale, che ci permetterà di costruire tutte le mappe di cui avremo bisogno in seguito.
\begin{proposition}\label{funct}
	Siano  $ A $ uno $ G $ modulo e $ A' $ uno $ G' $ modulo. Dati $ f \colon G' \to G $ un omomorfismo di gruppi e $ u \colon f^\times A \to A' $ un omorfismo $ G' $-equivariante, otteniamo una mappa in coomologia
	\[ \HH^i(G, \, A) \to \HH^i(G', \, A') \qquad\forall i \geq 0. \]
\end{proposition}

\begin{proof}
	Osserviamo che, per definizione di $ f^\times A $, abbiamo un'inclusione
	\[ A^G \hookrightarrow \left(f^\times A\right)^{G'} \]
	che vogliamo estendere a un morfismo tra risoluzioni iniettive. Presa una risoluzione iniettiva di $ A $
	\[ 0 \to A \to I^{\,0} \to I^{\,1} \to I^{\,2} \to \dots, \]
	pocihé il funtore $ f^\times $ preserva gli iniettivi, anche
	\[ 0 \to f^\times A \to f^\times I^{\,0} \to f^\times I^{\,1} \to f^\times I^{\,2} \to \dots \]
	è una risoluzione iniettiva. Ci troviamo quindi nella seguente situazione:
	\[\begin{tikzcd}[column sep = small]
	0 \rar & A^G \rar \dar & \left(I^{\,0}\right)^G \rar & \left(I^{\,1}\right)^G \rar & \left(I^{\,2}\right)^G \rar & \dots \\
	0 \rar & \left(f^\times A\right)^{G'}  \rar & \left(f^\times I^{\,0}\right)^{G'} \rar & \left(f^\times I^{\,1}\right)^{G'} \rar & \left(I^{\, 2}\right)^{G'} \rar & \dots
	\end{tikzcd} \]
	Per definizione di modulo iniettivo otteniamo tutte le mappe successive \todo[la sto facendo più facile del dovuto?]
	\[\begin{tikzcd}[column sep = small]
	0 \rar
	& A^G \dar \rar \arrow[dashed, "comp." description]{dr}
	& \left(I^{\,0}\right)^G \rar \dar[dashed, "inj." description] \arrow[dashed, gray]{dr}
	& \left(I^{\,1}\right)^{G} \rar \dar[dashed, gray] \arrow[dashed, gray!50]{dr}
	& \left(I^{\, 2}\right)^{G} \dar[dashed, gray!50] \rar
	& \dots \\
	0 \rar
	& \left(f^\times A\right)^{G'} \rar
	& \left(f^\times I^{\,0}\right)^{G'} \rar
	& \left(f^\times I^{\,1}\right)^{G'} \rar
	& \left(f^\times I^{\,2}\right)^{G'} \rar
	& \dots
	\end{tikzcd} \]
	e un morfismo di complessi tra risoluzioni iniettive passa al quoziente, fornendoci una mappa
	\[ \HH^i(G, \, A) \to \HH^i(G', \, f^\times A) \qquad \forall i \geq 0. \]
	Abbiamo già osservato che $ \HH^i $ è funtoriale nel secondo argomento, dunque otteniamo la tesi componendo la mappa appena trovata con quella indotta da $ u $.
\end{proof}

Affrontato il problema nella dovuta generalità, decliniamo la soluzione nei casi a cui siamo interessati: fissiamo $ H $ un sottogruppo di $ G $.
\begin{definition}[Restrizione]
	Chiamiamo mappa di restrizione l'omomorfismo
	\[ \Res \colon \HH^i(G, \, A) \to \HH^i(H, \, f^\times A) \]
	indotto da $ i \colon H \to G $.
\end{definition}

\begin{definition}[Inflazione]
	Chiamiamo mappa d'inflazione l'omomorfismo
	\[ \Inf \colon \HH^i(G/H, \, A^H) \to \HH^i(G, \, A) \]
	indotto dalla proiezione natuarle $ \pi \colon G \to G/H $ e dall'inclusione $ u \colon A^H \to A $.
\end{definition}

Avendo sopra introdotto i moduli indotti, si sarebbe potuta fare un'altra scelta naturale nel tentativo di definire una mappa di restrizione, anch'essa di un certo interesse: abbiamo un omomorfismo \[ \HH^i(G, \, \Ind_H^G(B)) \to \HH^i(H, \, B) \] indotto dall'inclusione naturale $ i \colon H \to G $ e dal morfismo $ u \colon \Ind_H^G(B) \to B $ che valuta ogni mappa nell'elemento neutro $ f \mapsto f(1) $. 
\begin{proposition}[Lemma di Shapiro]
	L'omomorfismo sopra definito
	\[ \HH^i(G, \, \Ind_H^G(B)) \to \HH^i(H, \, B) \]
	è un isomorfismo.
\end{proposition}
\begin{proof}
	Per l'aggiunzione del funtore $ \Ind_H^G $ con il funtore dimenticante $ f^\times $ (\ref{aggiunzione}), abbiamo che
	\[ \Hom_G(\Z,\, \Ind_H^G(B)) = \Hom_H(f^\times \Z, \, B), \]
	ma, essendo $ \Z $ già munito dell'azione banale, scopriamo che i due funtori $ \HH^i(G, \, \Ind_H^G(\,\bullet\,)) $ e $ \HH^i(H, \, \bullet\,) $ sono il derivato dello stesso $ \Hom_H(\Z, \,\bullet \,) $ e pertanto coincidono.
\end{proof}

Portiamo all'attenzione un'ultima rilevante costruzione: lasciamo agire $ G $ su se stesso per coniugio, fissato un elmento $ t \in G $ otteniamo un automorfismo interno $ f_t \colon G \to G $. Questo induce un automorfismo di $ \HH^i(G, \, A) $ che riusciamo a descrivere esplicitamente.

\begin{proposition}
	L'isomorfimo indotto dal coniugio per $ t $
	\[ \sigma_t \colon \HH^i(G, \, A) \to \HH^i(G, \, A) \]
	è l'identità.
\end{proposition}
\begin{proof}
	Dimostriamo il risultato per induzione su $ i $: per $ i = 0 $ non c'è nulla da dimostrare. Supponiamo ora la tesi vera fino al grado $ i $ per ogni $ G $ modulo. Preso ora un qualunque modulo $ A $, immergiamolo in modulo iniettivo $ I $:
	\[ 0 \to A \to I \to Q \to 0. \]
	Applicando $ \mathcal{F} $ otteniamo una successione esatta lunga (per \ref{fond}) costellata di $ 0 $ ovunque si prenda la coomologia di $ I $ (per \ref{injban}), rimangono quindi solo il segmento iniziale
	\[\begin{tikzcd}
	0 \rar & A^G \rar\dar["\sigma_t = \id" description] & I^G \rar\dar["\sigma_t = \id" description] & Q^G \rar\dar["\sigma_t = \id" description] & \HH^1(G, \, A) \rar\dar["\sigma_t"] & 0 \\
	0 \rar & A^G \rar & I^G \rar & Q^G \rar & \HH^1(G, \, A) \rar & 0
	\end{tikzcd}\]
	e alcuni frammenti per $ i > 0 $
	\[\begin{tikzcd}
	0 \rar & \HH^i(G, \, Q) \rar\dar["\sigma_t = \id" description] & \HH^{i+1}(G, \, A) \rar\dar["\sigma_t"] & 0 \\
	0 \rar & \HH^i(G, \, Q) \rar & \HH^{i+1}(G, \, A) \rar & 0,
	\end{tikzcd}\]
	da cui deduciamo la tesi in grado $ i+1 $.
\end{proof}
\newpage
\section{Successione Spettrale di Hochschild-Serre}

Questa sezione richiederà parecchio tempo, enuncio i due risultati

\begin{theorem}[Hochschild-Serre]
	Associata ad $ H $ sottogruppo normale di $ G $, si ha una successione spettrale
	\[ E^{pq}_2 = \HH^p(G/H, \, \HH^q(H, \, A)) \;\Rightarrow\; \HH^{p+q}(G, \, A). \]
	
\end{theorem}

\begin{corollary}[Succesione Restrizione-Inflazione]
	Associata ad $ H $ sottogruppo normale di $ G $, si ha una successione esatta
	\[\begin{tikzcd}[column sep = small]
	0 \rar & \HH^1(G/H, \, A^H) \rar["\Inf"]
	& \HH^1(G, \, A) \rar["\Res"]
	& \HH^1(H, \, A)^{G/H} \rar["\texttt{trs}"]
	& \HH^2(G/H, \, A^H) \rar["\Inf"]
	& \HH^2(G, \, A).
	\end{tikzcd} \]
\end{corollary}

\begin{corollary}[Succesione Restrizione-Inflazione]
	Associata ad $ H $, sottogruppo normale di $ G $ per cui $ \HH^i(H, \, A ) = 0 $ per ogni $ 1 \leq i < q $, si ha la successione esatta
	\[\begin{tikzcd}[column sep = small]
	0 \rar & \HH^q(G/H, \, A^H) \rar["\Inf"]
	& \HH^q(G, \, A) \rar["\Res"]
	& \HH^q(H, \, A).
	\end{tikzcd} \qquad  \]
\end{corollary}

\section{Corestrizione}
Costruiamo ora un'interessante mappa che va nella direzione opposta rispetto alla restrione, ovvero che in qualche modo risale dalla coomologia rispetto ad $ H $ a quella rispetto a $ G $. Per costruire questa mappa è fondamentale che il gruppo $ G $ sia finito, cosa che abbiamo comodamente assunto all'inizio del capitolo.

\begin{definition}[Corestrizione]
Fissiamo uno $ G $ modulo $ A $ e un sottogruppo $ H < G $. In grado $ i = 0 $ definiamo la corestrizione tra $ A^H \to A^G $  come la \emph{norma}
\[N_{G/H} \colon a \mapsto \sum_{x \in G/H} x \cdot a, \]
che, oltre a essere ben definita, sia perché $ G/H $ è finito che perché l'azione di $ G $ su $ A^H $ coincide su ogni laterale di $ H $, ricorda in qualche modo l'omonimo operatore tra estensioni finite di campi.

Per estendere la mappa a tutti i gradi, in memoria di quanto fatto l'ultima volta, prendiamo una risoluzione iniettiva di $ A $, prendiamo gli invarianti sia rispetto ad $ H $ che a $ G $ e, prima di prendere l'omologia, applichiamo la norma per ottenere un morfismo di complessi \todo[perché è un morismo di complessi?]
\[\begin{tikzcd}
0 \rar
& \left(I^{\,0}\right)^H \rar \dar["N"]
& \left(I^{\,1}\right)^H \rar \dar["N"]
& \left(I^{\,2}\right)^H \rar \dar["N"]
& \dots \\
0 \rar
& \left(I^{\,0}\right)^{G} \rar
& \left(I^{\,1}\right)^{G} \rar
& \left(I^{\,2}\right)^{G} \rar
& \dots
\end{tikzcd} \]
e quindi una mappa in coomologia che chiamiamo \emph{corestrizione}:

\[ \Cor \colon \HH^i(H, \, A) \to \HH^i(G, \, A). \]

\end{definition}

Naturalmente, la nostra prima preoccupazione è di controllare come la corestrizione rispetto alla restrizione: osserviamo che componendo le due mappe otteniamo un endomorfismo di, a seconda dell'ordine, $ \HH^i(G, \, A) $ oppure $ \HH^i(H, \, A) $. Esplicitando la definizione di entrambe riusciamo a descrivere questi omomorfismi esplicitamente.

\begin{theorem}[della corestrizione]
	Sia $ m = [G \,\colon H] $ l'indice di $ H $ in $ G $. La composizione
	\[\begin{tikzcd}
	\HH^i(G, \, A) \rar["\Res"] & \HH^i(H, \, A) \rar["\Cor"] & \HH^i(G, \, A)
	\end{tikzcd}  \]
	coincide con la moltiplicazione per $ m $.
\end{theorem}
\begin{proof}
	La dimostrazione è per induzione sul grado: in $ i = 0 $ ritroviamo l'azione della norma su elementi fissati dall'azione di $ G $
	\[\begin{tikzcd}
	A^G \rar["i"] & A^H \rar["N"] & A^G & \\
	a \rar[mapsto] & a \rar[mapsto] & \displaystyle\sum_{x \in G/H} x \cdot a = ma.
	\end{tikzcd}  \]
	Supponiamo ora la tesi vera fino al grado $ i $ per ogni $ G $ modulo. Preso ora un qualunque modulo $ A $, immergiamolo in modulo iniettivo $ I $:
	\[ 0 \to A \to I \to Q \to 0. \]
	Applicando $ \mathcal{F} $ otteniamo una successione esatta lunga (per \ref{fond}) costellata di $ 0 $ ovunque si prenda la coomologia di $ I $ (per \ref{injban}), rimangono quindi solo il segmento iniziale
	\[\begin{tikzcd}
	0 \rar
	& A^G \rar\dar["\cdot m" description]
	& I^G \rar\dar["\cdot m" description] 
	& Q^G \rar\dar["\cdot m" description] 
	& \HH^1(G, \, A) \rar\dar["\Cor \cdot \Res"] 
	& 0 \\
	0 \rar 
	& A^G \rar 
	& I^G \rar 
	& Q^G \rar 
	& \HH^1(G, \, A) \rar 
	& 0
	\end{tikzcd}\]
	e alcuni frammenti per $ i > 0 $
	\[\begin{tikzcd}
	0 \rar & \HH^i(G, \, Q) \rar\dar["\cdot m" description] & \HH^{i+1}(G, \, A) \rar\dar["\Cor \cdot \Res"] & 0 \\
	0 \rar & \HH^i(G, \, Q) \rar & \HH^{i+1}(G, \, A) \rar & 0,
	\end{tikzcd}\]
	da cui deduciamo la tesi in grado $ i+1 $.
\end{proof}

Abbiamo ottenuto un risultato molto piacevole: componendo due mappe particolarmente misteriose, definite in modo bizzarro e algebricamente macchinoso, agiamo sugli elementi in modo elementare. Questo trucco edifica un meraviglioso ponte tra l'algebra omologica e l'artimetica, che possiamo attaversare per tornare indietro con un paio di meravigliosi risultati; per esempio, prendedo il sottogruppo banale $ H = 1 $ la composizione di restrizione e corestrizione
\[\begin{tikzcd}
\HH^i(G, \, A) \rar["\Res"] & \HH^i(H, \, A) = 0 \rar["\Cor"] & \HH^i(G, \, A)
\end{tikzcd}  \]
è necessariamente nulla, dunque:

\begin{corollary}
	Sia $ n = [G 	\,\colon 1] $ l'ordine di $ G $. I gruppi di coomologia $ \HH^i(G, \, A) $ sono di $ n $-torsione.
\end{corollary}

Iniziamo ad avere qualche informazione importante su come sono fatti questi gruppi! Partendo da un gruppo $ A $ finitamente generato, pure $ \HH^i(G, \, A) $ sarà finitamente generato su $ \Z $ (per la descrizione esplicita in cocatene) e dunque finito, perché finitamente generato e di torsione.

Un secondo risultato che riusciamo a ricavare dall'aritmetica, appena più raffinato del precedente, lo otteniamo restringendo la nostra attenzione a una componente $ p $-primaria.

\begin{lemma}
	Se $ p $ è un primo per cui $ m = [G \, \colon H] $ è coprimo con $ p $, allora $ \Res $ è iniettiva sulla componente $ p $-primaria di $ \HH^i(G, \, A) $.
\end{lemma}

Sotto queste ipotesi, infatti, la composizione $ \Cor \cdot \Res $ si restringe ad una mappa iniettiva, dunque la mappa più interna, la restrizione, dev'essere iniettiva a sua volta.

\section{Gruppi di Tate}
La coomologia di gruppi introdotta fino a questo momento è accompagnata, poco sorprendentemente, da una duale teoria omologica. I due strumenti si riescono a mettere in comunicazione fra loro e, modificati opportunamente, a riunire in un unico grande macchinario: i gruppi di coomologia di Tate, $ \Hh^i(G, \, A) $. \\

Consideriamo il funtore $ \mathcal{G} $ che preso un modulo, restituisce il più grande quoziente su cui $ G $ agisce banalmente, detto modulo dei co-invarianti:
\begin{align*} 
\mathcal{G} \colon \Gmod &\to \mathcal{A}b \\
A &\mapsto A_G.
\end{align*}
Riusciamo a descrivere $ \mathcal{G} $ esplicitamente oservando che il nucleo della proiezione $ A \to A_G $ dev'essere il sottomodulo generato da $ \{ a - ga \mid a \in A, \, g \in G \} $, infatti quozientare per quest'ultimo corrisponde esattamante a imporre che l'azione sia banale. Nel caso dell'anello di gruppo otteniamo il cosidetto \emph{ideale di augemntazione}:
\[ 0 \to I_G \to \Z[G] \to \Z \to 0, \]
che ci fornisce una descrizione semplice del funtore in analisi
\[ \mathcal{G} \colon A \to A_G = A/I_GA = A \otimes_G \Z. \]
Averlo descritto come tensore ci dice, in particolare, che $ \mathcal{G} $ è esatto a destra ma, in generale, non a destra. Possiamo quindi prenderne il derivato sinistro:

\begin{definition}[Omologia di Gruppi]
	Dato un grupo finito $ G $ e un intero $ i \geq 0 $, chiamiamo $ i $-esimo gruppo di omologia di $ G $ il funtore
	\[ \HH_i(G, \, \bullet \,) := L^i\mathcal{G}(\, \bullet \,) = \Tor_G(\Z, \, \bullet \, ). \]
\end{definition}

Otteniamo così un funtore omologico, per cui vale il duale di praticamente ogni teorema enunciato fin'ora per il corrispondente funtore coomologico: ad una successione esatta corta sarà associata una lunga in omologia, potremo definire delle opportune mappe di restrizione, inflazione e corestrizione... Le dimostrazioni di questi teoremi si ottengono dualizzando quelle presentate, dunque non saremo così diligenti da risciverle: ci limitamo ad incoraggiare il lettore interessato a rileggerle, sostituendo "iniettivo" con "proiettivo" e guardando i diagrammi attraverso uno specchio. \todo[la faccenda del relativamente iniettivo] \\

Occupiamoci piuttosto del collegamento tra i due strumenti. L'obiettivo principale è quello di raccordare le due successioni esatte lunghe in una successione lunghissima, illimitata in entrambe le direzioni. L'unico problema che si presenta è che, al momento, entrambe le successioni terminano in una direzione, dovremo quindi modificare i primi elementi di ciascuna. Per fare questo iprendiamo la norma su $ A $ associata al sottogruppo banale
\begin{align*}
N \colon A  & \to A^G \\
a & \mapsto \sum_{g \in G} \, g \cdot a
\end{align*}
e osserviamo che passa al quoziente $ N \colon A/I_G \to A^G $. 

\begin{definition}[Gruppi di Tate]
	Questa definizione è quasi esclusivamente un cambio di notazione: i gruppi di grado positivo saranno i coomologici, mentre quelli di grado negativo saranno gli omologici
	\[ \Hh^i(G, \, A) = \HH^i(G, \, A) \quad \forall\, i > 0, \qquad\quad\qquad \Hh^{-i}(G, \, A) = \HH_{i-1}(G, \, A) \quad \forall\, i > 1;
	 \]
	 definiamo invece i gruppi di Tate di grado $ i = -1, \, 0 $ come $ \ker $ e $ \coker $ della norma:
	 \[ \begin{tikzcd}[column sep = small]
	 0 \rar& \Hh^{-1}(G, \, A) \rar& A_G \rar["N"]& A^G \rar&\Hh^0(G, \, A) \rar& 0.
	 \end{tikzcd} \]
\end{definition}

Sistemati i dettagli ad arte enunciamo il risultato principale.

\begin{theorem}\label{fond} \todo[questa si può fare meglio]
	Data una successione esatta corta di $ G $ moduli
	\[\begin{tikzcd}[column sep = small]
	0 \rar & A \rar & B \rar & C \rar & 0,
	\end{tikzcd}\]
	si ha una successione esatta lunga di gruppi abeliani
	\[\begin{tikzcd}[column sep = small, row sep = small]
	\dots \rar & \Hh^{-2}(G, \, C) \rar &\Hh^{-1}(G, \, A) \rar & \Hh^{-1}(G, \, B) \arrow[d,phantom, ""{coordinate, name=W}] \rar & \Hh^{-1}(G, \, C) \arrow[dll, 
	rounded corners = 7 pt, 
	to path={ --([xshift=3ex]\tikztostart.east)
		|- (W)[near end]\tikztonodes
		-| ([xshift=-3ex]\tikztotarget.west)
		-- (\tikztotarget)}] && \\
	&&\Hh^0(G, \, A) \rar & \Hh^0(G, \, B) \rar & \Hh^0(G, \, C) \rar
	 & \Hh^1(G, \, A) \rar & \dots \end{tikzcd}\]
\end{theorem}
\begin{proof}
	La dimostrazione è una delicata operazione di incollamento: avviciniamo le due estremità che desideriamo attaccare
	\[\begin{tikzcd}[column sep = small, row sep = small]
	\dots \rar
	& \Hh^{-2}(G, \, C) \rar["\delta_0"]
	& A_G \rar
	& B_G \rar
	& C_G \rar
	& 0 & \\
	& 0 \rar
	&A^G \rar
	& B^G \rar
	& C^G \rar["\delta^0"]
	& \Hh^1(G, \, A) \rar
	& \dots \end{tikzcd}\]
	e colleghiamole provvisoriamente tramite la norma
	\[\begin{tikzcd}[column sep = small]
	&& \Hh^{-1}(G, \, A) \dar 
	& \Hh^{-1}(G, \, B) \dar 
	& \Hh^{-1}(G, \, C) \dar
	&& \\
	\dots \rar
	& \Hh^{-2}(G, \, C) \rar["\delta_0"]
	& A_G \rar \dar{N} 
	& B_G \rar \dar{N}
	& C_G \rar \dar{N}
	& 0 & \\
	& 0 \rar
	&A^G \rar\dar
	& B^G \rar\dar
	& C^G \rar["\delta^0"]\dar
	& \Hh^1(G, \, A) \rar
	& \dots \\
	&& \Hh^{0}(G, \, A) 
	& \Hh^{0}(G, \, B)  
	& \Hh^{0}(G, \, C)
	&&
	\end{tikzcd}\]
	I quadrati che si sono formati aggiungendo la norma commutano: la norma coincide con la moltiplicazione per l'elemento $ \sum_{g \in G} g $ di $ \Z[G] $ e il morfismo $ f \colon A \to B $ è $ \Z[G] $-lineare per ipotesi.
	\[\begin{tikzcd}
	A_G \rar["f_*"] \dar{N} 
	& B_G \arrow[d, "N"]\\
	A^G \rar["f^*"]
	& B^G
	\end{tikzcd}\]
	Inoltre $ N\delta_0 = 0 $: infatti $ Nf_*\delta_0 = 0 $ per esattezza della prima riga e coincide con $ f^*N\delta_0 = Nf_*\delta_0 = 0 $ per commutatività del quadrato sopra; $ f^* $ è iniettiva per esattezza della seconda riga, da cui il claim. Abbiamo dunque scoperto che $ \delta_0 $ ha immagine in $ \ker N $, ovvero spezza attraverso la mappa curva qua sotto:
	\[\begin{tikzcd}[column sep = small]
	& \Hh^{-1}(G, \, A) \dar
	& \Hh^{-1}(G, \, B) \dar \\
	\Hh^{-2}(G, \, C) \rar["\delta_0"] \arrow[ur, dashed, bend left = 30]
	& A_G \rar["f_*"] \dar{N} 
	& B_G \arrow[d, "N"]\\
	0 \rar
	&A^G \rar["f^*"]
	& B^G
	\end{tikzcd}\]
	Analogamente si trova una mappa $ \Hh^0(G, \, C) \to \Hh^1(G, \, A) $. Possiamo infine invocare lo Snake Lemma per costruire tutte le mappe di collegamento:
	\[\begin{tikzcd}[column sep = small]
	&& \Hh^{-1}(G, \, A) \dar \rar[dashed]
	& \Hh^{-1}(G, \, B) \dar \rar[dashed]
	& \Hh^{-1}(G, \, C) \dar \arrow[dddll, dashed]
	&& \\
	\dots \rar
	& \Hh^{-2}(G, \, C) \rar["\delta_0"] \arrow[ur, dashed, bend left = 30]
	& A_G \rar \dar{N} 
	& B_G \rar \arrow[d, "N"]
	& C_G \rar \dar{N}
	& 0 & \\
	& 0 \rar
	&A^G \rar\dar
	& B^G \rar\dar
	& C^G \rar["\delta^0"]\dar
	& \Hh^1(G, \, A) \rar
	& \dots \\
	&& \Hh^{0}(G, \, A) \rar[dashed]
	& \Hh^{0}(G, \, B)  \rar[dashed]
	& \Hh^{0}(G, \, C), \arrow[ur, dashed, bend right = 30]
	&&
	\end{tikzcd}\]
	Da cui la tesi.
\end{proof}

