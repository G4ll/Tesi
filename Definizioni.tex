\usepackage[usenames,dvipsnames]{color}
\usepackage{amsmath}%
\usepackage{amsfonts}%
\usepackage{amssymb}%
\usepackage{amsthm}%
\usepackage{graphicx}
\usepackage{multirow}
\usepackage[utf8]{inputenc}
\usepackage[italian]{babel}
\usepackage{mathrsfs}
\usepackage{array}
\usepackage{rotating}
\usepackage{multirow}
%\usepackage{geometry}
\usepackage{enumitem}
\usepackage[normalem]{ulem}


\usepackage{titlesec}
\usepackage{titling}
%\usepackage{fontspec}

\usepackage{xcolor}

% Per disegnare diagrammi commuatativi
\usepackage{tikz-cd}
\usepackage{tikz}

\setcounter{secnumdepth}{5}
\setlength\extrarowheight{3pt}

\usepackage{hyperref}

\hypersetup{
    colorlinks,
    citecolor=black,
    %filecolor=black,
    linkcolor=black,
    urlcolor=black
}

\setlrmarginsandblock{3.5cm}{4cm}{*}
%\setulmarginsandblock{3.5cm}{4.5cm}{*}

% Comando folle
%\renewcommand{\textbf}{\bfseries\textsf}


% Titolo dei capitoli
%\renewcommand*{\chaptitlefont}{\textsf\HUGE\bfseries\sffamily}

% Specify different font for section headings
%\titleformat*{\section}{\LARGE\textsf\bfseries}
%\titleformat*{\subsection}{\Large\textsf}
%\titleformat*{\subsubsection}{\large\textsf}

%\makeheadrule{headings}{\textwidth}{0.3pt}

%\copypagestyle{fnsizeheadings}{headings}
%\makeevenhead{fnsizeheadings}{\thepage}{}{\footnotesize\slshape\leftmark}
%\makeoddhead{fnsizeheadings}{\footnotesize\slshape\rightmark}{}{\thepage}

%-------------------------------------------
\newtheorem{theorem}{$ \blacksquare $ Teorema}[section]
\newtheorem{acknowledgement}[theorem]{Acknowledgement}
\newtheorem{algorithm}[theorem]{Algoritmo}
\newtheorem{axiom}[theorem]{Assioma}
\newtheorem{case}[theorem]{Caso}
\newtheorem{claim}[theorem]{Claim}
\newtheorem{conclusion}[theorem]{Conclusione}
\newtheorem{condition}[theorem]{Condizione}
\newtheorem{conjecture}[theorem]{Congettura}
\newtheorem{corollary}[theorem]{$ \blacktriangledown $ Corollario}
\newtheorem{criterion}[theorem]{Criterio}
\newtheorem{lemma}[theorem]{Lemma}
\newtheorem{notation}[theorem]{Notazione}
\newtheorem{problem}[theorem]{Problema}
\newtheorem{proposition}[theorem]{$ \blacksquare $ Proposizione}
\newtheorem{summary}[theorem]{Riassunto}

\theoremstyle{definition}
\newtheorem{definition}[theorem]{$ \square $ Definizione}
\newtheorem{example}[theorem]{Esempio}
\newtheorem{exercise}{Esercizio}[section]
\newtheorem{solution}[exercise]{Soluzione}

\newcommand{\todo[1]}{\marginpar{\textbf{Rivedere:} \textcolor{red}{#1}}}

\theoremstyle{remark}
\newtheorem{remark}[theorem]{Osservazione}
\newtheorem*{profinite}{Verso il profinito}
\newtheorem*{Profinite}{Verso il profinito e oltre}
\addtolength{\topmargin}{-.5in}
\addtolength{\textheight}{0.5in}

%Comandi specifici
\newcommand{\leftquote}{\textquotedblleft}
\newcommand{\N}{\mathbb{N}}
\newcommand{\Z}{\mathbb{Z}}
\newcommand{\Q}{\mathbb{Q}}
\newcommand{\Qp}{\mathbb{Q}_p}
\newcommand{\R}{\mathbb{R}}
\newcommand{\F}{\mathbb{F}}

% Galois
\newcommand{\K}{K}
\newcommand{\knr}{K_\mathtt{nr}}
\newcommand{\knrr}{{K^\mathtt{nr}_n}}
\newcommand{\f}{f^\times}
\newcommand{\Gal}[1]{\mathcal{G}al\left( #1 \right)}

\DeclareMathOperator{\inv}{inv}
\DeclareMathOperator{\Br}{Br}

% Omologica
\newcommand{\Aut}[1]{\mathrm{Aut}\left( #1 \right)}

\DeclareMathOperator{\Hom}{Hom}

\DeclareMathOperator{\HH}{H}
\DeclareMathOperator{\Hh}{\widehat{H}^\mathnormal{i}}
\DeclareMathOperator{\Hhh}{\widehat{H}}
\DeclareMathOperator{\Hha}{\widehat{H}^\mathnormal{-i}}
\DeclareMathOperator{\Hhb}{\widehat{H}^{\mathnormal{i}+1}}
\DeclareMathOperator{\Hhc}{\widehat{H}^{\mathnormal{i+r}}}
\DeclareMathOperator{\Hhid}{\widehat{H}^{\mathnormal{i}+2}}

\DeclareMathOperator{\Hhq}{\widehat{H}^\mathnormal{q}}
\DeclareMathOperator{\Hhqu}{\widehat{H}^{\mathnormal{q}+1}}
\DeclareMathOperator{\Hhqd}{\widehat{H}^{\mathnormal{q}+2}}
\DeclareMathOperator{\Hhdq}{\widehat{H}^{2-\mathnormal{q}}}

\DeclareMathOperator{\Hhmm}{\widehat{H}^{-2}}
\DeclareMathOperator{\Hhm}{\widehat{H}^{-1}}
\DeclareMathOperator{\Hhz}{\widehat{H}^{0}}
\DeclareMathOperator{\Hhu}{\widehat{H}^{1}}
\DeclareMathOperator{\Hhd}{\widehat{H}^{2}}
\DeclareMathOperator{\Hht}{\widehat{H}^{3}}

\DeclareMathOperator{\Hhnu}{\widehat{H}^{\mathnormal{n}+1}}
\DeclareMathOperator{\Hhnd}{\widehat{H}^{\mathnormal{n}+2}}

\DeclareMathOperator{\Hhnru}{\widehat{H}^{\mathnormal{n}+1+\mathnormal{r}}}
\DeclareMathOperator{\Hhnr}{\widehat{H}^{\mathnormal{n}+\mathnormal{r}}}

\DeclareMathOperator{\Hhn}{\widehat{H}^{\mathnormal{n}}}
\DeclareMathOperator{\Hhnq}{\widehat{H}^{\mathnormal{n}+\mathnormal{q}}}

\DeclareMathOperator{\Hhnp}{\widehat{H}^{\mathnormal{n_p}}}
\DeclareMathOperator{\Hhnpp}{\widehat{H}^{\mathnormal{n_p}+1}}
\DeclareMathOperator{\Hhnppp}{\widehat{H}^{\mathnormal{n_p}+2}}

\DeclareMathOperator{\Hhnpq}{\widehat{H}^{\mathnormal{n_p}+\mathnormal{q}}}
\DeclareMathOperator{\Hhnppq}{\widehat{H}^{\mathnormal{n_p}+\mathnormal{q}+1}}
\DeclareMathOperator{\Hhnpppq}{\widehat{H}^{\mathnormal{n_p}+\mathnormal{q}+2}}


\DeclareMathOperator{\Ind}{Ind}
\DeclareMathOperator{\coInd}{coInd}
\DeclareMathOperator{\Res}{\mathtt{res}}
\DeclareMathOperator{\Cor}{\mathtt{cor}}
\DeclareMathOperator{\Inf}{\mathtt{inf}}

\DeclareMathOperator{\Ext}{Ext}
\DeclareMathOperator{\Tor}{Tor}

\DeclareMathOperator{\im}{im}
\DeclareMathOperator{\id}{id}
\DeclareMathOperator{\cd}{cd}
\DeclareMathOperator{\coker}{coker}

% Cose da decidere
\newcommand{\Gmod}{\mathsf{Mod}_\mathsf{G}}
\newcommand{\Hmod}{\mathsf{Mod}_\mathsf{H}}

\newcommand{\HS}{Hochschild-Serre}


%\renewcommand{\varinjlim}{\textstyle\varinjlim}
%\renewcommand{\varprojlim}{\textstyle\varprojlim}