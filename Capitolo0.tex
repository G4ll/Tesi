\chapter*{Introduzione}
\addcontentsline{toc}{chapter}{Introduzione} % aggiunge all'indice una riga senza numero

La teoria del campo di classe è lo studio delle estensioni abeliane di un campo globale o, nel nostro caso, locale. L'obiettivo principale è la descrizione del gruppo di Galois relativo alla massima estensione abeliana in termini della struttura aritmetica e topologica del campo stesso. In particolare, dimostreremo che questo gruppo è, nel caso locale, il completamento profinito del gruppo moltiplicativo del campo. \\

Affronteremo lo studio di questa teoria per la strada più recente: quella tracciata da John Tate. Introdurremo la coomologia di gruppi come funtore derivato di \leftquote prendere gli invarianti" e passeremo i primi due capitoli a raffinare questo linguaggio, in modo da essere sicuri di avere tutti i vocaboli necessari a discutere la Teoria di Galois. Mostreremo che la riformulazione di alcuni risultati classici attraverso il nostro strumento risulta piacevolmente elegante e sintetica: procederemo dunque in una minuziosa opera di traduzione delle proprietà aritmetiche dei campi locali in termini coomologici. \\

I campi locali presentano una struttura aritmetica relativamente semplice, della quale ci serviremo per calcolare la coomologia del loro gruppo di Galois assoluto. 




% dire che ho copiato tutto dal pdf francese

\vfill\break
