\chapter*{Introduzione}
\addcontentsline{toc}{chapter}{Introduzione} % aggiunge all'indice una riga senza numero

La teoria del campo di classe è lo studio delle estensioni abeliane di un campo globale o, come nel nostro caso, locale. L'obiettivo principale è la descrizione del gruppo di Galois relativo alla massima estensione abeliana in termini della struttura aritmetica e topologica del campo stesso. In particolare, dimostreremo che questo gruppo è, nel caso locale, il completamento profinito del gruppo moltiplicativo del campo. \\

Affronteremo lo studio di questa teoria per la strada più recente: introdurremo la coomologia di gruppi come il derivato del funtore che \leftquote prende gli invarianti"; passeremo i primi due capitoli a raffinare questo linguaggio, in modo da essere sicuri di avere tutti i vocaboli necessari per discutere di teoria di Galois. Mostreremo che la riformulazione di alcuni risultati classici in teoria dei campi, attraverso il nostro strumento, risulta piacevolmente elegante e sintetica: procederemo dunque in una minuziosa opera di traduzione delle proprietà aritmetiche dei campi locali in termini coomologici. \\

I campi locali presentano una struttura aritmetica relativamente semplice, della quale ci serviremo per calcolare la coomologia del gruppo di Galois assoluto. In particolare, le estensioni non ramificate forniscono dei naturali coefficienti sopra i quali la coomologia del gruppo è nulla; permettendoci così di produrre gli isomorfismi necessari a calcolare il gruppo di Brauer e la dimensione coomologica di questi campi. \\

Conclusi i preparativi, presenteremo il prodotto naturale della teoria coomologica in esame; questo induce degli interessanti isomorfismi tra gruppi di coomologia con coefficienti diversi, che possiamo raccogliere in due celebri risultati di dualità. Interpretando opportunamente la dualità di Tate-Nakayama, riusciremo a costruire un'isomorfismo esplicito tra i gruppi di Galois delle estensioni abeliane e alcuni quozienti del gruppo additivo del campo: l'applicazione di reciprocità locale; sfrutteremo invece la dualità di Tate (ma non di Nakayama) per calcolare l'abelianizzato del gruppo di Galois assoluto di un campo locale, concludendo in bellezza lo studio della teoria del campo di classe locale. \\

% Mostreremo infine una possibile applicazione della teoria sviluppata, presentando una rapidissima dimostrazione del teorema di Kronecker-Weber.


% dire che ho copiato tutto dal pdf francese?? magari no

\vfill\break
