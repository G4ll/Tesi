\chapter{i $ p $-adici}


\section{Dualità di Tate}

\begin{definition}[Duale di Pontryagin]
	Per ogni gruppo $ G $, definiamo il duale secondo Pontryagin come
	\[ G^* = \Hom(G, \, \Q/\Z). \]
\end{definition}

\begin{definition}[Duale di Cartier]
	Per ogni $ G $-modulo $ A $, definiamo il duale secondo Cartier come
	\[ A^* = \Hom_G(A, \, \mu_\infty). \]
\end{definition}

% Misteriosa esistenza del modulo dualizzante
\begin{proposition}
	Sia $ G $ un gruppo profinito di dimensione coomologica finita $ \cd(G) = n $. Se ogni modulo $ A $ ha $ n $-esimo gruppo di coomologia $ \HH^n(G, \, A) $ finito, allora il funtore $ \HH^n(G,\, \bullet\,)^* $ è rappresentabile: esiste un modulo $ I $ di torsione per cui
	\[ \HH^n(G, \, \bullet \, )^* = \Hom_G(\,\bullet\, \, I). \]
\end{proposition}

% Lemmino
\begin{lemma}
	Sia $ G $ un gruppo di dimensione coomologica $ n $. Detto $ I $ un suo modulo dualizzante, questo è dualizzante per ogni sottogruppo $ U < G $ aperto.
\end{lemma}
\begin{proof}
	Il risultato segue immediatamente dal Lemma di Shapiro (\ref{Shapiro}), per cui
	\[ \HH^n(U, \, A)^* = \HH^n(G, \, \Ind_U^G(A))^* = \Hom_G(\Ind_U^G(A), \, I) = \Hom_U(A, \, \f I). \]
\end{proof}

% Trovare il dualizzante
\begin{proposition}
	Il modulo dualizzante di $ \Gamma_K $ è canonicamente isomorfo al modulo di tutte le radici dell'unità $ \mu_\infty $.
\end{proposition}

% Dualità di Tate
\begin{theorem}[Dualità di Tate]
	
\end{theorem}

% Calcolo del gruppo di Galois assoluto

\section{Kronecker-Weber}