\chapter{i $ p $-adici}


\section{Dualità di Tate}

\begin{definition}[Duale di Pontryagin]
	Per ogni gruppo $ G $, definiamo il duale secondo Pontryagin come
	\[ G^\vee = \Hom(G, \, \Q/\Z). \]
\end{definition}

\begin{proposition}
	Il biduale di un gruppo profinito è canonicamnete isomorfo all'abelianizzato del gruppo stesso. 
\end{proposition}

\begin{definition}[Duale di Cartier]
	Per ogni $ G $-modulo $ A $, definiamo il duale secondo Cartier come
	\[ A^* = \Hom_G(A, \, \mu_\infty). \]
\end{definition}

\begin{proposition}
	Il biduale di un modulo finito è canonicamente isomorfo al modulo stesso.
\end{proposition}

% Misteriosa esistenza del modulo dualizzante
\begin{proposition}
	Sia $ G $ un gruppo profinito di dimensione coomologica finita $ \cd(G) = n $. Se ogni modulo $ A $ ha $ n $-esimo gruppo di coomologia $ \HH^n(G, \, A) $ finito, allora il funtore $ \HH^n(G,\, \bullet\,)^\vee $ è rappresentabile: esiste un modulo $ I $ di torsione per cui
	\[ \HH^n(G, \, \bullet \, )^\vee = \Hom_G(\,\bullet\, \, I). \]
\end{proposition}

% Lemmino dei p-adici
\begin{lemma}
	Sia $ K $ un campo $ p $-adico e $ \mu_n $ le radici $ n $-esime dell'unità. I gruppi $ \HH^1(K, \, \mu_n) = K^\times / {K^\times}^n $ sono finiti.
\end{lemma}

% Lemmino
\begin{lemma}
	Sia $ G $ un gruppo di dimensione coomologica $ n $. Detto $ I $ un suo modulo dualizzante, questo è dualizzante per ogni sottogruppo $ U < G $ aperto.
\end{lemma}
\begin{proof}
	Il risultato segue immediatamente dal Lemma di Shapiro (\ref{Shapiro}), per cui
	\[ \HH^n(U, \, A)^\vee = \HH^n(G, \, \Ind_U^G(A))^\vee = \Hom_G(\Ind_U^G(A), \, I) = \Hom_U(A, \, \f I). \]
\end{proof}

% Trovare il dualizzante
\begin{proposition}
	Il modulo dualizzante di $ \Gamma_K $ è canonicamente isomorfo al modulo di tutte le radici dell'unità $ \mu_\infty $.
\end{proposition}

% Dualità di Tate
\begin{theorem}[Dualità di Tate]
	Sia $ K $ un campo $ p $-adico e $ M $ un $ \Gamma_K $-modulo finito. Il tazza-prodotto induce un accoppiamento di dualità
	\[ \HH^i(\Gamma_K, \, M) \times \HH^{2-i}(\Gamma_K, \, M^*) \to \HH^2(\Gamma_K, \, \bar{K}^\times) = \Q / \Z \]
	per $ i = 0, \, 1, \, 2 $.
\end{theorem}

% Calcolo del gruppo di Galois assoluto

\begin{corollary}
	$ \Gamma^\mathtt{ab} $ è isomorfo al completamento profinito $ \hat{K}^\times = \varprojlim_n K^\times/ {K^\times}^n $.
\end{corollary}
\begin{proof}
	\begin{align*}
		\Gamma^\mathtt{ab}
		& = \Hom_\Z(\Gamma,\, \Q/\Z)^\vee & \text{per dualità di Pontryagin} \\
		& = \Hom_\Z(\Gamma,\, \varinjlim_n \Z/n\Z)^\vee & \text{per ricondursi a moduli finiti}\\ 
		& = \varprojlim_n \Hom_\Z(\Gamma,\, \Z/n\Z)^\vee & \text{per abstract nonsense} \\
		& = \varprojlim_n \HH^1(\Gamma,\, \Z/n\Z)^\vee & \text{perché l'azione è banale} \\
		& = \varprojlim_n \HH^1(\Gamma,\, \mu_n) & \text{per dualità di Tate} \\
		& = \varprojlim_n \K^\times / {\K^\times}^n & \text{dalla successione esatta lunga} \\
		& = \hat{K}^\times. 
	\end{align*}
\end{proof}

\section{Kronecker-Weber}